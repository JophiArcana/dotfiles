desc: f(g(n))=f(n)+1
source: ISL 2010 A6
tags: [2020-04, oly, hard, alg, fe, arrows, nice, waltz, involved]

---

Let $f,g:\mathbb Z_{>0}\to\mathbb Z_{>0}$ be functions so that \[f(g(n))=f(n)+1\quad\text{and}\quad g(f(n))=g(n)+1\]
for all positive integers $n$. Prove that $f(n)=g(n)$ for all positive integers $n$.

---

First note the following identities:
\begin{itemize}[itemsep=0em]
    \item For each $n$, we have $f(n)\ne n$; else $g(n)=g(f(n))=g(n)+1$. (Similarly $g(n)\ne n$.)
    \item $f(a)=f(b)\implies g(f(a))=g(f(b))\implies g(a)=g(b)$, i.e.\ $f(a)=f(b)\iff g(a)=g(b)$.
    \item $f(f(a))=f(f(b))\iff g(f(a))=g(f(b))\iff f(a)=f(b)$.
\end{itemize}
In particular, if $f(a)=f(b)$ and $a,b\in f_*(\mathbb N)$, then $a=b$. Henceforth we will refer to this property as \emph{pseudo-injectivity}.

The key is to describe the ranges of $f$, $g$, which we will do in the following two claims.
\setcounter{claim}0
\begin{claim}
    For some $j$, $k$, we have $f_*(\mathbb N)=\{j,j+1,\ldots\}$ and $g_*(\mathbb N)=\{k,k+1,\ldots\}$.
\end{claim}
\begin{proof}
Note that if $f(n)\in f_*(\mathbb N)$, then $f(n)+1=f(g(n))\in f_*(\mathbb N)$, and similarly $g(n)\in g_*(\mathbb N)$ implies $g(n)+1\in g_*(\mathbb N)$.
\end{proof}
\begin{claim}
    $f(j)=g(k)$.
\end{claim}
\begin{proof}
    Assume without loss of generality $f(j)\le g(k)$.

    Since $f(f(j))\ne f(j)$, for some $\ell$ we have $f(f(j))=f(\ell)+1=f(g(\ell))$. But $g(\ell)\ge g(k)\ge f(j)$, so $g(\ell)\in f_*(\mathbb N)$. By pseudo-injectivity $f(j)=g(\ell)\ge g(k)$, thus $f(j)=g(k)$.
\end{proof}

Henceforth we let $f(j)=g(k)=c$, so that $f_*(\mathbb N)=g_*(\mathbb N)=\{c,c+1,\ldots\}$. In the next two claims we prove $f(n)=g(n)$ for cofinitely many $n$:
\begin{claim}
    $f(c)=c+1$.
\end{claim}
\begin{proof}
    Assume for contradiction $f(c)-2\ge c$. Then for some $\ell$, we have \[f(c)-2=f(\ell)=f(g(\ell))-1=f(g(g(\ell)))-2.\]
    Note $g(\ell)=f(\ell')$ for some $\ell'$. By pseudo-injectivity \[c=g(g(\ell))=g(f(\ell'))=g(\ell')+1,\]
    so $c-1\in g_*(\mathbb N)$, which is absurd.
\end{proof}
\begin{claim}
    $f(n)=g(n)=n+1$ for $n\ge c$.
\end{claim}
\begin{proof}
    The proof is by induction on $n$, where the base case $n=c$ is the above claim. If $f(n-1)=g(n-1)=n$, then \[f(n)=f(g(n-1))=f(n-1)+1=n+1,\]
    and similarly $g(n)=n+1$, as claimed.
\end{proof}

To finish, note that for each $n$, we have $f(f(n))=f(n)+1=f(g(n))$ by the above claim. By pseudo-injectivity, $f(n)=g(n)$, and we are done.
