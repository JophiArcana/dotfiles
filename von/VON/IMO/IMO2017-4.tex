desc: Parallelogram Reim
source: IMO 2017/4
tags: [2020-03, oly, easy, geo, angle-chasing, parallelogram]

---

Let $R$ and $S$ be different points on a circle $\Omega$ such that $\seg{RS}$ is not a diameter. Let $\ell$ be the tangent line to $\Omega$ at $R$. Point $T$ is such that $S$ is the midpoint of $\seg{RT}$. Point $J$ is chosen on minor arc $RS$ of $\Omega$ so that the circumcircle $\Gamma$ of $\triangle JST$ intersects $\ell$ at two distinct points. Let $A$ be the common point of $\Gamma$ and $\ell$ that is closer to $R$. Line $AJ$ meets $\Omega$ again at $K$. Prove that line $KT$ is tangent to $\Gamma$.

---

\begin{center}
\begin{asy}
    size(9cm); defaultpen(fontsize(10pt));
    pen pri=orange;
    pen sec=lightred;
    pen tri=lightblue;
    pen qua=purple+pink;
    pen fil=yellow+opacity(0.05);
    pen sfil=sec+opacity(0.05);
    pen tfil=cyan+opacity(0.05);
    pen qfil=qua+opacity(0.05);

    pair R,SS,J,T,A,K,B;
    R=dir(270);
    SS=dir(-5);
    J=dir(-20);
    T=2SS-R;
    A=intersectionpoint(circumcircle(J,SS,T),R--R+(2,0));
    K=2*foot(origin,A,J)-J;
    B=2SS-A;

    filldraw(circumcircle(B,T,SS),qfil,qua);
    filldraw(R--A--T--B--cycle,tfil,tri);
    draw(A--B,tri+dashed);
    draw( (-1,T.y)--(3.75,T.y),tri);
    draw( (-1,-1)--(3.75,-1),tri);
    filldraw(circumcircle(J,SS,T),sfil,sec);
    draw(K--T,sec);
    filldraw(circle(origin,1),fil,pri);
    draw(K--A,pri);
    draw(R--T,pri);

    dot("$R$",R,S);
    dot("$S$",SS,dir(5));
    dot("$J$",J,W);
    dot("$T$",T,dir(75));
    dot("$A$",A,SW);
    dot("$K$",K,W);
    dot("$B$",B,NW);
\end{asy}
\end{center}
Let $B$ be the reflection of $A$ over $S$, so that $ARBT$ is a parallelogram. By Reim's theorem on $\Omega$ and $\Gamma$, we have $\seg{RK}\parallel\seg{TA}$, so $R$, $K$, $B$ are collinear. By the converse of Reim's theorem with $\Omega$, we have $TBKS$ cyclic, so \[\da KTA=\da TKB=\da TSB=\da TSA,\]
as desired.
