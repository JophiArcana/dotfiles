desc: a*ceil(bv)-b*floor(av)=m
source: ISL 2013 N7
tags: [2020-03, oly, brutal, nt, rigid, involved, nice]
author: Sam Vandervelde

---

Let $\nu$ be an irrational positive integer, and let $m$ be a positive integer. A pair $(a,b)$ of positive integers is \emph{good} if \[a\lceil b\nu\rceil-b\lfloor a\nu\rfloor=m.\]
A good pair $(a,b)$ is \emph{excellent} if neither the pair $(a-b,b)$ or $(a,b-a)$ is good. Prove that the number of excellent pairs is equal to the sum of the positive divisors of $m$.

---

Let $f(a,b)=a\lceil b\nu\rceil-b\lfloor a\nu\rfloor=a-a\{b\nu\}+b\{a\nu\}$.

It suffices to prove, for any $\nu$, the number of excellent $(a,b)$ with $\gcd(a,b)=1$ is equal to $m$. To see this, let $f_\nu(a,b)$ denote $f(a,b)$ with that particular value of $\nu$; then the number of $(a,b)$ with $\gcd(a,b)=k$ and $f_\nu(a,b)=m$ equals the number of $(a',b')$ with $\gcd(a',b')=1$ and $f_{\nu k}(a',b')=m/k$. Then summing over all $k$ gives the desired result.

Consider the binary tree whose nodes correspond to $(a,b)$ with $\gcd(a,b)=1$. Its root is $(1,1)$, and the left child of $(a,b)$ is $(a,a+b)$ and the right child $(a+b,b)$. By the Euclidean algorithm all $(a,b)$ with $\gcd(a,b)=1$ appear in the tree. Shown below is $\nu=\pi/2$.
\begin{center}
\begin{asy}
    size(14cm); defaultpen(fontsize(10pt));
    pen edge=gray+Dotted;
    real v=pi/2;

    pair[] loc = new pair[128]; // location
    pair[] mark = new pair[128]; // marked ordered pair
    int[] fval = new int[128]; // f-value

    loc[0] = (0,0);
    mark[0] = (1,1);
    fval[0] = 1;

    loc[1] = loc[0] + (-3.2,-1);
    loc[2] = loc[0] + (3.2,-1);
    mark[1] = mark[0] + (0,mark[0].x);
    mark[2] = mark[0] + (mark[0].y,0);
    draw(loc[0] -- loc[1],edge);
    draw(loc[0] -- loc[2],edge);

    for (int i = 1; i < 3; ++i) {
        loc[2*i + 1] = loc[i] + (-1.6,-1);
        loc[2*i + 2] = loc[i] + (1.6,-1);
        mark[2*i + 1] = mark[i] + (0,mark[i].x);
        mark[2*i + 2] = mark[i] + (mark[i].y,0);
        draw(loc[i] -- loc[2*i + 1],edge);
        draw(loc[i] -- loc[2*i + 2],edge);
    }

    for (int i = 3; i < 7; ++i) {
        loc[2*i + 1] = loc[i] + (-0.8,-1);
        loc[2*i + 2] = loc[i] + (0.8,-1);
        mark[2*i + 1] = mark[i] + (0,mark[i].x);
        mark[2*i + 2] = mark[i] + (mark[i].y,0);
        draw(loc[i] -- loc[2*i + 1],edge);
        draw(loc[i] -- loc[2*i + 2],edge);
    }
    for (int i = 7; i < 15; ++i) {
        loc[2*i + 1] = loc[i] + (-0.4,-1);
        loc[2*i + 2] = loc[i] + (0.4,-1);
        mark[2*i + 1] = mark[i] + (0,mark[i].x);
        mark[2*i + 2] = mark[i] + (mark[i].y,0);
        draw(loc[i] -- loc[2*i + 1],edge);
        draw(loc[i] -- loc[2*i + 2],edge);
    }
    { int i = 15;
        loc[2*i + 1] = loc[i] + (-0.3,-1);
        loc[2*i + 2] = loc[i] + (0.3,-1);
        mark[2*i + 1] = mark[i] + (0,mark[i].x);
        mark[2*i + 2] = mark[i] + (mark[i].y,0);
        draw(loc[i] -- loc[2*i + 1],edge);
        draw(loc[i] -- loc[2*i + 2],edge);
    }
    { int i = 30;
        loc[2*i + 1] = loc[i] + (-0.3,-1);
        loc[2*i + 2] = loc[i] + (0.3,-1);
        mark[2*i + 1] = mark[i] + (0,mark[i].x);
        mark[2*i + 2] = mark[i] + (mark[i].y,0);
        draw(loc[i] -- loc[2*i + 1],edge);
        draw(loc[i] -- loc[2*i + 2],edge);
    }
    { int i = 23;
        loc[2*i + 1] = loc[i] + (-0.3,-1);
        loc[2*i + 2] = loc[i] + (0.3,-1);
        mark[2*i + 1] = mark[i] + (0,mark[i].x);
        mark[2*i + 2] = mark[i] + (mark[i].y,0);
        draw(loc[i] -- loc[2*i + 1],edge);
        draw(loc[i] -- loc[2*i + 2],edge);
    }
    { int i = 62;
        loc[2*i + 1] = loc[i] + (-0.3,-1);
        loc[2*i + 2] = loc[i] + (0.3,-1);
        mark[2*i + 1] = mark[i] + (0,mark[i].x);
        mark[2*i + 2] = mark[i] + (mark[i].y,0);
        draw(loc[i] -- loc[2*i + 1],edge);
        draw(loc[i] -- loc[2*i + 2],edge);
    }


    pen[] color = {black, heavyred, orange, purple, fuchsia};
    void drawnode(int i) {
        fval[i] = floor(mark[i].x * ceil(mark[i].y * v)-mark[i].y * floor(mark[i].x * v) + 1e-6);
        label("\tiny$" + (string)mark[i] + "$",loc[i],S);

        int index = 0;
        if (i == 0 || fval[i] != fval[(i - 1) # 2] && fval[i] <= 4) index = fval[i];
        dot(loc[i],color[index]);
        label( (string)fval[i],loc[i],N,color[index]);
    }

    for (int i = 0; i < 31; ++i) drawnode(i);
    drawnode(31); drawnode(32);
    drawnode(47); drawnode(48);
    drawnode(61); drawnode(62);
    drawnode(125); drawnode(126);
\end{asy}
\end{center}

The following two claims will eliminate $\nu$ from the problem.
\setcounter{claim}0
\begin{claim}
    Let $f(a,b)=N$. Then
    \begin{itemize}[itemsep=0em]
        \item if $\{a\nu\}+\{b\nu\}\ge1$, then $f(a,a+b)=f(a,b)+a$ and $f(a+b,b)=f(a,b)$.
        \item if $\{a\nu\}+\{b\nu\}<1$, then $f(a,a+b)=f(a,b)$ and $f(a+b,b)=f(a,b)+b$.
    \end{itemize}
\end{claim}
\begin{proof}
    Note $\{a\nu\}+\{b\nu\}-\{(a+b)\nu\}$ equals $0$ if $\{a\nu\}+\{b\nu\}\le1$ and $1$ otherwise. It is easy to compute
    \begin{itemize}[itemsep=0em]
        \item $f(a,a+b)=a-a\{(a+b)\nu\}+a\{a\nu\}+b\{a\nu\}$.
        \item $f(a+b,b)=a+b-a\{b\nu\}-b\{b\nu\}+b\{(a+b)\nu\}$;
    \end{itemize}
    Hence, we have
    \begin{itemize}[itemsep=0em]
        \item $f(a,a+b)-f(a,b)=a(\{a\nu\}+\{b\nu\}-\{(a+b)\nu\})$;
        \item $f(a+b,b)-f(a,b)=b(1-\{a\nu\}-\{b\nu\}+\{(a+b)\nu\})$.
    \end{itemize}
    This verifies the claim.
\end{proof}
\begin{claim}
    If $f(a,b)=N$, then there is a postive integer $n$ with $f(a,b+an)=N+a$. (Similarly, there is an $m$ with $f(a+bm,b)=N+b$.)
\end{claim}
\begin{proof}
    Assume not. Then $f(a,b+an)=N$ for all $n$. In particular, $\{a\nu\}+\{(b+an)\nu\}<1$ for all $n$, but $\nu$ is irrational, so $\nu$, $2\nu$, $\ldots$ is dense in $(0,1)$, contradiction.
\end{proof}

\begin{claim}
    [Main claim] Let $f(a,b)=N$. Then the number of descendants of $(a',b')$ with $f(a',b')=N+T$ is the number of nonnegative $m$ and $n$ with $T=ma+nb$.
\end{claim}
\begin{proof}
    First we check that if the claim holds for $(a,a+b)$ and $(a+b,b)$, then it holds for $(a,b)$. Without loss of generality $f(a,a+b)=N+a$ and  $f(a+b,a)=N$.

    Consider each representation of $T$ as a linear combination $ma+nb$. Then
    \begin{itemize}[itemsep=0em]
        \item $T=a+(m-n-1)a+n(a+b)$ is in the subtree of $(a,a+b)$ if and only if $m>n$;
        \item $T=m(a+b)+(n-m)b$ is in the subtree of $(a+b,b)$ if and only if $m\le n$.
    \end{itemize}
    Thus it suffices to find cofinitely many base cases. Consider any path starting from $(a,b)$ going down. At some point, there is a point $(a',b')$ such that either $f(a',b')>T$, or both $f(a',b')+a'>T$ and $f(a',b')+b'>T$ since $f(a',b')$ cannot remain constant while one of $a'$, $b'$ is bounded by the second claim.
\end{proof}

Taking $(a,b)=(1,1)$ in the above claim provides the desired conclusion.
\begin{remark}
    We color an edge red if its two endpoints have the same output in $f$, and blue otherwise; for convenience, direct edges from a node to its child.

    After the first two claims, we completely forget $\nu$. The only things of importance are:
    \begin{itemize}
        \item Each node has exactly one red and exactly one blue outgoing edge;
        \item There can be finitely many consecutive red left-child edges, and similarly there can only be finitely many consecutive red right-child edges.
    \end{itemize}
    The third and most critical claim generalizes the original problem statement for smaller subtrees, but the two bullet points above hold for all such subtrees. Thus the reduction by the first two claim implies that the third claim is true, and this motivates the entire solution.
\end{remark}
