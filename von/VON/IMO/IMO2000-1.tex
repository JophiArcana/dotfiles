desc: EP=EQ
source: IMO 2000/1
tags: [2019-11, oly, easy, geo, angle-chasing]

---

Two circles $\omega_1$ and $\omega_2$ intersect at two points $M$ and $N$. Let $\seg{AB}$ be the line tangent to these circles at $A$ and $B$, respectively, so that $M$ lies closer to $\seg{AB}$ than $N$. Let $\seg{CD}$ be the line parallel to $\seg{AB}$ and passing through the point $M$, with $C$ on $\omega_1$ and $D$ on $\omega_2$. Lines $AC$ and $BD$ meet at $E$; lines $AN$ and $CD$ meet at $P$; lines $BN$ and $CD$ meet at $Q$. Show that $EP=EQ$.

---

\begin{center}
    \begin{asy}
        size(9cm);
        defaultpen(fontsize(10pt));
        pen pri=red+linewidth(0.5);
        pen sec=orange+linewidth(0.5);
        pen tri=fuchsia+linewidth(0.5);
        pen fil=red+opacity(0.05);
        pen sfil=orange+opacity(0.05);

        pair A, B, O1, O2, M, NN, C, D, L, P, Q, EE;
        A=(-4, -6);
        B=(3, -4.5);
        O1=extension((-1, 0), (1, 0), A, rotate(90, A)*B);
        O2=extension((-1, 0), (1, 0), B, rotate(-90, B)*A);
        path g1=circle(O1, length(A-O1));
        path g2=circle(O2, length(B-O2));
        M=intersectionpoints(g1, g2)[1];
        NN=intersectionpoints(g1, g2)[0];
        C=intersectionpoint(g1, (M+0.001*(A-B)) -- (M+1000*(A-B)));
        D=intersectionpoint(g2, (M+0.001*(B-A)) -- (M+1000*(B-A)));
        L=extension(A, B, M, NN);
        P=extension(A, NN, C, D);
        Q=extension(B, NN, C, D);
        EE=extension(A, C, B, D);

        filldraw(g1, fil, pri); filldraw(g2, fil, pri);
        fill(EE -- C -- D -- cycle, sfil);
        draw(A -- B, pri);
        draw(EE -- L -- NN, pri);
        draw(C -- EE -- D -- C, pri);
        draw(P -- EE -- Q,  sec);
        draw(A -- NN -- B, sec);
        draw(A -- O1 -- O2 -- B, tri);
        draw(A -- M -- B, pri);

        dot("$A$", A, S);
        dot("$B$", B, S);
        dot("$O_1$", O1, NW);
        dot("$O_2$", O2, NE);
        dot("$C$", C, SW);
        dot("$D$", D, SE);
        dot("$L$", L, S);
        dot("$M$", M, NW);
        dot("$N$", NN, N);
        dot("$P$", P, NW);
        dot("$Q$", Q, dir(150));
        dot("$E$", EE, S);
    \end{asy}
\end{center}
I claim that $E$ and $M$ are reflections across $\seg{AB}$. Indeed, $\da EAB=\da ACM=\da BAM$ and similarly $\da ABE=\da MBA$, so $\triangle EAB\cong\triangle MAB$. Now if $L=\seg{AB}\cap\seg{MN}$, $LA^2=LM\cdot LN=LB^2$, so $L$ is the midpoint of $\seg{AB}$. Since $\seg{PQ}\parallel\seg{AB}$, $M$ is the midpoint of $\seg{PQ}$. But $\seg{ME}\perp\seg{AB}$, so $\seg{ME}$ is the perpendicular bisector of $\seg{PQ}$, as desired.
