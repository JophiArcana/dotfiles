desc: Optimal partitions into parts
source: ISL 2013 C4
tags: [2020-04, oly, tricky, combo, rigid, waltz]

---

Let $n$ be a positive integer, and let $A$ be a subset of $\{1,\ldots,n\}$. An \emph{$A$-partition of $n$ into $k$ parts} is a representation of $n$ as a sum $n=a_1+\cdots+a_k$, where the parts $a_1$, $\ldots$, $a_k$ belong to $A$ and are not necessarily distinct. The \emph{number of different parts} in such a partition is the number of (distinct) elements in the set $\{a_1,a_2,\ldots,a_k\}$.

We say that an $A$-partition of $n$ into $k$ parts is \emph{optimal} if there is no $A$-partition of $n$ into $r$ parts with $r<k$. Prove that any optimal $A$-partition of $n$ contains at most $\sqrt[3]{6n}$ different parts.

---

Let $\sigma(S)$ denote the sum of the elements of $S$.

Say a subset $A$ is $n$-\emph{antidisuninasuboptimal} if there is an optimal $A$-partition of $n$ into $|A|$ parts; that is, all elements of $A$ are used in the optimal $A$-partition.
\setcounter{claim}0
\begin{claim}
    Let $A$ be an $n$-antidisuninasuboptimal set. For any subsets $C$, $D$ of $A$, if $\sigma(C)=\sigma(D)$, then $|C|=|D|$.
\end{claim}
\begin{proof}
    Assume $|C|>|D|$ but $\sigma(C)=\sigma(D)$. Each element of $C$ appears in the optimal $A$-partition, so we may repeatedly remove one instance of each element of $C$ and add an instance of each element of $D$. Eventually the number of parts decreases, and $A$ is not antidisuninasuboptimal.
\end{proof}
\begin{claim}
    For each $0<i\le k$, there are at least $ik-i^2+1$ distinct values of $\sigma(T)$ among $T\subseteq A$ with $|T|=i$.
\end{claim}
\begin{proof}
    Let $A=\{a_1,a_2,\ldots,a_k\}$, with $a_1<a_2<\cdots<a_k$. Start with the subset $T=\{a_1,\ldots,a_i\}$; then repeatedly apply the following process: choose an index $t$ such that $a_t\in T$ but $a_{t+1}\notin T$, and replace $a_t$ with $a_{t+1}$, thereby increasing the sum of the elements of $T$.

    The process terminates when $T=\{a_{k-i+1},\ldots,a_k\}$, and we apply the process $i(k-i)$ times.
\end{proof}

For $n$-antidisuninasuboptimal sets $A$, we may compute \[n\ge\sigma(A)\ge\left\lvert\{\sigma(T):\varnothing\ne T\subseteq A\}\right\rvert=\sum_{i=1}^k\left(ik-i^2+1\right)=\frac{k^3}6+\frac{5k}6>\frac{k^3}6,\]
and we are done.
