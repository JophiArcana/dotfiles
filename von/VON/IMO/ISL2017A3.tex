desc: fgf != gfg then f(f(S))=f(S)
source: ISL 2017 A3
tags: [2020-04, oly, medium, alg, fe, arrows]

---

Let $S$ be a finite set, and let $f:S\to S$ be a function. Suppose that $f\circ g\circ f\ne g\circ f\circ g$ for every function $g:S\to S$ with $g\ne f$. Show that $f(f(S))=f(S)$.

---

Since $S$ is a finite set, $f$ is eventually periodic, say with period $M$. Take $n$ large with $n\equiv-1\pmod M$, so \[f^{n+2}=f^{2n+1}.\]
I contend $f=f^n$. Assume not. Then set $g=f^n$ and note \[f\circ g\circ f=f^{n+2}=f^{2n+1}=g\circ f\circ g,\]
contradiction.

Hence $f$ is a bijection on $f(S)$.
\begin{remark}[Motivation]
    The above solution is algebraic, but this problem is inherently an arrows problem. We consider the directed graph as usual, and task is to show that if $f(f(S))\ne f(S)$, then there is a $g$ with $f\circ g\circ f=g\circ f\circ g$.

    Following the arrows eventually leads to a cycle. The idea is that $f(f(S))\ne f(S)$ means $f(x)$ is not in a cycle for some $x\in S$; i.e.\ there is a tree of height $\ge2$ growing out of the cycle. For instance, see below:
    \begin{center}
        \begin{asy}
            size(6cm); defaultpen(fontsize(10pt));
            pair A,B,C,D,EE,F,G,H;
            A=(-2sqrt(2),0);
            B=(-1sqrt(2),0);
            C=(0,0);
            D=(1,1);
            EE=(1+sqrt(2),1);
            F=(2+sqrt(2),0);
            G=(1+sqrt(2),-1);
            H=(1,-1);

            void arr(pair I, pair J) {
                real t=0.2;
                draw( (I+t*unit(J-I))--(J+t*unit(I-J)),arrow=Arrow(TeXHead));
            }

            arr(A,B);
            arr(B,C);
            arr(C,D);
            arr(D,EE);
            arr(EE,F);
            arr(F,G);
            arr(G,H);
            arr(H,C);

            dotfactor *= 1.2;
            dot("$x$",A,N,filltype=FillDraw(fillpen=white, drawpen=black));
            dot("$y$",B,N,filltype=FillDraw(fillpen=white, drawpen=black));
            dot("$v_1$",C,dir(110),filltype=FillDraw(fillpen=white, drawpen=black));
            dot("$v_2$",D,NW,filltype=FillDraw(fillpen=white, drawpen=black));
            dot("$v_3$",EE,NE,filltype=FillDraw(fillpen=white, drawpen=black));
            dot(F,filltype=FillDraw(fillpen=white, drawpen=black));
            dot("$v_{M-2}$",G,SE,filltype=FillDraw(fillpen=white, drawpen=black));
            dot("$v_{M-1}$",H,SW,filltype=FillDraw(fillpen=white, drawpen=black));
        \end{asy}
    \end{center}
    Then we can pretend the vertices of the tree are actually part of the cycle as follows: for each $x$ a distance $k$ away from the cycle, let $g(x)=v_{M-k+1}$ (with indices modulo $M$).
\end{remark}
