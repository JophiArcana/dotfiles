desc: Tons of equal lengths
source: IMO 2016/1
tags: [2019-11, oly, easy, geo, angle-chasing]

---

Triangle $BCF$ has a right angle at $B$. Let $A$ be the point on line $CF$ such that $FA=FB$ and $F$ lies between $A$ and $C$. Point $D$ is chosen so that $DA=DC$ and $\seg{AC}$ is the bisector of $\angle DAB$. Point $E$ is chosen so that $EA=ED$ and $\seg{AD}$ is the bisector of $\angle EAC$. Let $M$ be the midpoint of $\seg{CF}$. Let $X$ be the point such that $AMXE$ is a parallelogram. Prove that $\seg{BD}$, $\seg{FX}$, and $\seg{ME}$ are concurrent.

---

\begin{center}
    \begin{asy}
        size(7cm); defaultpen(fontsize(10pt));

        pair M,F,C,B,A,D,EE,X;
        M=(0,0);
        F=(-1,0);
        C=(1,0);
        B=dir(110);
        A=F-(length(B-F),0);
        D=reflect(M, circumcenter(A,B,M))*B;
        EE=extension(B,F,D,D+(1,0));
        X=M+EE-A;

        draw(circumcircle(A,B,M),gray);
        draw(circumcircle(B,F,C),gray);
        draw(EE--F,gray);
        draw(M--B--F--D--A--B--C--A--EE--X--M--EE);
        draw(B--D); draw(F--X);

        dot("$M$",M,NE);
        dot("$F$",F,NW);
        dot("$C$",C,E);
        dot("$B$",B,N);
        dot("$A$",A,W);
        dot("$D$",D,S);
        dot("$E$",EE,SW);
        dot("$X$",X,SE);
    \end{asy}
\end{center}
Redefine $D$ as the circumcenter of $\triangle ABC$. With this, $\da ADB=2\da ACB=\da AMB$, so $ABMD$ is cyclic, and $\da CDB=2\da CAB=\da CFB$, so $CBFD$ is cyclic. Since $M$ is the center of $(CBFD)$, $MB=MD$, whence $\seg{AC}$ bisects $\angle DAB$, and $D$ is the point described in the problem.

Note that $\da CAD=\da DAE=\da EDA$, so $\seg{AC}\parallel\seg{DE}$. Since $\da AED=\da EAM=\da EDM$, $AEDM$ is an isosceles trapezoid, id est $E$ lies on $(ABMD)$. Since $\da FBD=\da FCD=\da DAM=\da MAB=\da ABF$, $E$ lies on line $BF$. Furthermore $MD=EA=MX$, so $X$ lies on $(CBFD)$, and by the Incenter-Excenter lemma, $EF=ED$.

To finish, note that $\seg{BD}$ and $\seg{FX}$ are reflections across $\seg{ME}$.
\begin{remark}
    Another finish is to note that $EF=EA=MX$, so $EFMX$ is an isosceles trapezoid and the conclusion follows from Radical Axis theorem on $(ABMDE)$, $(CBFDX)$, and $(EFMX)$.
\end{remark}
