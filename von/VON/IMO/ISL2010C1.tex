desc: Singers and permutations with fixed orders
source: ISL 2010 C1
tags: [2020-04, oly, tricky, combo, free, construction]

---

In a concert, $20$ singers will perform. For each singer, there is a (possibly empty) set of other singers such that he wishes to perform later than all the singers in that set. Can it happen that there are exactly $2010$ orders of the singers such that all their wishes are satisfied?

---

The answer is yes. Here is a construction:
\begin{itemize}[itemsep=0em]
    \item For each $i=1,2,3,4,5$ and $j\ge6$, singer $j$ performs after $i$.
        \begin{itemize}[itemsep=0em]
            \item Singer $2$ performs after $1$.
            \item Singer $4$ performs after $3$.
        \end{itemize}
    \item For each $i,j\in\{6,\ldots,16\}$ with $i<j$, singer $j$ performs after $i$.
    \item Singer $17$ performs after $13$.
    \item Singer $16$ performs after $18$.
    \item For each $i=19,20$ and $j\le19$, singer $i$ performs after $j$.
        \begin{itemize}
            \item Singer $20$ performs after $19$.
        \end{itemize}
\end{itemize}
The first five singers are arranged in $30$ ways, and singers $19$ and $20$ are fixed, along with the order of $1$ through $16$. Thus it remains to place singers $17$ and $18$.

If singer $17$ performs after singer $16$, we have $16$ ways to place singer $18$. Otherwise we have $17$ ways to place singer $18$, so singers $17$ and $18$ are placed in $16+3\cdot17=67$ ways.

Hence there are $30\cdot67=2010$ valid orders, as required.
