desc: Jumping fleas
source: IMO 2000/3
tags: [2020-04, oly, hard, combo, process, invariant, nice, waltz, involved]

---

Let $n\ge2$ be a positive integer and $\lambda$ a positive real number. Initially there are $n$ fleas on a horizontal line, not all at the same point. We define a move as choosing two fleas at some points $A$ and $B$, with $A$ to the left of $B$, and letting the flea from $A$ jump over the flea from $B$ to the point $C$ so that $\frac{BC}{AB}=\lambda$.

Determine all values of $\lambda$ such that, for any point $M$ on the line and for any initial position of the $n$ fleas, there exists a sequence of moves that will take them all to the position right of $M$.

---

The answer is $\lambda\ge\frac1{n-1}$.

\bigskip

\textbf{Proof of lower bound:} We will use this obvious strategy --- the leftmost flea always jumps over the rightmost flea. Consider the nondecreasing sequence $(x_i)_{i\ge1}$ with $0<x_1\le\cdots\le x_n$ the starting coordinates of the fleas; for each $k>n$, let $x_k=x_{k-1}(\lambda+1)-x_{k-n}\lambda$. After $k$ moves, the fleas have coordinates $x_{k+1}$, $\ldots$, $x_{k+n}$. I will prove this sequence is unbounded.

Summing over $x_k-x_{k-1}=\lambda(x_{k-1}-x_{k-n})$, we have \[x_k-x_n=\lambda\left(x_{k-1}+\cdots+x_{k-(n-1)}\right)-\lambda(x_1+\cdots+x_{n-1}).\]
In other words, $x_k-\lambda\left(x_{k-1}+\cdots+x_{k-(n-1)}\right)$ is a fixed constant $C>0$. Then since $x_{k-(n-1)}\le x_{k-i}$ for $0\le i\le n-2$, we have
\begin{align*}
    \frac{x_k+\cdots+x_{k-(n-2)}}{n-1}&\ge\frac{x_k+\cdots+x_{k-(n-1)}}n\\
    &=\frac{C+(\lambda+1)\left(x_{k-1}+\cdots+x_{k-(n-1)}\right)}n\\
    &\ge\frac Cn+\frac{x_{k-1}+\cdots+x_{k-(n-1)}}{n-1},
\end{align*}
so $x_{k-1}+\cdots+x_{k-(n-1)}$ is unbounded, and thus so is $(x_i)_{i\ge1}$.

\bigskip

\textbf{Proof of upper bound:} At some point in time, let the fleas have coordinates $0<x_1\le x_2\le\cdots\le x_n$, and let $I=x_n-\lambda(x_1+\cdots+x_{n-1})$.
\begin{claim*}
    $I$ is nonincreasing.
\end{claim*}
\begin{proof}
    Let $x_i$ jump over $x_j$ to $x_i'$, so that $x_i'=x_j(\lambda+1)-x_i\lambda$, and let $I$ map to $I'$. If $x_i'<x_n$, then \[I'=I-\lambda(x_i'-x_i)=I-\lambda(\lambda+1)(x_j-x_i)<I.\]
    Otherwise $x_i'\ge x_n$, so $x_i'$ is the new maximum and
    \begin{align*}
        I'&=I+x_i'-x_n-\lambda(x_n-x_i)\\
        &=I-x_n(\lambda+1)+(x_i'+\lambda x_i)\\
        &=I-(x_n-x_j)(\lambda+1)\le I,
    \end{align*}
    which proves the claim.
\end{proof}

However if at some point $x_n>M$, then
\begin{align*}
I&=x_n-\lambda(x_1+\cdots+x_{n-1})\\
&>x_n-\lambda(x_n+\cdots+x_n)\\
&=x_n\big[1-(n-1)\lambda\big]\\
&>M\big[1-(n-1)\lambda\big],
\end{align*}
which is larger than $I$ for sufficiently large $M$.
