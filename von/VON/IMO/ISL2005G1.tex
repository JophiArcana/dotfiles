desc: ABKL concyclic when AC+BC=3AB
source: ISL 2005 G1
tags: [2019-11, oly, trivial, geo, homothety, nice, waltz]

---

Let $ABC$ be a triangle satisfying $AC+BC=3AB$. The incircle of $\triangle ABC$ has center $I$ and touches sides $BC$ and $CA$ at points $D$ and $E$, respectively. Let $K$ and $L$ be the reflections of points $D$ and $E$ across $I$. Prove that points $A$, $B$, $K$, $L$ are concyclic.

---

\paragraph{First solution, by Reim's theorem}     Let $M$ and $N$ be the midpoints of $\seg{CB}$ and $\seg{CA}$ respectively. By Pitot's theorem, $\seg{MN}$ is tangent to the incircle, say at a point $T$. Let the $A$-excircle touch $\seg{CB}$ at $D'$ and the $B$-excircle touch $\seg{CA}$ at $E'$. By inspection, $D$, $D'$, $T$ lie on a circle centered at $M$, but by homothety $A$, $K$, $D'$ are collinear. Thus $\angle ATD=\angle D'TD=90\dg=\angle KTD$, so $A$, $K$, $T$ are collinear.
\begin{center}
    \begin{asy}
        size(5cm); defaultpen(fontsize(10pt));
        pair A,B,C,I,M,NN,D,EE,Dp,Ep,T,K,L;
        A=(-2,0);
        B=(5,0);
        C=(0,4sqrt(6));
        I=incenter(A,B,C);
        M=(C+B)/2;
        NN=(C+A)/2;
        D=foot(I,C,B);
        EE=foot(I,C,A);
        Dp=2M-D;
        Ep=2NN-EE;
        T=foot(I,M,NN);
        K=2I-D;
        L=2I-EE;

        draw(D--K,gray);
        draw(L--EE,gray);
        draw(A--Dp,gray);
        draw(B--Ep,gray);
        draw(arc(circumcenter(A,K,B),length(circumcenter(A,K,B)-A),0,180));
        draw(incircle(A,B,C),gray);
        draw(A--B--C--A);
        draw(M--NN);

        dot("$A$",A,W);
        dot("$B$",B,E);
        dot("$C$",C,N);
        dot("$I$",I,S);
        dot("$M$",M,dir(30));
        dot("$N$",NN,W);
        dot("$D$",D,dir(30));
        dot("$E$",EE,W);
        dot("$D'$",Dp,dir(30));
        dot("$E'$",Ep,W);
        dot("$T$",T,N);
        dot("$K$",K,dir(-15));
        dot("$L$",L,dir(210));
    \end{asy}
\end{center}
Similarly $B$, $L$, $T$ are collinear, so $A$, $B$, $K$, $L$ are concyclic by Reim's theorem. To spell it out, $\da AKL=\da TKL=\da MTL=\da ABL$.

\paragraph{Second solution, by Ptolemy's theorem}     Let $\seg{CI}$ intersect $(ABC)$ again at $T$, and let $I_C$ be the $C$-excenter. Note by the Incenter-Excenter Lemma that $TA=TI=TB$, and by Ptolemy's theorem on $ACBT$, \[CT\cdot AB=TI(CA+CB)=3TI\cdot AB\implies CI=2IT=II_C,\]
whence $I_C$ is the reflection of $C$ across $I$. This implies that $\seg{KI_C}$ and $\seg{LI_C}$ are tangent to the incircle, so $\angle IKI_C=\angle ILI_C=90\dg$, and $K$ and $L$ lie on $(AIBI_C)$, as desired.

