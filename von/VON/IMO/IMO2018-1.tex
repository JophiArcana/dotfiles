desc: Reim's on circle at A
source: IMO 2018/1
tags: [2019-10, oly, trivial, geo, parallelogram, angle-chasing]

---

Let $\Gamma$ be the circumcircle of acute triangle $ABC$. Points $D$ and $E$ are on segments $AB$ and $AC$ respectively such that $AD=AE$. The perpendicular bisectors of $\seg{BD}$ and $\seg{CE}$ intersect minor arcs $AB$ and $AC$ of $\Gamma$ at points $F$ and $G$ respectively. Prove that lines $DE$ and $FG$ are either parallel or are the same line.

---

\begin{customenv}{First solution, by constructing parallelograms}
    Construct points $P$ and $Q$ on $\Gamma$ such that $ABFP$ and $ACGQ$ are isosceles trapezoids, and let $M$ and $N$ be the midpoints of minor arcs $AB$ and $AC$ respectively. It is obvious that $M$ and $N$ are the midpoints of arcs $PF$ and $QG$ as well. Since $AP=BF=DF$ and $AQ=CG=EG$, $APFD$ and $AQGE$ are parallelograms.
    \begin{center}
        \begin{asy}
            size(7cm);
            defaultpen(fontsize(10pt));

            pen pri=heavygreen;
            pen sec=chartreuse+linewidth(1);
            pen tri=springgreen+linewidth(1);
            pen qua=lightgreen;
            pen fil=pri+opacity(0.05);
            pen sfil=sec+opacity(0.05);
            pen tfil=tri+opacity(0.05);
            pen qfil=qua+opacity(0.05);

            pair O, A, B, C, M, NN, F, G, D, EE, P, Q;
            O=(0,0);
            A=dir(110);
            B=dir(220);
            C=dir(320);
            M=unit(A+B);
            NN=unit(A+C);
            F=dir(190);
            G=M*NN/F;
            D=2*foot(F,A,B)-B;
            EE=2*foot(G,A,C)-C;
            P=A*B/F;
            Q=A*C/G;

            filldraw(circle(O,1),fil,pri);
            draw(A--B--C--A,pri);
            draw(D--EE,pri);
            draw(P--A--Q,sec);
            draw(B--F--D,sec);
            draw(C--G--EE,sec);
            draw(F--P,tri);
            draw(G--Q,tri);
            draw(D--A--EE,tri);
            fill(A--P--F--D--cycle,tfil);
            fill(A--Q--G--EE--cycle,tfil);
            fill(B--F--D--cycle,sfil);
            fill(C--G--EE--cycle,sfil);
            filldraw(M--NN--G--F--cycle,qfil,qua);

            dot("$A$",A,A);
            dot("$B$",B,B);
            dot("$C$",C,C);
            dot("$D$",D,NW);
            dot("$E$",EE,dir(60));
            dot("$F$",F,F);
            dot("$G$",G,G);
            dot("$M$",M,M);
            dot("$N$",NN,NN);
            dot("$P$",P,P);
            dot("$Q$",Q,Q);
        \end{asy}
    \end{center}
    Note that $PF=AD=AE=QG$, so $\widehat{PF}=\widehat{QG}$ and thus $\widehat{MF}=\widehat{NG}$. It follows that $FGNM$ is an isosceles trapezoid, so $\seg{FG}\parallel\seg{MN}$. But both $\seg{DE}$ and $\seg{MN}$ are perpendicular to the internal angle bisector of $\angle A$, so $\seg{DE}$ and $\seg{FG}$ are parallel, as desired. 
\end{customenv}
\begin{customenv}{Second solution, by angle chasing}
    Let $\seg{FD}$ and $\seg{GE}$ intersect $\Gamma$ again at $X$ and $Y$ respectively. Notice that \[\da AXD=\da AXF=\da ABF=\da DBF=\da FDB=\da XDA,\]
    whence $AX=AD$. Analogously, $AY=AE$, so $D$, $E$, $X$, $Y$ lie on a circle with center $A$. Finally, by Reim's Theorem, $\seg{DE}\parallel\seg{FG}$, as desired.
\end{customenv}
