desc: angles XAB=XBC=XCD=XDA
source: mini IMO 2018/6
tags: [2020-03, oly, medium, geo, angle-chasing, variation]

---

Let $ABCD$ be a quadrilateral and $X$ a point inside $ABCD$ so that \[\angle XAB=\angle XBC=\angle XCD=\angle XDA.\]
\begin{enumerate}[label=(\alph*),itemsep=0em]
    \item Prove that $\angle BXA+\angle DXC=180\dg$.
    \item Prove that $AB\cdot CD=BC\cdot DA$.
\end{enumerate}

---

Let $P=\overline{AB}\cap\overline{CD}$ and $Q=\overline{AD}\cap\overline{BC}$. We have that \[\measuredangle XBQ=\measuredangle XBC=\measuredangle XDA=\measuredangle XDQ,\]
so $Q$ lies on $(BXD)$. Applying a similar argument shows that $P$ lies on $(BXD)$, so $BDPQ$ is cyclic. Consequently, $\measuredangle ABC=\measuredangle PBQ=\measuredangle PDQ=\measuredangle ADC$, so $ABCD$ is cyclic. Then if $\theta=\angle XAB$, then \[\angle BXA+\angle DXC=\big[180^\circ-\theta-(B-\theta)\big]+\big[180^\circ-\theta-(D-\theta)\big]=180^\circ,\]
proving part (a).
\begin{center}
    \begin{asy}
        size(10cm);
        defaultpen(fontsize(10pt));

        pen pri=black;
        pen sec=dashed;
        pen tri=gray;
        pen fil=invisible;
        pen sfil=invisible;
        pen tfil=invisible;

        pair O, A, D, C, B, K, L, Q, R, X;
        O=(0,0);
        A=dir(140);
        B=dir(190);
        C=dir(350);
        D=reflect(O,A+C)*(2*foot(O,B,(A+C)/2)-B);
        Q=extension(A, D, B, C);
        R=extension(A, C, B, D);
        K=extension(A, reflect(A, incenter(A, B, D)) * R, C, reflect(C, incenter(C, B, D)) * R);
        L=extension(B, reflect(B, incenter(B, A, C)) * R, D, reflect(D, incenter(D, A, C)) * R);
        X=reflect(circumcenter(A, B, K), circumcenter(C, D, K)) * K;

        filldraw(A--B--C--D--cycle,fil,pri);
        filldraw(circumcircle(B,Q,D),sfil,sec);
        filldraw(circumcircle(A,B,K),tfil,tri);
        filldraw(circumcircle(C,D,K),tfil,tri);
        draw(C--A--Q--B--D,pri);
        draw(A--K--C,pri);

        clip((Q+(-0.1,-100))--(Q+(-0.1,1.3))--(Q+(100,1.3))--(Q+(100,-100))--cycle);

        dot("$A$",A,N);
        dot("$D$",D,NW);
        dot("$C$",C,SE);
        dot("$B$",B,S);
        dot("$Q$",Q,SW);
        dot("$R$",R,N);
        dot("$K$",K,W);
        dot("$X$",X,E);
    \end{asy}
\end{center}
For part (b), construct point $K$ the intersection of $(AXB)$ and $(CXD)$, and let $R=\seg{AC}\cap\seg{BD}$. Note that \[\da BKX=\da BAX=\da DCX=\da DKX,\]
so $K\in\seg{BD}$. Furthermore \[\da BAK=\da BAX+\da XAK=\da CBX+\da XBK=\da CBK=\da RAD,\]
and similarly $\da BCK=\da RCD$, whence by Steiner \[\left(\frac{AB}{AD}\right)^2=\frac{KB}{KD}\cdot\frac{RB}{RD}=\left(\frac{CB}{CD}\right)^2,\]
and the result follows.
