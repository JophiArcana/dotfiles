desc: Counting roots of P^k(x)=x
source: IMO 2006/5
tags: [2020-02, oly, medium, nt, int-poly]

---

Let $P(x)$ be a polynomial of degree $n>1$ with integer coefficients and let $k$ be a positive integer. Consider the polynomial \[Q(x)=\underbrace{P(P(\ldots P}_{k\text{ times}}(x)\ldots)).\]
Prove that there are at most $n$ integers $t$ such that $Q(t)=t$.

---

Note that \[P(t)-t\mid P^2(t)-P(t)\mid\cdots\mid P^{k+1}(t)-P^k(t)=P(t)-t,\]
whence \[|P(t)-t|=|P^2(t)-P(t)|\cdots=|P^k(t)-P^{k-1}(t)|.\]
Let $P(t)-t=c$; we have that $P^{i+1}(t)=P^i(t)\pm c$ for all $i$. If $P^{i+1}(t)-P^i(t)=c$ for all $i$, then \[t=P^k(t)=t+kc\implies c=0,\]
so $P(t)=t$, which has at most $n$ roots.

Otherwise there exists $i<k$ with $P^{i+1}(t)=P^i(t)+c$ and $P^{i+2}(t)=P^{i+1}(t)-c$. Thus $P^i(t)=P^{i+2}(t)$, and so $P^j(t)=P^{j+2}(t)$ for all $j\ge i$. Since $P^k(t)=t$, we have $P^2(t)=t$.

Say an integer $t$ is fun iff $P(P(t))=t$. It suffices to prove there are at most $n$ fun integers.
\begin{claim*}
    If $a$ and $b$ are fun, then either $P(a)=a$ and $P(b)=b$, or $P(a)+a=P(b)+b$.
\end{claim*}
\begin{proof}
    For all fun $a$ and $b$, we have $a-b\mid P(a)-P(b)\mid a-b$. Consequently $|a-b|=|P(a)-P(b)|$, id est either $P(a)-a=P(b)-b$ or $P(a)+a=P(b)+b$. Since $P(b)$ is also fun, either $P(a)-a=b-P(b)$ or $P(a)+a=b+P(b)$.

    If $P(a)+a\ne P(b)+b$, then $P(a)-a=P(b)-b=a-P(a)$, so $P(a)=a$. Analogously $P(b)=b$, as claimed.
\end{proof}

Suppose we have two fixed points $P(a)=a$ and $P(b)=b$. Then for all other fun $t$, if $P(t)\ne t$, then $P(t)+t=P(a)+a=2a$ and $P(t)+t=P(b)+b=2b$. But $a\ne b$, contradiction.

Thus if we have a single fixed point $P(a)=a$, then for all other fun $t$, we have $P(t)+t=2a$, which has at most $n$ roots.
