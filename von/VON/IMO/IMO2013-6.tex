desc: Beautiful labelings
source: IMO 2013/6
tags: [2020-03, oly, brutal, combo, rigid, nice, involved]

---

Let $n\ge3$ be an integer, and consider a circle with $n+1$ equally spaced points marked on it. Consider all labelings of these points with the numbers $0$, $1$, $\ldots$, $n$ such that each label is used exactly once; two such labelings are considered to be the same if one can be obtained from the other by a rotation of the circle. A labeling is \emph{beautiful} if, for any four labels $a<b<c<d$ with $a+d=b+c$, the chord joining the points labeled $a$ and $d$ does not intersect the chord joining the points labeled $b$ and $c$.

Let $M$ be the number of beautiful labelings, and let $N$ be the number of ordered pairs $(x,y)$ of positive integers such that $x+y\le n$ and $\gcd(x,y)=1$. Prove that $M=N+1$.

---

It is not hard to show \[N+1=\vphi(1)+\vphi(2)+\cdots+\vphi(n).\]
Before we tackle $M=N+1$, we need some preliminary results. Say a chord is an $m$-chord if the sum of its endpoints is $m$.
\setcounter{claim}0
\begin{claim}
    Given any three $m$-chords, there is a line intersecting all of them.
\end{claim}
\begin{proof}
    Assume not. Note that we can replace one of the three $m$-chords with $(0,m)$ while still violating the claim hypothesis.
    \begin{center}
    \begin{asy}
        size(5cm); defaultpen(fontsize(10pt));
        pair A,B,C,D,EE,F;
        real t=45;
        A=dir(90-t);
        B=dir(90+t);
        C=dir(210-t);
        D=dir(210+t);
        EE=dir(330-t);
        F=dir(330+t);

        draw(A--B,lightred+linewidth(1));
        draw(C--D,lightred+linewidth(1));
        draw(EE--F,lightred+linewidth(1));
        draw(circle(origin,1));

        label("$0$",A,A);
        label("$m$",B,B);
        label("$a$",C,C);
        label("$m-a$",D,D);
        label("$b$",EE,EE);
        label("$m-b$",F,F);
    \end{asy}
    \end{center}
    Consider the above diagram. Without loss of generality $a<b$. (Else relabel $k$ with $m-k$ for each $k$ and apply the same argument.) We say a number is between $i$ and $j$ if it lies on the counterclockwise arc $ij$.

    I claim there is nowhere to place $m-b+a$. Note that
    \begin{itemize}[itemsep=0em]
        \item $m+(m-b)=(m-a)+(m-b+a)$ means $m-b+a$ lies between $m$ and $m-b$;
        \item $a+(m-b)=0+(m-b+a)$ means $m-b+a$ lies between $m-b$ and $a$;
        \item $m+a=b+(m-b+a)$ means $m-b+a$ lies between $a$ and $m$.
    \end{itemize}
    But the first two imply $m-b+a$ lies between $m$ and $a$, contradiction.
\end{proof}

Note that the claim works with the degenerate case $b=m/2$. Thus $m/2$ is not between any two $m$-chords.\\

We proceed with the main result by induction. The base case $n=3$ is clear. Given a beautiful labeling of the points $0$, $1$, $\ldots$, $n-1$, draw in the $n$-chords (colored red below). Say the $n$-chords divide the circle into \emph{slices}, with two \emph{edge slices}.

If $0$ does not lie on an edge slice, then $n$ must be on the opposite side of the slice, and if $0$ does lie on an edge slice, then $n$ may be in any one of the two positions adjacent to $0$.
\begin{claim}
    Both the aforementioned ways to insert $n$ result in beautiful labelings.
\end{claim}
\begin{proof}
    Since the original labeling is beautiful, it suffices to show no two chords with the same sum intersect when one of the chords includes $n$; that is, $d=n$, where $a<b<c<d$ are as in the problem statement.

    Then if $a=0$, we have constructed $n$ so that no $n$-chords intersect; $0\in\{b,c\}$ violates the condition $a<b,c$, so we may safely delete $0$.

    Replace each label $k$ with $n-k$, and reverse the order of the points on the circle. It is easy to check that this does not affect whether the labeling is beautiful. Moreover the resulting labeling is identical to the original labeling, so it is beautiful.
\end{proof}
\begin{center}
    \begin{asy}
        size(5cm); defaultpen(fontsize(10pt));
        real n = 11;
        pair p(real r) {
            return dir(360/n * r);
        }

        for (real r = 0; r <= 4+1e-9; r += 1)
        draw(p(r) -- p(n - 1 - r),lightred+linewidth(1));
        for (real r = 0; r <= 3+1e-9; r += 1)
        draw(p(r) -- p(n - 4 - r),lightblue+linewidth(1));
        draw(p(n - 3) -- p(n - 2),lightblue+linewidth(1));
        draw(circle(origin,1));

        label("$0$",p(0),p(0));
        label("$n$",p(n - 1),p(n - 1));
        label("$1$",p(3),p(3));
        label("$n-1$",p(7),p(7));
        label("$n-k$",p(8),p(8));
        label("$k-1$",p(9),p(9));
        label("$k$",p(2),p(2));
        label("$n/2$",p(5),p(5));
    \end{asy}
\end{center}

Thus it suffices to prove the number of labelings of $0$, $1$, $\ldots$, $n-1$ with $0$ on an edge slice is $\vphi(n)$. Place $n$ in the circle as shown in the above diagram, so the task is to count the number of beautiful labelings of $0$, $1$, $\ldots$, $n$ with $n$ directly clockwise to $0$.

Let $p(k)$ be the distance between $0$ and $k$; for instance, in the above diagram, $p(1)=3$. In fact, $p(n-1)$ determines everything.
\begin{claim}
    $p(n-k)\equiv kp(n-1)\pmod n$ for all $k$. (Note modulo $n$, not $n+1$.)
\end{claim}
\begin{proof}
    The result is by induction on $k$; for the base cases $k=0,1$ there is nothing to prove. Since all points except $n/2$ (which we have already located) lie on an $n$-chord, and all points except $n$ and $(n-1)/2$ (which we have already located) lie on an $(n-1)$-chord, we can check that the quantities $p(i)+p(n-i)$ and $p(i)+p(n-1-i)$ are fixed modulo $n$ for all $i$.

    Hence $p(n-1-i)-p(n-i)$ is fixed modulo $n$ for all $i$. Taking $i=0$ shows that this always equals $p(n-1)$, and so $p(n-1-k)-p(n-k)\equiv p(n-1)\pmod n$, as desired.
\end{proof}

But $p(k)$ is distinct for all $k$, so $p(n-1)$ must be relatively prime to $n$. In particular, the labeling is unique, thus the number of such labelings is $\vphi(n)$. This completes the proof.
\begin{remark}
    If we relax the condition that the points are equally spaced (only the order matters), then we can describe all beautiful labelings.

    Let the circle be the unit circle, and without loss of generality $(1,0)$ is labeled $0$. Then for all beautiful labelings, there is an irrational number $\nu$ so that the point labeled $k$ has argument $2\pi k\nu$.

    It is not hard to check there are $\vphi(1)+\vphi(2)+\cdots+\vphi(n)$ such labelings: As $\nu$ varies, the labeling changes only when $k\nu$ crosses an integer. From here, $k$ is the smallest integer such that $k\nu$ crosses an integer exactly $\vphi(k)$ times, and summing over $k$ gives the desired result.

    It is easy to check that all labelings of this form are beautiful, so this describes all beautiful labelings.
\end{remark}
