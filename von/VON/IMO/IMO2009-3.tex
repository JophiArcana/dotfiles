desc: Indexing itself gives arithmetic sequence
source: IMO 2009/3
tags: [2019-10, oly, hard, alg, sequence, FE, nice]

---

Suppose that $s_1$, $s_2$, $s_3$, $\ldots$ is a strictly increasing sequence of positive integers such that the sub-sequences \[s_{s_1},\; s_{s_2},\; s_{s_3},\; \ldots\quad\text{and}\quad s_{s_1+1},\; s_{s_2+1},\; s_{s_3+1},\; \ldots\]
are both arithmetic progressions. Prove that the sequence $s_1$, $s_2$, $s_3$, $\ldots$ is itself an arithmetic progression.

---

For convenience let $s(n)=s_n$.
\setcounter{claim}0
\begin{claim}
    The sequences $s(s(n))$ and $s(s(n)+1)$ have the same common difference.
\end{claim}
\begin{proof}
    Since $s(n)$ is strictly increasing, $s(s(i))<s(s(i)+1)\le s(s(i+1))$ for all $i$. Since $s(s(i)+1)$ is always between two arithmetic sequences with the same common difference, it must have the same common difference.
\end{proof}

Henceforth call this common difference $a$, and say $s(s(n)+1)-s(s(n))=c$, which is constant. Denote $t(n)=s(n+1)-s(n)$; since $1\le t(n)\le a$, there exist an upper and lower bound, which we will denote by $X$ and $Y$, respectively. Also let $x$ and $y$ obey $t(x)=X$ and $t(y)=Y$.
\begin{claim}
    $t(i)=X$ for $s(y)\le i<s(y+1)$.
\end{claim}
\begin{proof}
    By definition \[a=s(s(y+1))-s(s(y))\le X\big(s(y+1)-s(y)\big)=XY.\]
    Similarly \[a=s(s(x+1))-s(s(x))\ge Y\big(s(x+1)-s(x)\big)=XY,\]
    so $a=XY$. Furthermore \[\sum_{i=s(y)}^{s(y+1)-1}t(i)=s(s(y+1))-s(s(y))=XY.\]
    However by definition $t(i)\le X$, so the claim is proven.
\end{proof}

Hence $c=s(s(y)+1)-s(s(y))=X$. By an analogous argument $c=Y$, so $X=Y$. It follows that $t(n)$ is constant, and we are done.
