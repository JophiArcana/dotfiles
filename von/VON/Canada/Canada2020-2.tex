desc: Fixed triangle area tangent to rhombus incircle
source: Canada 2020/2
tags: [2020-03, oly, easy, geo, area]

---

Let $ABCD$ be a rhombus with incircle $\omega$. Points $P$ and $Q$ lie on sides $AB$ and $AD$ so that $\seg{PQ}$ is tangent to $\omega$. Show that as $P$ and $Q$ vary, the area of $\triangle CPQ$ is fixed.

---

\begin{center}
    \begin{asy}
        size(6cm); defaultpen(fontsize(10pt));

        pair I,WW,X,Y,Z,T,A,B,C,D,P,Q,R;
        I=(0,0);
        WW=dir(150);
        X=reflect( (0,0),(0,1))*WW;
        Y=-WW;
        Z=-X;
        T=dir(110);
        A=2*WW*X/(WW+X);
        B=2*X*Y/(X+Y);
        C=2*Y*Z/(Y+Z);
        D=2*Z*WW/(Z+WW);
        P=2*WW*T/(WW+T);
        Q=2*T*X/(X+T);
        R=reflect(A,C)*P;

        draw(P--C--Q,gray);
        draw(circle(I,1),gray);
        draw(P--R,dashed);
        draw(circumcircle(P,Q,R));
        draw(B--D);
        draw(A--B--C--D--cycle);
        draw(P--Q);

        dot("$I$",I,S);
        dot("$A$",A,N);
        dot("$D$",B,E);
        dot("$C$",C,S);
        dot("$B$",D,W);
        dot("$P$",P,NW);
        dot("$Q$",Q,NE);
        dot("$R$",R,E);
    \end{asy}
\end{center}
\begin{claim*}
    $BP\cdot DQ$ is fixed.
\end{claim*}
\begin{proof}
    In fact it equals $(BD/2)^2$. Let $I$ be the center of $\omega$, and let $R$ lie on $\seg{AD}$ such that $\seg{PR}\parallel\seg{BD}$. Then $IP=IR$ and $\seg{DI}$ bisects $\angle PQR$, so $PQRI$ is cyclic.

    It follows that $BP\cdot DQ=DR\cdot DQ=DI^2$, as desired.
\end{proof}

Let $x=BP$ and $y=DQ$. Without loss of generality $\operatorname{Area}(ABCD)=2$. Then $\operatorname{Area}(\triangle APQ)=(1-x)(1-y)$, $\operatorname{Area}(\triangle BPC)=x$, $\operatorname{Area}(\triangle DQC)=y$, so \[\operatorname{Area}(\triangle CPQ)=2-(1-x)(1-y)-x-y=1-xy,\]
which is fixed.
