desc: Concur at anti-Steiner point wrt. medial triangle
source: First Fontene theorem
tags: [2019-11, oly, hard, geo, angle-chasing, construction, steiner]

---

Let $ABC$ be a triangle with median triangle $M_AM_BM_C$. Let $Q$ be a point inside $\triangle ABC$ with pedal triangle $A_3B_3C_3$. Then the following lines concur on the pedal circle of $Q$:
\begin{itemize}[itemsep=0em]
    \item The line through $A_3$ and $\seg{M_BM_C}\cap\seg{B_3C_3}$,
    \item The line through $B_3$ and $\seg{M_CM_A}\cap\seg{C_3A_3}$, and
    \item The line through $C_3$ and $\seg{M_AM_B}\cap\seg{A_3B_3}$.
\end{itemize}

---

Let $O$ be the circumcenter of $\triangle ABC$ and thus the orthocenter of $\triangle M_AM_BM_C$. I claim that this concurrence point is the anti-Steiner point of $\seg{OQ}$ with respect to $\triangle M_AM_BM_C$. Denote $A_0=\seg{M_BM_C}\cap\seg{B_3C_3}$, $Y=\seg{M_CM_A}\cap\seg{C_3A_3}$, $Z=\seg{M_AM_B}\cap\seg{A_3B_3}$, and let $A_1$ be the foot from $A$ to $\seg{OQ}$ and $K$ and $P$ be the reflections of $A_3$ and $A_1$ over $\seg{M_BM_C}$ (thus $\seg{AK}\parallel\seg{BC}$).

We will show that $P$ lies on $(M_AM_BM_C)$ (and is thus the anti-Steiner point), $(A_3B_3C_3)$, and line $A_3A_0$, from which the desired conclusion is clear.
\begin{center}
    \begin{asy}
        size(8cm); defaultpen(fontsize(10pt));

        pair O,A,B,C,MA,MB,MC,Q,A3,B3,C3,X,Y,Z,P,A0,K;
        O=(0,0);
        A=dir(130);
        B=dir(210);
        C=dir(330);
        MA=(B+C)/2;
        MB=(C+A)/2;
        MC=(A+B)/2;
        Q=0.4*dir(190);
        A3=foot(Q,B,C);
        B3=foot(Q,C,A);
        C3=foot(Q,A,B);
        X=extension(MB,MC,B3,C3);
        Y=extension(MC,MA,C3,A3);
        Z=extension(MA,MB,A3,B3);
        P=extension(B3,Y,C3,Z);
        A0=foot(A,Q,O);
        K=reflect(MB,MC)*A3;

        draw(A3--P,gray);
        draw(circumcircle(A,Q,A0),gray);
        draw(circumcircle(A3,B3,C3),gray);
        draw(circumcircle(MC,C3,X),gray);
        draw(circumcircle(MB,B3,X),gray);
        draw(K--A0);
        draw(circumcircle(MA,MB,MC));
        draw(circumcircle(A,B,C));
        draw(A--B--C--A);
        draw(MB--MC);
        draw(B3--C3);

        dot("$A$",A,N);
        dot("$B$",B,SW);
        dot("$C$",C,SE);
        dot("$M_A$",MA,SE);
        dot("$M_B$",MB,dir(30));
        dot("$M_C$",MC,dir(130));
        dot("$A_3$",A3,S);
        dot("$B_3$",B3,dir(60));
        dot("$C_3$",C3,dir(170));
        dot("$A_0$",X,dir(-35));
        dot("$P$",P,dir(105));
        dot("$Q$",Q,SE);
        dot("$K$",K,N);
        dot("$A_1$",A0,dir(255));
    \end{asy}
\end{center}
First notice that $AKB_3QA_1C_3$ is cyclic with diameter $\seg{AQ}$ and $AM_BOA_1M_C$ is cyclic with diameter $\seg{AO}$. Hence \[\da A_1M_BA_0=\da A_1M_BM_C=\da A_1AM_C=\da A_1AC_3=\da A_1B_3C_3=\da A_1B_3A_0,\]
so $A_1$ lies on $(A_0B_3M_B)$ and $(A_0C_3M_C)$. Furthermore \[\da B_3A_1K=\da B_3AK=\da B_3M_BA_0=\da B_3A_1A_0,\]
whence $K$, $A_0$, $A_1$ are collinear. By reflection, $P$, $A_0$, $A_3$ are collinear as well.

Since the reflection of $A$ over $\seg{M_BM_C}$ is the foot from $A$ to $\seg{BC}$, which lies on the nine-point circle $(M_AM_BM_C)$, and $A_1\in(AM_BM_C)$, $P$ lies on $(M_AM_BM_C)$. Finally \[A_0P\cdot A_0A_3=A_0A_1\cdot A_0K=A_0B_3\cdot A_0C_3,\]
so $P$ lies on $(A_3B_3C_3)$, and we are done.
