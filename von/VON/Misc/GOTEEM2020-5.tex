desc: Fixed point on (AB_2C_2)
source: GOTEEM 2020/5
tags: [2020-01, oly, medium, geo, spiral-sim, angle-chasing, nice]

---

Let $ABC$ be a triangle and let $B_1$ and $C_1$ be variable points on sides $\overline{BA}$ and $\overline{CA}$, respectively, such that $BB_1=CC_1$. Let $B_2\ne B_1$ denote the point on $(ACB_1)$ such that $\seg{BC_1}$ is parallel to $\seg{B_1B_2}$, and let $C_2\ne C_1$ denote the point on $(ABC_1)$ such that $\seg{CB_1}$ is parallel to $\seg{C_1C_2}$. Prove that as $B_1$ and $C_1$ vary, the circumcircle of $\triangle AB_2C_2$ passes through a fixed point other than $A$.

---

\begin{center}
    \begin{asy}
        size(10cm); defaultpen(fontsize(10pt));

        pair A,B,C,I,D,B1,C1,B2,C2,K,X,Y;
        A=dir(125);
        B=dir(210);
        C=dir(330);
        I=incenter(A,B,C);
        real t=0.7;
        D=(t+1)*I-t*A;
        B1=2*foot(circumcenter(A,D,C),A,B)-A;
        C1=2*foot(circumcenter(A,D,B),A,C)-A;
        B2=2*foot(circumcenter(A,D,C),B1,B1+C1-B)-B1;
        C2=2*foot(circumcenter(A,D,B),C1,C1+B1-C)-C1;
        K=dir(90);
        X=extension(B1,B2,C1,C2);
        Y=extension(B1,C1,B2,C2);

        //draw(B1--B2,Dotted); draw(C1--C2,Dotted);
        draw(B--C1,Dotted); draw(C--B1,Dotted);
        draw(A--D,dashed);
        draw(arc(circumcenter(B1,C1,B2),circumradius(B1,C1,B2),170,10,CCW),gray);
        draw(B2--D--C2,gray);
        draw(circumcircle(A,B1,C1),dashed);
        draw(circumcircle(A,D,C));
        draw(circumcircle(A,D,B));
        draw(A--B--C--A);
        draw(K--Y--C1);
        draw(Y--B2);

        dot("$A$",A,NW);
        dot("$B$",B,SW);
        dot("$C$",C,SE);
        dot("$D$",D,S);
        dot("$B_1$",B1,dir(215));
        dot("$C_1$",C1,E);
        dot("$B_2$",B2,E);
        dot("$C_2$",C2,dir(140));
        dot("$K$",K,N);
        dot("$X$",Y,W);
    \end{asy}
\end{center}
Let $K$ be the midpoint of arc $BAC$ on the circumcircle of $\triangle ABC$, and let $(ABC_1)$ and $(ACB_1)$ intersect again at $D$. Denote $X=\seg{B_1B_2}\cap\seg{C_1C_2}$ and $Y=\seg{B_1C_1}\cap\seg{B_2C_2}$. I claim that $K$ is the fixed point.

Since $KB=KC$, $BB_1=CC_1$, and $\da KBB_1=\da KCC_1$, $\triangle KBB_1\cong\triangle KCC_1$, so $K$ is the center of spiral similarity sending $\seg{BB_1}$ to $\seg{CC_1}$ and $K$ lies on $(AB_1C_1)$. Moreover, $D$ is the center of spiral similarity sending $\seg{BB_1}$ to $\seg{C_1C}$. Since $BB_1=CC_1$, we have $DB=DC_1$, so $\seg{AD}$ bisects $\angle BAC$.

Notice that \[\da ADB_1=\da ACB_1=\da AC_1C_2=\da ADC_2\]
and $\da ADC_1=\da ADB_2$, so $D=\seg{B_1C_2}\cap\seg{B_2C_1}$. Furthermore \[\da B_1B_2C_1=\da B_1AD=\da DAC_1=\da B_1DC,\]
so $B_1C_1B_2C_2$ is cyclic.

Finally $A$ is the Miquel point of $B_1B_2C_1C_2$, so by a well-known property of complete quadrilaterals, $A$ is the foot from $X$ to $\seg{AD}$, id est $\angle XAD=90\dg$. It follows that $K\in\seg{AX}$, so \[XB_2\cdot XC_2=XB_1\cdot XC_1=XA\cdot XK,\]
and we are done.
