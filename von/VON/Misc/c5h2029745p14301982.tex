author: Tovi Wen
desc: Incenter AIME I 2020/15
source: c5h2029745p14301982
tags: [2020-03, answer, chard, geo, pop, aops, waltz, variation]

---

Let $ABC$ be a triangle with circumcircle $\omega$ and incenter $I$. Suppose the tangent to the circumcircle of $\triangle IBC$ at $I$ intersects $\omega$ at points $X$ and $Y$ with $IA=3$, $IX=2$, $IY=6$. The area of $\triangle ABC$ can be written as $\tfrac mn$, whre $m$ and $n$ are relatively prime positive integers. Find $m+n$.

---

Without loss of generality $B$ is closer to $X$ than $Y$. Let $L$ be the midpoint of arc $BC$ opposite $A$, and let $K$ be the midpoint of arc $BAC$.
\begin{center}
    \begin{asy}
        size(6cm); defaultpen(fontsize(10pt));
        pair A,B,C,O,I,L,K,X,Y;
        A=(0,33);
        B=(0,0);
        C=(56,0);
        O=(A+C)/2;
        I=incenter(A,B,C);
        L=extension(A,I,O,(B+C)/2);
        K=2O-L;
        X=intersectionpoint(circumcircle(A,B,C),I -- (I+rotate(90)*(A-I)));
        Y=2*foot(O,I,X)-X;

        draw(A--X,gray);
        draw(K--Y,gray);
        draw(A--L,gray+dashed);
        draw(arc(L,abs(I-L),0,180,CCW),dashed);
        draw(X--Y);
        draw(K--A);
        draw(circumcircle(A,B,C));
        draw(A--B--C--A);

        dot("$A$",A,NW);
        dot("$B$",B,W);
        dot("$C$",C,E);
        dot("$I$",I,dir(75));
        dot("$L$",L,S);
        dot("$K$",K,N);
        dot("$X$",X,W);
        dot("$Y$",Y,E);
    \end{asy}
\end{center}
By power of a point, $IA\cdot IL=IX\cdot IY\implies IL=4\implies AL=7$. By Incenter-Excenter lemma $L$ is the circumcenter of $\triangle IBC$, so $\seg{XY}\perp\seg{AL}\perp\seg{AK}$. Hence $AXYK$ is an isosceles trapezoid, so if $T$ is the foot from $K$ to $\seg{XY}$, then $AK=IT=4$.

But $\angle KAL=90\dg$ and $AK=LC$, so $AKCL$ is a rectangle. From this $AC=\sqrt{65}$ and $\angle ABC=90\dg$. Also by Incenter-Excenter lemma, the $A$-excenter $I_A$ is the reflection of $I$ over $L$, so $AI_A=11$ and $AB\cdot AC=AI\cdot AI_A=33$. Thus $AB=\tfrac{33}{\sqrt{65}}$, and since $AB:BC:CA=33:56:65$, we have $BC=\tfrac{56}{\sqrt{65}}$ and \[\operatorname{Area}(\triangle ABC)=\frac12\cdot\frac{33}{\sqrt{65}}\cdot\frac{56}{\sqrt{65}}=\frac{924}{65}.\]
The requested sum is $924+65=989$.
