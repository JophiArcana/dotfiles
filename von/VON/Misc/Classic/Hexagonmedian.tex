desc: Hexagon median
source: Hexagon median
tags: [2019-10, answer, cbrutal, geo, secret]
author: Espen Slettnes

---

Points $P$ and $Q$ are chosen uniformly and at random on the perimeter of a unit hexagon. Compute the median length of $PQ$; that is, find the real number $r$ such that the probability that $PQ<r$ is $1/2$.

---

Let the hexagon be $ABCDEF$, and assume WLOG point $P$ is on $\seg{AB}$. Let the second point be $Q$, and let the circle centered at $P$ with radius $r$ be $\omega$.
\begin{center}
    \begin{asy}
        size(4cm);
        defaultpen(fontsize(9pt));
        pair A,B,C,D,EE,F,P,Q;
        A=dir(120);
        B=dir(60);
        C=dir(0);
        D=dir(300);
        EE=dir(240);
        F=dir(180);
        P=(7A+3B)/10;
        Q=intersectionpoint(B--C,circle(P,sqrt(sqrt(27)/pi)));

        draw(A--B--C--D--EE--F--A);
        draw(P--Q);
        draw(circle(P,sqrt(sqrt(27)/pi)),gray);

        clip((A+(-100,0.2))--(B+(100,0.2))--(B+(100,-100))--(A+(-100,-100))--cycle);
        dot("$A$",A,A);
        dot("$B$",B,B);
        dot("$C$",C,dir(-10));
        dot("$D$",D,D);
        dot("$E$",EE,EE);
        dot("$F$",F,dir(120));
        dot("$P$",P,N);
        dot("$Q$",Q,dir(-10));
    \end{asy}
\end{center}
\begin{claim*}
    $1<r<\sqrt3$.
\end{claim*}
\begin{proof}
    If $r\le 1$, then $\omega$ intersects $\seg{AF}$ and $\seg{BC}$. Thus the intersection of the sides of the hexagon and $\omega$ has length less than $3$, contradiction.

    If $r\ge\sqrt3$, then $\omega$ intersects the opposite side of the hexagon (or completely contains it). Then the intersection of the sides of the hexagon and $\omega$ has length more than $3$, contradiction.
\end{proof}

Now that we know $\omega$ contains all of $\seg{AB}$ and none of $\seg{DE}$, restrict $Q$ to $\seg{BC}\cup\seg{CD}$. It is not hard to check that we need the probability that $PQ<r$ to be $1/2$ as well. Fix $\omega$ and line $AB$, and animate point $A$. Then $\seg{BC}\cup\seg{CD}$ traces out an arrow-shaped region. In particular point $Q$ is uniformly selected in this region.
\begin{center}
    \begin{asy}
        size(4cm);
        defaultpen(fontsize(9pt));
        real r=sqrt(sqrt(27)/pi);
        pair P, B, C, D, EE, F, X, Y;
        B=P+(1,0);
        C=P+sqrt(3)*dir(-30);
        D=P+2*dir(-60);
        EE=P-(0,sqrt(3));
        F=P+C-B;
        X=intersectionpoint(circle(P,r),B--C);
        Y=intersectionpoint(circle(P,r),EE--F);

        draw(circle(P,r));
        draw(P--B--C--D--EE--F--P);
        draw(X--P--Y,gray);

        clip((P+(-100,0.2))--(B+(100,0.2))--(B+(100,-100))--(P+(-100,-100))--cycle);
        dot("$P$",P,NW);
        dot("$B$",B,N);
        dot("$C$",C,E);
        dot("$D$",D,D);
        dot("$E$",EE,SW);
        dot("$G$",F,W);
        dot("$X$",X,E);
        dot("$Y$",Y,dir(300));
    \end{asy}
\end{center}
Here points are labeled as if $A=P$, and $G$ is where point $C$ is when $B=P$. Clearly $PBCDE$ has area $\sqrt3$, so we need the intersection of $\omega$ and $PBCDE$ to have area $\sqrt3/2$. Let $\omega$ intersect $\seg{BC}$ at $X$ and $\seg{EG}$ at $Y$.

Note that $PB=PG$ and $PX=PY$. Since $\angle PBX=\angle PGY=120^\circ$ (which is obtuse), $\triangle PBX\cong\triangle PGY$. Thus the intersection of $\omega$ and $PBCDE$ has the same area as the sector of $\omega$ from $X$ to $Y$. This sector has central angle $60^\circ$, so \[\frac16\pi r^2=\frac{\sqrt3}2\implies r=\sqrt[4]{\frac{27}{\pi^2}},\]
and we are done.

---

$\sqrt[4]{\dfrac{27}{\pi^2}}$
