desc: Midpoint AH on incircle iff <AIH=90
source: HMMT 2011 T12
tags: [2020-03, oly, easy, geo, conditional, waltz]

---

Let $ABC$ be an acute scalene triangle with orthocenter $H$ and incenter $I$. Prove that the midpoint of $\overline{AH}$ lies on the incircle of $\triangle ABC$ if and only if $\angle AIH=90^\circ$.

---

\begin{center}
\begin{asy}
    size(8cm); defaultpen(fontsize(10pt));
    pen pri=lightred;
    pen sec=lightblue;
    pen tri=purple+pink;
    pen fil=pri+opacity(0.05);
    pen sfil=sec+opacity(0.05);
    pen tfil=tri+opacity(0.05);

    real t=0.4;
    pair O,M,I,B,C,A,H,K,D,EE,F,NN;
    O=(0,0);
    M=(0,-t);
    I=(-sqrt(1-2t),0);
    B=(-sqrt(1-t*t),-t);
    C=(sqrt(1-t*t),-t);
    A=extension(B,reflect(B,I)*C,C,reflect(C,I)*B);
    H=A+B+C;
    K=(A+H)/2;
    D=foot(A,B,C);
    EE=foot(B,C,A);
    F=foot(C,A,B);
    NN=circumcenter(D,EE,F);

    filldraw(circumcircle(A,EE,F),tfil,tri);
    draw(A--I,tri+dashed);
    draw(K--NN,sec+Dotted);
    draw(EE--F,sec);
    filldraw(circumcircle(D,EE,F),sfil,sec);
    filldraw(incircle(A,B,C),fil,pri);
    filldraw(A--B--C--cycle,fil,pri);
    draw(A--D,pri);

    dot("$A$",A,N);
    dot("$B$",B,SW);
    dot("$C$",C,SE);
    dot("$D$",D,SW);
    dot("$E$",EE,dir(60));
    dot("$F$",F,SW);
    dot("$H$",H,SE);
    dot("$K$",K,SE);
    dot("$I$",I,W);
    dot("$N$",NN,SE);
\end{asy}
\end{center}
Let $I$ be the incenter, $N$ the nine-point center, and $K$ the midpoint of $\seg{AH}$; also let $DEF$ be the orthic triangle. Note that $\seg{KN}$ is the perpendicular bisector of $\seg{EF}$. Hence the following are equivalent:
\begin{align*}
    &K\text{ lies on the incircle}\\
    &\iff K\text{ is the Feuerbach point}\\
    &\iff K,\ I,\ N\text{ collinear}\\
    &\iff IE=IF\\
    &\iff I\in(AEF)\\
    &\iff \angle AIH=90\dg.
\end{align*}
Most of the steps above should be easy to follow. The only step that may require clarification is $K$, $I$, $N$ collinear imply $K$ is the Feuerbach point; if $K$ is not the Feuerbach point, the Feuerbach point must be the antipode of $K$ on the nine-point circle, which is the midpoint $M$ of $\seg{BC}$. However for $M$ to lie on the incircle, we must have $AB=AC$, contradiction.

This completes the proof.

\begin{remark}
    Some notes on configuration. Let $A_0$ be the $A$-intouch point, $O$ the circumcenter, and $M$ the midpoint of $\seg{BC}$. Then $IA_0=IK=AK=OM$, so $IOMA_0$ is a rectangle and $\seg{IO}\parallel\seg{BC}$.

    We also have parallelograms $AKA_0I$ and $KHA_0I$, which lead to some interesting properties, e.g.\ $\seg{A_0H}\perp\seg{EF}$.
\end{remark}
\begin{remark}
    I will explain how the above asymptote diagram is constructed. As usual the circumradius is $R=1$. I will choose a real number $r\in(0,1)$ to be the length $OM$, i.e.\ the inradius. Then $OI^2=R(R-2r)=1-2r$, so $OI=\sqrt{1-2r}$. Thus
    \begin{align*}
        M&=(0,-t);\\
        B&=\left(-\sqrt{1-t^2},-t\right);\\
        C&=\left(\sqrt{1-t^2},-t\right);\\
        I&=\left(-\sqrt{1-2t},0\right).
    \end{align*}
    From here, reflect $\seg{BC}$ across $\seg{BI}$ and $\seg{CI}$. By degrees of freedom/intuition, they intersect at $A$ on the circumcircle, and there we have $\triangle ABC$.
\end{remark}
