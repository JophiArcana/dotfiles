desc: Rational polygon areas
source: CIME II 2020/15
tags: [2020-02, answer, cbrutal, alg, nt, intpoly]

---

Let $P_1P_2\cdots P_{72}$ be a regular dodecagon with area $1$, and let $P_i=P_{i+72}$ for all integers $i$. Let $S$ be the sum of the squares all positive integers $a<72$ such that
\begin{itemize}
    \item for all $i$, $P_{i-3a}\ne P_{i+a}$ and $P_{i-a}\ne P_{i+3a}$;
    \item for all $i$, lines $P_{i-3a}P_{i+a}$ and $P_{i-a}P_{i+3a}$ are not parallel, do not coincide, and intersect at a point $Q_i$; and
    \item the points $Q_1$, $Q_2$, $\ldots$, $Q_{72}$ form a polygon with positive, rational area.
\end{itemize}
Find the remainder when $S$ is divided by $1000$.

---

Toss on the complex plane, and assume $P_0=1$ and the center of the polygon is $0$. It suffices for $OQ_i^2$ to be rational. We find all angles $\theta$ such that if $\omega=e^{\theta i}$, then the intersection $q$ of the line through $\omega^3$ and $\omega^{-1}$ and real axis is the square root of a rational number.

Then $q$ is the intersection of $\seg{\omega^3\omega^{-1}}$ and $\seg{(1)(-1)}$, so by the complex chord intersection formula, \[q=\frac{\omega^3\cdot\omega^{-1}(1+(-1))-1(-1)(\omega^3+\omega^{-1})}{\omega^3\cdot\omega^{-1}-1(-1)}=\frac{\omega^3+\omega^{-1}}{\omega^2+1}.\]
However note that
\begin{align*}
    \left|\omega^3+\omega^{-1}\right|^2&=\big[\cos(3\theta)+\cos(-\theta)\big]^2+\big[\sin(3\theta)+\sin(-\theta)\big]^2\\
    &=2+2\big[\cos(3\theta)\cos(-\theta)+\sin(3\theta)\sin(-\theta)\big]\\
    &=2+2\cos(4\theta),
\end{align*}
and similarly
\begin{align*}
    \left|\omega^2+1\right|^2&=\left[\cos(2\theta)+1\right]^2+\sin^2(2\theta)\\
    &=2+2\cos(2\theta).
\end{align*}

Since $q$ is real, \[q^2=\frac{2+2\cos(4\theta)}{2+2\cos(2\theta)}=\frac{\cos^22\theta}{\cos^2\theta}\]
needs to be rational. Note that \[|q|=\frac{\cos2\theta}{\cos\theta}=2\cos\theta-\frac1{\cos\theta}\implies0=2\cos^2\theta-|q|\cos\theta-1,\]
so since $q^2\in\mathbb Q$, $\cos\theta$ is the root of a polynomial with rational coefficients and degree $4$.

We turn to the following well-known lemma:
\begin{lemma*}[Minimal polynomial of $\cos\theta$]
    Let $\theta=\frac{2\pi k}n$, where $k$ and $n$ are relatively prime. Then the minimal polynomial of $\cos\theta$ over $\mathbb Q$ has degree $\frac{\varphi(n)}2$.
\end{lemma*}

Thus the minimal polynomial of $\cos\theta$ over $\mathbb R$ has degree at most $\frac{\varphi(n)}2$, so we have $\frac{\varphi(n)}2\le4$, or $\varphi(n)\le8$.
\begin{claim*}[Extracting $n$]
    If $\varphi(n)\le8$ and $n\mid 72$, then $n\in\{2,3,4,6,8,9,12,18,24\}$.
\end{claim*}
\begin{proof}
    (Details omitted.) The largest $n$ with $\varphi(n)\le8$ is $30$, and an exhaustive check gives the above list.
\end{proof}

Now all that remains is answer extraction. We will find all $\theta<\pi$, since $\theta$ works if and only if $2\pi-\theta$ works, and $\theta=\pi$ violates the first condition.
\begin{itemize}
    \item We first take care of $n\mid12$, so $\theta=\frac{\pi k}6$ for some $0<k<6$. Of these, $k\in\{1,2,4,5\}$ work, giving $\frac13$, $1$, $1$, $\frac13$, respectively.
    \item Let $n=8$, so $\theta=\frac{\pi k}4$ for $k\in\{1,3\}$. These both give zero area, thus they don't work.
    \item Let $n=9$ or $n=18$, so $\theta=\frac{\pi k}9$ for $k\in\{1,2,4,5,7,8\}$. These all give irrational area.
    \item Let $n=24$, so $\theta=\frac{\pi k}{24}$ for $k\in\{1,5,7,11\}$. These all give irrational area.
\end{itemize}
Hence the $\theta$ that work are $\pm\frac\pi6$, $\pm\frac{2\pi}6$, $\pm\frac{4\pi}6$, $\pm\frac{5\pi}6$, which give $\frac a6\in\{1,2,4,5,7,8,10,11\}$. It follows that the sum of the squares of all possible values of $\frac a6$ is \[\sum\frac{a^2}{36}=1^2+2^2+4^2+5^2+7^2+8^2+10^2+11^2=380,\]
from which $S=36\cdot380=13680$, and the requested remainder is $680$.

---

680
