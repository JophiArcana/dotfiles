author: Eric Shen, Raymond Feng
desc: tanh FE
source: CAMO 2020/1
tags: [2019-12, oly, easy, alg, FE]

---

Let $f:\mathbb R_{>0}\to\mathbb R_{>0}$ (meaning $f$ takes positive real numbers to positive real numbers) be a nonconstant function such that for any positive real numbers $x$ and $y$, \[f(x)f(y)f(x+y)=f(x)+f(y)-f(x+y).\]
Prove that there is a constant $a>1$ such that \[f(x)=\frac{a^x-1}{a^x+1}\]
for all positive real numbers $x$.

---

Rewrite our functional equation as \[f(x+y)=\frac{f(x)+f(y)}{1+f(x)f(y)}.\]The key claim is that $f(x)<1$ or $f\equiv 1$.
\setcounter{claim}0
\begin{claim}
    $f(x)\ge 1\implies f(\tfrac x2)=1$.
\end{claim}
\begin{proof}
    Plugging in $(\tfrac x2, \tfrac x2)$ into our functional equation gives \[\frac{2f(\tfrac x2)}{1+f(\tfrac x2)^2}=f(x)\ge 1\implies \left(f\left(\frac x2\right)-1\right)^2\le 0\implies f\left(\frac x2\right)=1,\]
    as desired.
\end{proof}
\begin{claim}
    $f(x)\ge 1\implies f\equiv 1$.
\end{claim}
\begin{proof}
    By Claim 1, there exists $y$ such that $f(y)=1$. Furthermore, if $f(y)=1$ then $f(\tfrac y2)=1$ by Claim 1, so we can take  $y$ infinitely small. Then, by our functional equation, \[f(x+y)=\frac{f(x)+1}{1+f(x)}=1\]for all $x$, so $f\equiv 1$.
\end{proof}

Now, discard the trivial solution $f\equiv 1$. We have that $f(x)<1$ for all $x$. Let \[g(x)=\ln\left(\frac{1+f(x)}{1-f(x)}\right).\]
Then, \[g(x+y)=\ln\left(\frac{(1+f(x))(1+f(y))}{(1-f(x))(1-f(y))}\right)=g(x)+g(y),\]
so $g$ satisfies Cauchy's Functional Equation. Since $f(x)>0$, $g(x)>0$, so $g$ is bounded and there exists a positive constant $k$ such that $g(x)\equiv kx$. Thus $a=e^k$, and we are done.
