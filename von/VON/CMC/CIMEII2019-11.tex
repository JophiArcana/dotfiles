author: Eric Shen
desc: Quadrilateral SIDE
source: CIME II 2019/11
tags: [2019-12, answer, ctricky, geo, area]

---

In triangle $ABC$ with incenter $I$, $AB=4$, $BC=5$, and $CA=6$. If lines $AI$ and $BI$ meet the circumcircle of $\triangle ABC$ again at $S$ and $L$, respectively, and $\overline{LB}$ and $\overline{LS}$ intersect $\overline{AC}$ at $D$ and $E$, respectively, then the square of the area of quadrilateral $SIDE$ can be expressed as $\tfrac mn$, where $m$ and $n$ are relatively prime positive integers. Find $m+n$.

---

To begin, let $L'$ be the reflection of $A$ over the perpendicular bisector of $\overline{BC}$, so that $\triangle ABC\cong\triangle L'CB$. However, by Ptolemy's on $ABCL'$, \[5AL'+4\cdot 4=6\cdot 6\implies AL'=4.\]However, it follows that $AB=AL'=CL'=4$, so $L=L'$; that is, $ABCL$ is an isosceles trapezoid. Now, by the Incenter-Excenter Lemma, $SB=SI=SC=d$ for some $d$. By Ptolemy's on $ABSC$, \[4\cdot d+6\cdot d=10\cdot AS\implies d=\frac{AS}2,\]whence $I$ is the midpoint of $\overline{AS}$.

The semiperimeter $s$ of $\triangle ABC$ is clearly $15/2$. It follows by Heron's and $K=rs$, where $r$ denotes the inradius of $\triangle ABC$, that \[r=\sqrt{\frac{(s-a)(s-b)(s-c)}s}=\sqrt{\frac72\cdot\frac52\cdot\frac32\div\frac{15}2}=\frac{\sqrt7}2,\]

\begin{center}
    \begin{asy}
        size(4cm); defaultpen(fontsize(10pt));
        pair A, B, C, S, L, P, Q, R;
        B=dir(270-aCos(9/16));
        C=dir(270+aCos(9/16));
        A=intersectionpoint(circle((0, 0), 1), (B+0.01*(1, 3sqrt(7))) -- (B+100*(1, 3sqrt(7))));
        S=dir(270);
        L=intersectionpoint(circle((0, 0), 1), (C+0.01*(-1, 3sqrt(7))) -- (C+100*(-1, 3sqrt(7))));
        P=extension(A, S, B, L);
        Q=extension(A, C, B, L);
        R=extension(A, C, L, S);

        draw(A -- B -- C -- A);
        draw(A -- S -- L -- B, dashed);

        draw(circumcircle(A, B, C));
        dot("$A$", A, A);
        dot("$B$", B, B);
        dot("$C$", C, C);
        dot("$S$", S, S);
        dot("$L$", L, L);
        dot("$I$", P, W);
        dot("$D$", Q, N);
        dot("$E$", R, E);

        draw(incircle(A, B, C), dotted);
        draw(arc(S, length(B-S), 0, 180), dotted);
    \end{asy}
\end{center}
If $F$ denotes the tangency point between the incircle and $\overline{AB}$, we can compute that $AF=s-a=5/2$. Then, by the Pythagorean Theorem on $\triangle AFI$, $AI^2=\sqrt{(s-a)^2+r^2}=2\sqrt2$.

Now, since $I$ and $E$ are the midpoints of $\overline{SA}$ and $\overline{SL}$, respectively, $D$ is the centroid of $\triangle ASL$, so $[SIDE]=[ASL]/3$. However, since $AS=2AI=4\sqrt2$ and $AL=4$, the distance from $S$ to $AL$ is $\sqrt{(4\sqrt2)^2-2^2}=2\sqrt7$, whence \[[SIDE]=\frac{[ASL]}3=\frac{\frac12\cdot 4\cdot 2\sqrt7}3=\frac{4\sqrt7}3\implies [SIDE]^2=\frac{112}9,\]
and the requested sum is $112+9=121$.

\begin{remark}[Alternative solution]
    By the Angle Bisector Theorem, $AD=8/3$ and $DC=10/3$. However, \[BD^2=BA\cdot BC-DA\cdot DC=20\left(1-\left(\frac23\right)^2\right)=\frac{100}9,\]
    so $\triangle BDC$ is isosceles. Then, $AL=LC=AB=4$, and we may proceed as above.
\end{remark}

---

121
