author: Eric Shen, Raymond Feng
desc: Squid hunting
source: CAMO 2020/6
tags: [2019-12, oly, brutal, combo, optimization, nice, involved]

---

Let $n$ be a positive integer. Eric and a squid play a turn-based game on an infinite grid of unit squares. Eric's goal is to capture the squid by moving onto the same square as it.

Initially, all the squares are colored white. The squid begins on an arbitrary square in the grid, and Eric chooses a different square to start on. On the squid's turn, it performs the following action exactly $2020$ times: it chooses an adjacent unit square that is white, moves onto it, and sprays the previous unit square either black or gray. Once the squid has performed this action $2020$ times, all squares colored gray are automatically colored white again, and the squid's turn ends. If the squid is ever unable to move, then Eric automatically wins. Moreover, the squid is claustrophobic, so at no point in time is it ever surrounded by a closed loop of black or gray squares. On Eric's turn, he performs the following action at most $n$ times: he chooses an adjacent unit square that is white and moves onto it. Note that the squid can trap Eric in a closed region, so that Eric can never win.

Eric wins if he ever occupies the same square as the squid. Suppose the squid has the first turn, and both Eric and the squid play optimally. Both Eric and the squid always know each other's location and the colors of all the squares. Find all positive integers $n$ such that Eric can win in finitely many moves.

---

Let $s=2020$. In general, the answer for $s\ge8$ is $n\ge2s-5$. Henceforth, by ``distance,'' we refer to the length of the shortest path between Eric and the squid that does not intersect the squid ink. For all shown diagrams, a white circle represents Eric's initial position, a black circle represents the squid's initial position, a gray line represents Eric's path, and a solid line represents the squid's path.

\bigskip

\textbf{Proof of upper bound:}     Say $n<2s-5$. The key here is that Eric cannot get close enough to the squid, or the squid can surround Eric. We use the following estimate.
\setcounter{claim}0
\begin{claim}
    Suppose it's the squid's turn, the squid has only used gray ink (so there are no black squares), and the distance between Eric and the squid is $d\le s-6$. Then the squid wins.
\end{claim}
\begin{proof}
    Consider the following picture.

    \setcounter{figure}0
    \begin{figure}[h]
        \begin{center}
            \begin{asy}
                size(3cm); defaultpen(fontsize(10pt));
                dotfactor *= 2;

                draw( (5,-3)--(5,0)--(1,0)--(1,1)--(-1,1)--(-1,-1)--(1,-1),Arrow());
                dot( (5,-3));
                dot( (0,0),filltype=FillDraw(fillpen=white,drawpen=black));
            \end{asy}
        \end{center}
        \caption{Surrounding Eric}
        \label{fig:squid-surround}
    \end{figure}

    Here it takes the squid $d-1$ moves to get to the closest point adjacent to Eric, and then $7$ moves to surround him. Hence if $s\ge d+6$, the squid can surround Eric using black ink, as claimed.
\end{proof}

Now assume that at an arbitrary point in time, Eric is a distance of $d\ge s-5$ away, and it's the squid's turn. As the squid continues moving, it is able to increase its distance from Eric by $s$, so after the squid's turn, Eric may be as much as $s+d=2s-5$ units away. Thus if $n<2s-5$, Eric is unable to capture the squid.

\bigskip

\textbf{Proof of lower bound:}     Since Eric moves at most $n$ times every turn, it suffices to show Eric can win when $n=2s-5$. In fact I claim Eric can win on his first move.

\begin{figure}[h]
    \begin{center}
        \begin{asy}
            size(3cm); defaultpen(fontsize(10pt));
            dotfactor *= 2;

            draw( (0,0)--(2,0)--(2,3)--(0,3),Arrow());
            dot( (0,0));
            draw( (-1,-1)--(-1,1)--(1,1)--(1,2)--(0,2)--(0,2.8),gray,Arrow());
            draw( (-1,-1)--(3,-1)--(3,4)--(0,4)--(0,3.2),gray,Arrow());
            dot( (-1,-1),filltype=FillDraw(fillpen=white,drawpen=black));
        \end{asy}
    \end{center}
    \caption{Eric's strategy}
    \label{fig:squid-strat}
\end{figure}

Without loss of generality the squid starts at $(0,0)$. We claim Eric can win on his first turn if he starts from $(-1,-s+6)$. Call the final position of the squid the \emph{destination}. Assume that the squid only places black ink. This is worse for Eric, and does not affect the squid's first turn.

First note that the squid cannot surround Eric, as that would take at least $s+1$ moves. Furthermore the squid cannot surround itself, as then it can only perform finitely many moves before losing.

We consider two paths, the ``right'' and ``left'' paths, from Eric to the destination; in summary, Eric reaches some point on the squid's path, and moves around the path in the two possible directions until it reaches the destination. We will show at least one of these two paths has length at most $n=2s-5$. Let $T$ be the sum of the lengths of the two paths.
\begin{claim}
    If the squid's ink ever blocks Eric from following the ink, then Eric is able to ``jump'' to another point on the squid's ink that is closer to the destination, and this decreases the length of Eric's path.
\end{claim}
\begin{proof}
    Without loss of generality, Eric is on the right-hand path. Say a \emph{blockade} is when some part of the ink blocks the square directly to the right of some square on the path. There are two possible blockades, as shown below.

    \begin{figure}[h]
        \begin{minipage}[b]{0.5\textwidth}
            \begin{center}
                \begin{asy}
                    size(3cm); defaultpen(fontsize(10pt));
                    dotfactor *= 2;

                    draw( (0,0)--(0,4)--(3,4)--(3,2)--(1,2)--(1,3)--(2,3),Arrow());
                    dot( (0,0));
                    draw( (1,0)--(1,1.5),gray,Arrow());
                    dot( (1,0),filltype=FillDraw(fillpen=white,drawpen=black));
                \end{asy}
            \end{center}
            \caption{Left-hand blockade}
            \label{fig:squid-lhblock}
        \end{minipage}
        \begin{minipage}[b]{0.5\textwidth}
            \begin{center}
                \begin{asy}
                    size(3cm); defaultpen(fontsize(10pt));
                    dotfactor *= 2;

                    draw( (0,0)--(0,4)--(3,4)--(3,3)--(1,3)--(1,2)--(3,2),Arrow());
                    dot( (0,0));
                    draw( (1,0)--(1,1.5),gray,Arrow());
                    dot( (1,0),filltype=FillDraw(fillpen=white,drawpen=black));
                \end{asy}
            \end{center}
            \caption{Right-hand blockade}
            \label{fig:squid-rhblock}
        \end{minipage}
    \end{figure}

    The blockade must be closer to the destination than the current position. Otherwise, we will have already dealt with the intersection before.

    In the case of a left-hand blockade, the squid has trapped itself in a closed region, and there are only finitely many squares it can reach. Thus the squid will eventually lose, contradiction. In the case of a right-hand blockade, once Eric reaches the blockade, he can turn right and skip a portion of the blockade. Thus this is in Eric's favor.
\end{proof}

We proceed with the computation. Eric's first step is to reach a point adjacent to some point on the squid's path. (It is possible that the destination is the only possible point Eric can reach in this manner.)

\begin{figure}[h]
    \begin{center}
        \begin{asy}
            size(3cm); defaultpen(fontsize(10pt));
            dotfactor *= 2;

            draw( (5,5)--(1,5)--(1,1)--(6,1)--(6,6),Arrow());
            dot( (5,5));
            draw( (0,0)--(0,6)--(5,6)--(5,6)--(5.8,6),gray,Arrow());
            draw( (0,0)--(7,0)--(7,6)--(6.2,6),gray,Arrow());
            dot( (0,0),filltype=FillDraw(fillpen=white,drawpen=black));
        \end{asy}
    \end{center}
    \caption{Surrounding the squid ink}
    \label{fig:squid-surround}
\end{figure}

It takes $s-5$ moves to reach a point adjacent to the closest point on the ink path. Once there, we split off into two different directions to surround the squid's path. As shown in Figure \ref{fig:squid-strat} and Figure \ref{fig:squid-surround}, the union of these two paths (with the first $s-5$ moves omitted) forms a cycle, and $T'=T-2(s-5)$ is the length of this cycle.

\begin{claim}
    $T'\le2(s+1)$.
\end{claim}
\begin{proof}
    Note that for each corner flanked by two sides of the cycle, the averages of the additional lengths the two surrounding paths gain equals the total length the corner contributes to the length of the squid's ink. Refer to the bottom-right corner in Figure \ref{fig:squid-strat}.

    Thus each unit along the ink corresponds to at most two units along the length of $T'$. Finally we need to consider the final move from a point adjacent to the destination onto the destination. This yields an additional $2$ units, so $T'\le2(s+1)$, as claimed.
\end{proof}

By definition, we have $T\le2(2s-4)$. Note that if $T<2(2s-4)$, by Pigeonhole, one of Eric's two choices has length at most $n=2s-5$, so Eric can win. Assume instead that equality holds.

\begin{figure}[h]
    \begin{center}
        \begin{asy}
            size(3cm); defaultpen(fontsize(10pt));
            dotfactor *= 2;

            draw( (0,0)--(4,0)--(4,4)--(0,4),Arrow());
            dot( (0,0));
            draw( (1,1)--(3,1)--(3,3)--(1.2,3),gray,Arrow());
            draw( (1,1)--(1,2.8),gray+dashed,Arrow());
            dot( (1,1),filltype=FillDraw(fillpen=white,drawpen=black));
        \end{asy}
    \end{center}
    \caption{Skipping the U-turn}
\end{figure}

If there are any \emph{U-turns}, as shown above, Eric can shorten one of the paths. Then the sum of his two choices is now less than $2(2s-4)$, so we may finish as above. Henceforth also assume there are no U-turns. Thus in the worst-case scenario, all of the squid's moves are either north or east.

\begin{figure}[h]
    \begin{center}
        \begin{asy}
            size(3cm); defaultpen(fontsize(10pt));
            dotfactor *= 2;

            draw( (1,1)--(4,1)--(4,3)--(6,3),Arrow());
            dot( (1,1));
            draw( (0,0)--(6,0)--(6,2.8),gray,Arrow());
            dot( (0,0),filltype=FillDraw(fillpen=white,drawpen=black));
        \end{asy}
    \end{center}
    \label{fig:squid-equality}
    \caption{The equality case}
\end{figure}

If the last move of the squid's turn is to the east, Eric can move all the way to the east, and then up. An analogous argument holds if the last move is to the north. The length of this path is equal to the taxicab distance from Eric to the destination, which is $(s-5)+s=2s-5=n$, as desired.

Finally, the answer is $n\ge2s-5=4035$, and we are done.

