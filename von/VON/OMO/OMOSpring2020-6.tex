author: Brandon Wang
desc: Rectangles in square
source: OMO Spring 2020/6
tags: [2020-03, answer, cmedium, combo, combo-geo, optimization, nice, waltz]

---

Alexis has $2020$ paintings, the $i$th one of which is a $1\times i$ rectangle for $i=1,2,\ldots,2020$. Compute the smallest integer $n$ for which she can place all of the paintings onto an $n\times n$ table without overlapping or hanging off the table.

---

The answer is $1430$. The lower bound is since the smallest square in which a $1\times2020$ rectangle can be inscribed is one with sides of length $\tfrac{2021}{\sqrt2}>1429$.

Now I claim all these rectangles can be inscribed in a square with sides of length $1011\sqrt2<1430$, which will complete the proof. Place the $1\times2020$ rectangle in the square as shown.
\begin{center}
    \begin{asy}
        size(4cm); defaultpen(fontsize(10pt));
        real t=5;
        filldraw( (0,0)--(0,t*sqrt(2))--(t*sqrt(2),t*sqrt(2))--(t*sqrt(2),0)--cycle,yellow+white+white+white+white,orange);
        //filldraw( (1/sqrt(2),sqrt(2))--(t*sqrt(2)-sqrt(2),t*sqrt(2)-1/sqrt(2))--(t*sqrt(2)-3*sqrt(2)/2,t*sqrt(2))--(0,3*sqrt(2)/2)--cycle,lightred+white,red);
        filldraw( (1/sqrt(2),0)--(t*sqrt(2),(t-0.5)*sqrt(2))--( (t-0.5)*sqrt(2),t*sqrt(2))--(0,1/sqrt(2))--cycle,lightblue+white,blue);
    \end{asy}
\end{center}
Let $S_k$ denote the set of positive integers not exceeding $k$ and with the same parity as $k$. The main claim:
\begin{claim*}
    $1\times i$ rectangles, for $i\in S_k$, fit in an isosceles right triangle with legs of length $\frac{k+2}{\sqrt2}$
\end{claim*}
\begin{proof}
    We induct with step size $2$. The base cases $k=1,2$ may be promptly verified. Now assume the hypothesis holds for $k-2$; we will prove it holds for $k$.
    \begin{center}
        \begin{asy}
            size(4cm); defaultpen(fontsize(10pt));
            real t=5;
            filldraw( (0,0)--(t*sqrt(2),0)--(0,t*sqrt(2))--cycle,yellow+white+white+white+white,orange);
            filldraw( ( (t-1)*sqrt(2),0)--( (t-1)*sqrt(2)+1/sqrt(2),1/sqrt(2))--(1/sqrt(2),(t-1)*sqrt(2)+1/sqrt(2))--(0,(t-1)*sqrt(2))--cycle,lightblue+white,blue);
        \end{asy}
    \end{center}
    Inscribe a $1\times k$ rectangle in a isosceles right triangle with legs $\tfrac{k+2}{\sqrt2}$ as shown above. Then the bottom-left isosceles right triangle has sides of length $\tfrac k{\sqrt2}$, so by the inductive hypothesis, the $1\times i$ rectangle for all the other $i$ in $S_k$ fit in the bottom-left triangle, as claimed.
\end{proof}

In the first diagram, the upper-left and bottom-right isosceles right triangles have leg length $\tfrac{2021}{\sqrt2}$, so by the claim, we can fit $1\times i$ rectangles for $i=1,3,\ldots,2019$ in one of them and $1\times i$ rectangles for $i=2,4,\ldots,2018$ in the other.

---

1430
