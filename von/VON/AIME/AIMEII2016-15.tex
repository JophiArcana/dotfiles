desc: Cauchy-schwarz on AIME
source: AIME II 2016/15
tags: [2019-12, answer, cmedium, alg, ineq]

---

For $1\leq i\leq 215$ let $a_i=\frac{1}{2^i}$ and $a_{216}=\frac{1}{2^{215}}$. Let $x_1,x_2,\ldots,x_{216}$ be positive real numbers such that \[ \sum\limits_{i=1}^{216} x_i=1 \text{\quad and \quad} \sum\limits_{1\leq i<j \leq 216} x_ix_j = \frac{107}{215}+ \sum\limits_{i=1}^{216} \frac{a_ix_i^2}{2(1-a_i)}.\]The maximum possible value of $x_2=\frac{m}{n}$, where $m$ and $n$ are relatively prime positive integers. Find $m+n$.

---

Multiplying by $2$, we obtain \[1-\sum_{i=1}^{216}x_i^2=\frac{214}{215}+\sum_{i=1}^{216}\frac{a_ix_i^2}{1-a_i}.\]
Conveniently, notice that $\sum_{i=1}^{216}a_i=1$, whence $\sum_{i=1}^{216}(1-a_i)=215$. It follows that by Titu's Lemma, \[\frac1{215}=\sum_{i=1}^{216}x_i^2\left(\frac1{1-a_i}\right)\ge\frac{\left(\displaystyle\sum_{i=1}^{216}x_i\right)^2}{\displaystyle\sum_{i=1}^{216}(1-a_i)}=\frac1{215},\]
so we are in the equality case. Then, the ratio $\frac{x_i}{1-a_i}$ is fixed. Remark that \[\sum_{i=1}^{216}x_i=\frac1{215}\sum_{i=1}^{216}(1-a_i),\]
so \[x_2=\frac{1-a_2}{215}=\frac1{215}\cdot\frac34=\frac3{860},\]
and the requested sum is $3+860=863$.

---

863
