desc: Angle bisectors intersect perpendicular bisector of angle bisector
source: AIME I 2020/13
tags: [2020-03, answer, chard, geo, length]

---

Point $D$ lies on side $BC$ of $\triangle ABC$ so that $\seg{AD}$ bisects $\angle BAC$. The perpendicular bisector of $\seg{AD}$ intersects the bisectors of $\angle ABC$ and $\angle ACB$ in points $E$ and $F$, respectively. Given that $AB=4$, $BC=5$, $CA=6$, the area of $\triangle AEF$ can be written as $\tfrac{m\sqrt n}p$, where $m$ and $p$ are relatively prime positive integers, and $n$ is a positive integer not divisible by the square of any prime. Find $m+n+p$.

---

\begin{center}
    \begin{asy}
        size(8cm); defaultpen(fontsize(10pt));

        pair A,B,C,I,D,M,T,Y,Z,EE,F;
        A=(0,3sqrt(7));
        B=(-1,0);
        C=(9,0);
        I=incenter(A,B,C);
        D=extension(A,I,B,C);
        M=(A+D)/2;
        T=extension(B,C,M,M+rotate(90)*(A-D));
        EE=extension(B,I,M,T);
        F=extension(C,I,M,T);

        draw(B--EE,gray+dashed);
        draw(C--F,gray+dashed);
        draw(circumcircle(A,B,C),gray);
        draw(A--T,gray);
        draw(T--EE);
        draw(A--B--C--A);
        draw(B--T);
        draw(A--D);

        dot("$A$",A,NW);
        dot("$B$",B,SW);
        dot("$C$",C,SE);
        dot("$D$",D,S);
        dot("$I$",I,dir(255));
        dot("$M$",M,dir(70));
        dot("$T$",T,SW);
        dot("$E$",EE,NE);
        dot("$F$",F,N);

        label("$2$",B--D,S);
        label("$3$",D--C,S);
        label("$4$",T--B,S);
    \end{asy}
\end{center}
Let the tangent to $(ABC)$ at $A$ intersect $\seg{BC}$ at $T$. Then angle-chasing (or noting $T$ is the center of the Apollonian circle of $A$, $D$ wrt.\ $\seg{BC}$) gives $TA=TD$, so $T$ lies on $\seg{EF}$.

Note the lengths $BD=2$, $CD=3$, $TB=4$ (conveniently integers!) by $BD/DC=AB/AC=2/3$ and $BT/TC=(AB/AC)^2=4/9$. Furthermore $AD^2=AB\cdot AC-BD\cdot CD$, so $AM=MD=3\sqrt2/2$. By Pythagorean theorem on $\triangle TDM$ we have $TM=3\sqrt{14}/2$.

It remains to compute $EF$. By Menelaus on $\triangle TDM$ with traversal $\seg{BIE}$, \[\frac{TE}{EM}=\frac{TB}{BD}\cdot\frac{DI}{DM}=4,\]
so $ME=TM/3$. Similarly by Menelaus on $\triangle TDM$ with traversal $\seg{CIF}$, \[\frac{TF}{FM}=\frac{TC}{CD}\cdot\frac{DI}{DM}=6,\]
so $MF=TM/7$. Hence \[EF=\frac{TM}3+\frac{TM}7=\frac{10}{21}TM=\frac{5\sqrt{14}}7,\]
and the area is \[\operatorname{Area}(\triangle AEF)=\frac{AM\cdot EF}2=\frac{3\sqrt2\cdot5\sqrt{14}}{2\cdot2\cdot7}=\frac{15\sqrt7}{14}.\]
The requested sum is $15+7+14=36$.

---

036
