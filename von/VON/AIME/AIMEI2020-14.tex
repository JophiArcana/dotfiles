desc: P(P(x))=0 has four distinct solutions
source: AIME I 2020/14
tags: [2020-03, answer, ctricky, alg, polynomial]

---

Let $P(x)$ be a quadratic polynomial with complex coefficients whose $x^2$ coefficient is $1$. Suppose the equation $P(P(x))=0$ has four distinct solutions, $x=3,4,a,b$. Find the sum of all possible values of $(a+b)^2$.

---

Let $P(x)=(x-r)(x-s)$. There are two cases: in the first case, $(3-r)(3-s)=(4-r)(4-s)$ equals $r$ (without loss of generality), and thus $(a-r)(a-s)=(b-r)(b-s)=s$. By Vieta's formulas $a+b=r+s=3+4=7$.

In the second case, say without loss of generality $(3-r)(3-s)=r$ and $(4-r)(4-s)=s$. Subtracting gives $-7+r+s=r-s$, so $s=7/2$. From this, we have $r=-3$.

Note $r+s=1/2$, so by Vieta's, we have $\{a,b\}=\{1/2-3,1/2-4\}=\{-5/2,-7/2\}$. In this case, $a+b=-6$.

The requested sum is $36+49=85$.

---

085
