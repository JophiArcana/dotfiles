desc: tau(n))+tau(n+1)=7
source: AIME I 2019/9
tags: [2019-10, answer, cmedium, nt, casework]

---

Let $\tau(n)$ denote the number of positive integer divisors of $n$. Find the sum of the six least positive integers $n$ that are solutions to $\tau(n)+\tau(n+1)=7$.

---

Clearly $n\ne 1$. Then, if $\{\tau(n),\tau(n+1)\}=\{2,5\}$, there must be primes $p$ and $q$ such that $p$ and $q^4$ are consecutive. One must be even, so only $n=16$ works.

If $\{\tau(n),\tau(n+1)\}=\{3,4\}$, then either $p^2$ and $q^3$ are consecutive, or $p^2$ and $qr$ are consecutive. In the former case, it is well-known that only $n=8$ works. Else, listing squares of primes, we find that $n=9,25,121,361$ are the smallest four that work, so the answer is $16+8+9+25+121+361=540$, the answer.

---

540
