desc: Midpoint circle
source: AIME I 2019/15
tags: [2019-10, answer, ctricky, geo, projective]

---

Let $\overline{AB}$ be a chord of circle $\omega$, and let $P$ be a point on chord $\overline{AB}$. Circle $\omega_1$ passes through $A$ and $P$ and is internally tangent to $\omega$. Circles $\omega_1$ and $\omega_2$ intersect at points $P$ and $Q$. Line $PQ$ intersects $\omega$ at $X$ and $Y$. Assume that $AP=5$, $PB=3$, $XY=11$, and $PQ^2=\tfrac mn$, where $m$ and $n$ are relatively prime positive integers. Find $m+n$.

---

\begin{center}
    \begin{asy}
        size(8cm);
        pair O, A, B, P, O1, O2, Q, X, Y;
        O=(0, 0);
        A=dir(140); B=dir(40);
        P=(3A+5B)/8;
        O1=extension((A+P)/2, (A+P)/2+(0, 1), A, O);
        O2=extension((B+P)/2, (B+P)/2+(0, 1), B, O);
        Q=intersectionpoints(circle(O1, length(A-O1)), circle(O2, length(B-O2)))[1];
        X=intersectionpoint(P -- (P+(P-Q)*100), circle(O, 1));
        Y=intersectionpoint(Q -- (Q+(Q-P)*100), circle(O, 1));

        draw(circle(O, 1));
        draw(circle(O1, length(A-O1)));
        draw(circle(O2, length(B-O2)));
        draw(A -- B); draw(X -- Y); draw(A -- O -- B); draw(O1 -- P -- O2);

        dot("$O$", O, S);
        dot("$A$", A, A);
        dot("$B$", B, B);
        dot("$P$", P, dir(70));
        dot("$Q$", Q, dir(200));
        dot("$O_1$", O1, SW);
        dot("$O_2$", O2, SE);
        dot("$X$", X, X);
        dot("$Y$", Y, Y);
    \end{asy}
\end{center}
Let $O_1$ and $O_2$ be the centers of $\omega_1$ and $\omega_2$, respectively. There is a homothety at $A$ sending $\omega$ to $\omega_1$ that sends $B$ to $P$ and $O$ to $O_1$, so $\overline{OO_2}\parallel\overline{O_1P}$. Similarly, $\overline{OO_1}\parallel\overline{O_2P}$, so $OO_1PO_2$ is a parallelogram. Moreover, \[\angle O_1QO_2=\angle O_1PO_2=\angle O_1OO_2,\]
whence $OO_1O_2Q$ is cyclic. However, \[OO_1=O_2P=O_2Q,\]
so $OO_1O_2Q$ is an isosceles trapezoid. Since $\overline{O_1O_2}\perp\overline{XY}$, $\overline{OQ}\perp\overline{XY}$, so $Q$ is the midpoint of $\overline{XY}$.

By Power of a Point, $PX\cdot PY=PA\cdot PB=15$. Since $PX+PY=XY=11$ and $XQ=11/2$, \[XP=\frac{11-\sqrt{61}}2\implies PQ=XQ-XP=\frac{\sqrt{61}}2\implies PQ^2=\frac{61}4,\]
and the requested sum is $61+4=65$.
\end{solution}
\begin{customsol}{2}
    Let the tangents to $\omega$ at $A$ and $B$ intersect at $R$. Then, since $RA^2=RB^2$, $R$ lies on the radical axis of $\omega_1$ and $\omega_2$, which is $\overline{PQ}$. It follows that \[-1=(A,B;X,Y)\stackrel{A}{=}(R,P;X,Y).\]
    Let $Q'$ denote the midpoint of $\overline{XY}$. By the Midpoint of Harmonic Bundles Lemma, \[RP\cdot RQ'=RX\cdot RY=RA^2=RP\cdot RQ,\]
    whence $Q=Q'$. Like above, $XP=\frac{11-\sqrt{61}}2$. Since $XQ=\frac{11}2$, we establish that $PQ=\frac{\sqrt{61}}2$, from which $PQ^2=\frac{61}4$, and the requested sum is $61+4=65$.
\end{customsol}

---

065
