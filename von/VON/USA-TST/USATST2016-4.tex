desc: Too many 0's in binary representation of sqrt(3)
source: USA TST 2016/4
tags: [2020-03, oly, medium, nt, pell]

---

Let $\sqrt3=1.b_1b_2b_3\ldots_{(2)}$ be the binary representation of $\sqrt3$. Prove that for any positive integer $n$, at least one of the digits $b_n$, $b_{n+1}$, $\ldots$, $b_{2n}$ equals $1$.

---

Assume for contradiction $b_n$, $\ldots$, $b_{2n}$ are zero, and consider \[\frac pq=1.b_1\ldots b_{n-1}=1.b_1\ldots b_{2n},\]
where $q=2^k$ for some $k\le n-1$. By hypothesis $0<\sqrt3-p/q<2^{-2n}$,
from which \[0<q\sqrt3-p<\frac q{4^n}\le\frac1{4q}.\]
Multiplying this by $0<q\sqrt3+p<2\sqrt3q$, we have \[0<3q^2-p^2<\frac{\sqrt3}2<1,\]
absurd since $p$ and $q$ are integers.
