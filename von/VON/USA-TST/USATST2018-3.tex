desc: Mathematical entrees
source: USA TST 2018/3
tags: [2020-01, oly, brutal, combo, graph, global, local, nice, involved]

---

At a university dinner, there are $2017$ mathematicians who each order two distinct entr\'ees, with no two mathematicians ordering the same pair of entr\'ees. The cost of each entr\'ee is equal to the number of mathematicians who ordered it, and the university pays for each mathematician's less expensive entr\'ee (ties broken arbitrarily). Over all possible sets of orders, what is the maximum total amount the university could have paid?

---

The answer is $63\binom{64}2+1=127009$. Interpret a graph $G$ as a graph where each vertex is an entr\'ee, and there is an edge between two entr\'ees if and only if some mathematician ordered both these entr\'ees.

Let $E(G)$ be the set of all edges of $G$. Then over all graphs $G$ with $2017$ edges, we seek to maximize \[S(G)=\sum_{\seg{V_iV_j}\in E(G)}\min\{\deg V_i,\deg V_j\}.\]
\setcounter{claim}0
\begin{claim}
    Let $G$ be a graph. If $G$ has either $\binom k2$ edges for some $k\ge3$, or $\binom k2+1$ edges for some $k\ge4$, then we can find a graph $G'$ with a universal vertex, the same number of edges, and $S(G')\ge S(G)$.
\end{claim}
\begin{proof}
    Let $n$ be the number of vertices in $G$. We prove the claim by induction on $n$. The base case is as follows:
    \begin{itemize}
        \item If $G$ has $\binom k2$ edges and $n=k$, then $G$ itself contains a universal vertex.
        \item If $G$ has $\binom k2+1$ edges and $n=k+1$ and $G$ does not contain a univerself itself, then I claim we may take $G'$ as any complete graph $K_k$ with any another vertex connected to one of the $k$ vertices of the $K_k$.

            Each vertex of $G$ has degree at most $k-1$. Take the vertex $V$ of minimum degree $m$. Clearly $m<k-1$, as otherwise \[|E(G)|=\frac{(k+1)(k-1)}2>\binom k2+1,\]
            contradiction. Then
            \begin{align*}
                S(G)&\le(k-1)\left[\binom k2+1-m\right]+m^2\\
                &=(k-1)\binom k2+(k-1)-m(k-1-m)\\
                &\le(k-1)\binom k2+1=S(G'),
            \end{align*}
            as desired.
    \end{itemize}
    Now the inductive step: assume the hypothesis holds for all graphs with less than $n$ vertices. Take the vertex $V$ of minimum degree $m$. In summary, we will delete the vertex $V$ and relocate all edges incident to $V$.

    Iterate through the neighbors $U$ of $V$. If we can find a vertex $W$ that is not a neighbor of $U$, then replace the edge $UV$ with $\seg{UW}$. Otherwise, delete $\seg{UV}$ and connect any two vertices not already connected by an edge. (If such an edge does not exist, then $G$ has less than $\binom{n-1}2$ edges. However $\binom{n-1}2\ge\binom k2$, contradiction.)

    At the end of this process, each vertex has a degree no less than its original degree, and each edge incident to $V$ is mapped to an edge whose minimum endpoints degree is at least $m$, so this new graph $G'$ has $n-1$ vertices, the same number of edges, and $S(G')>S(G)$. The claim is proven by applying the inductive hypothesis on $G'$ to generate a graph $G''$ with a universal vertex, the same number of edges, and $S(G'')\ge S(G')\ge S(G)$.
\end{proof}

Thus it suffices to prove the upper bound if $G$ has a universal vertex. The upper bound is immediate from the following claim.
\begin{claim}
    Let $G$ be a graph with a universal vertex.
    \begin{itemize}
        \item If $k\ge3$ and $G$ has at most $\binom k2$ edges, then $S(G)\le(k-1)\binom k2$.
        \item If $k\ge4$ and $G$ has at most $\binom k2+1$ edges, then $S(G)\le(k-1)\binom k2+1$.
    \end{itemize}
\end{claim}
\begin{proof}
    We prove inductively that if $k\ge3$ and for all graphs $G$ with $\binom k2$ edges, $S(G)\le(k-1)\binom k2$, then both bullet points hypothesis holds for $k+1$. Our base case is to check the first bullet point for $k=3$, but this is easy, so we omit the details.

    Now suppose the graph $G$ has $\binom{k+1}2+c$ edges, $c\in\{0,1\}$. Take the universal vertex $V$, and let the other vertices be $V_1$, $V_2$, $\ldots$, $V_n$. Note that if $G$ has $\binom k2$ edges, then $n\ge k$, and if $G$ has $\binom{k+1}2$ edges, then $n\ge k+1$.

    It follows that $G'$ has at most $\binom k2$ edges, so
    \begin{align*}
        S(G)&=S(G')+\left[\binom{k+1}2+c-n\right]+\sum_{i=1}^n\deg V_i\\
        &\le(k-1)\binom k2+\left[\binom{k+1}2+c-n\right]+n+2\left[\binom{k+1}2+c-n\right]\\
        &\le(k-1)\binom k2+3\left[\binom{k+1}2-k\right]+k+c\\
        &=k\binom{k+1}2+c,
    \end{align*}
    and the induction is complete.
\end{proof}

This establishes the upper bound. To see that $127009$ is attainable, consider a complete graph $K_{64}$ with another vertex connected to one of the $64$ vertices of the $K_{64}$. End proof.
