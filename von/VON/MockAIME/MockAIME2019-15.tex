desc: Iran lemma on excircles of orthic triangle
source: Mock AIME 2019/15
tags: [2019-10, answer, cbrutal, geo, iran-lemma, reference]
author: Eric Shen

---

In triangle $ABC$, $AB=11$, $BC=19$, and $CA=20$. Let $O$ denote the circumcenter of $\triangle ABC$, and $D$, $E$, and $F$ denote the feet of the altitudes from $A$, $B$, and $C$, respectively. Points $X$ and $Y$ are the feet of the perpendiculars from $E$ and $F$, respectively, to $\overline{AD}$. If $\overline{AO}$ intersects $\overline{EF}$ at $Z$, then there exists a point $T$ such that $\angle DTZ=90^\circ$ and $AZ=AT$. Suppose that $\overline{ZT}$ intersects $\overline{AD}$ at $P$. Then, there exist relatively prime positive integers $m$ and $n$ such that $\tfrac{PX}{PY}=\tfrac mn$. Find $m+n$.

---

\begin{center}
    \begin{asy}
        size(11cm);
        defaultpen(fontsize(10pt));

        pen pri=red;
        pen sec=orange;
        pen tri=fuchsia;
        pen qua=purple;
        pen fil=red+opacity(0.05);
        pen sfil=orange+opacity(0.05);
        pen tfil=fuchsia+opacity(0.05);

        pair A, B, C, M, D, EE, F, H, SS, X, Y, Z, T, P, Pp, K, Ep, Fp;
        A=(0, 20sqrt(105));
        B=(-41, 0);
        C=(320, 0);
        M=(B+C)/2;
        D=foot(A, B, C); EE=foot(B, C, A); F=foot(C, A, B);
        H=orthocenter(A, B, C);
        SS=extension(B, C, EE, F);
        X=foot(EE, A, D); Y=foot(F, A, D);
        Z=foot(A, EE, F);
        T=foot(Z, D, foot(H, EE, F));
        P=extension(A, D, Z, T);
        Pp=extension(A, M, EE, F);
        K=extension(A, A+B-C, EE, F);
        Ep=foot(A, D, F);
        Fp=foot(A, D, EE);

        filldraw(A -- B -- C -- cycle, fil, pri); draw(A -- D, pri);
        draw(EE -- X, tri); draw(F -- Y, tri);
        draw((2A-Z) -- Z, qua); filldraw(circumcircle(Z, Ep, Fp), sfil, sec);
        draw(extension(D, foot(A, D, EE), A+(0, length(Z-A)), A+(1, length(Z-A))) -- D -- extension(D, foot(A, D, F), A+(0, length(Z-A)), A+(1, length(Z-A))), sec); filldraw(D -- EE -- F -- cycle, sfil, sec);
        draw(A -- K -- SS -- B, pri); draw(EE -- F, sec); //draw(EE -- SS -- B);
        draw(D -- T -- (2A-Z), sec); draw(Z -- T, qua);
        filldraw(circumcircle(A, EE, F), fil, pri);
        // draw(EE -- M -- F, dashed);
        // draw(A -- M, dotted); draw(P -- Pp, dotted);
        draw(Ep -- Z, qua); draw(Fp -- Y, qua);

        dot("$A$", A, dir(67.5));
        dot("$B$", B, S);
        dot("$C$", C, SE);
        //dot("$M$", M, S);
        dot("$D$", D, S);
        dot("$E$", EE, dir(5));
        dot("$F$", F, dir(195));
        dot("$H$", H, dir(-32));
        dot("$S$", SS, SW);
        dot("$X$", X, NE);
        dot("$Y$", Y, dir(10));
        dot("$Z$", Z, dir(260));
        dot("$T$", T, dir(55));
        dot("$P$", P, dir(50));
        //dot("$P'$", Pp, S);
        dot("$K$", K, NE);
        dot("$E'$", Ep, W);
        dot("$F'$", Fp, dir(120));
        dot("$Z'$", 2A-Z, unit(A-Z));
    \end{asy}
\end{center}
Since wrt. $\angle A$, $\overline{BC}$ and $\overline{EF}$ are antiparallel, and furthermore $O$ and $H$ are isogonal conjugates wrt. $\triangle ABC$, $Z$ is the projection of $A$ onto $\overline{EF}$. Since $\triangle ABC$ is the excentral triangle of $\triangle DEF$, $Z$ is the $D$-extouch point of $\triangle DEF$. Denote by $E'$ and $F'$ the points where the $D$-excircle, $\omega_D$, touches $\overline{DF}$ and $\overline{DE}$, respectively.

Let $Z'$ be the antipode of $Z$ on $\omega_D$. Then, since $\measuredangle DTZ=90^\circ=\measuredangle Z'TZ$, we have that $D,T,Z'$ are collinear. Since $AXZE$ is cyclic, \[\measuredangle AZX=\measuredangle AEX=\measuredangle ACB=\measuredangle EFA=90^\circ-\measuredangle FAZ=\measuredangle AZE',\]
so $Z,X,E'$ are collinear. Similarly, $Z,Y,F'$ are collinear. It follows that \[-1=(E',F';Z',T)\stackrel Z=(X,Y;A,P).\]
Hence, \[\frac{PX}{PY}=\frac{AX}{AY}=\frac{AE\sin C}{AF\sin B}=\left(\frac{AB}{AC}\right)^2=\frac{121}{400},\]
and the requested sum is $121+400=521$.

\begin{remark}
    If $P_\infty$ denotes the point at infinity along $\overline{BC}$ and $P'$ denotes the point on $\overline{EF}$ such that $\overline{BC}\parallel\overline{PP'}$, then \[-1=(X,Y;A,P)\stackrel{P_\infty}=(E,F;K,P').\]
    However, it is known that $\overline{KA}$ is tangent to $(AEF)$, so $\overline{AP'}$ is the $A$-symmedian of $\triangle AEF$. Since $\overline{BC}$ and $\overline{EF}$ are antiparllel wrt. $\angle A$, line $AP'$ bisects $\overline{BC}$.
\end{remark}

---

521
