author: Tiger Che
desc: Quadrilateral ACPQ
source: Mock AIME 2019 Tiebreaker/R
tags: [2019-12, answer, ceasy, geo, area]

---

In triangle $ABC$, $AB=5$, $BC=8$, and $CA=7$. Let the internal angle bisector of $\angle BAC$ intersect $\overline{BC}$ at $P$, and let $Q$ be the point on $\overline{AB}$ distinct from $A$ such that $CQ=7$. The square of the area of quadrilateral $ACPQ$ can be expressed in the form $\tfrac mn$, where $m$ and $n$ are relatively prime positive integers. Find $m+n$.

---

\begin{center}
    \begin{asy}
        size(8cm);
        defaultpen(fontsize(10pt));

        pair A, B, C, P, X, Q;
        A=(0, 5sqrt(3)/2);
        B=(-5/2, 0);
        C=(11/2, 0);
        P=(7B+5C)/12;
        X=foot(C, A, B);
        Q=2X-A;

        draw(A -- B -- C -- A -- P);
        draw(C -- Q); draw(C -- X, dashed);
        draw(P -- Q, dashed);
        draw(rightanglemark(B, X, C));

        dot("$A$", A, N);
        dot("$B$", B, SW);
        dot("$C$", C, SE);
        dot("$P$", P, S);
        dot("$Q$", Q, W);
        dot("$R$", X, NW);
    \end{asy}
\end{center}
Check that \[\cos B=\frac{5^2+8^2-7^2}{2\cdot 5\cdot 8}=\frac12,\]
so $\angle B=60^\circ$. Then, \[[ABC]=\frac{5\cdot 8\cdot\sin A}2=10\sqrt3.\]
By the Angle Bisector Theorem, \[\frac{BP}{BC}=\frac{AB}{AB+AC}=\frac5{12}\implies BP=\frac{10}3.\]
Now, let $R$ be the foot of the altitude from $C$ to $\overline{AB}$. By HL, $\triangle CAR\cong\triangle CQR$. However, since $\angle B=60^\circ$, $BR=4$, so $QR=AR=1$. Hence, $BQ=3$, and \[[BPQ]=\frac{10/3\cdot 3\cdot\sin A}2=\frac{5\sqrt3}2,\]
so \[[ACPQ]=10\sqrt3-\frac{5\sqrt3}2=\frac{15\sqrt3}2\implies [ACPQ]^2=\frac{675}4,\]
and the requested sum is $675+4=679$.

---

679
