desc: ASCII sum
source: Mock AIME 2018/1
tags: [2019-12, answer, ctrivial, alg]
author: Eric Shen

---

The ASCII value of a digit is $48$ more than the digit. For instance, the digit $0$ has a value of $48$, while the digit $7$ has a value of $55$. Let $g(n)$ be defined as the sum of the ASCII values of the digits of $n$ for all positive integers $n$, when expressed in base $10$. For instance, $g(10)=97$ and $g(1234)=202$. The sum of all positive integers $n$ such that $g(n)=n$ is $N$. Find the remainder when $N$ is divided by $1000$.

---

First note that a one-digit integer cannot satisfy the condition, as $g(n)\ge 48$. In addition, integers with $k\ge 4$ digits cannot satisfy the condition as well, as $g(n)\le (48+9)k=57k<n$. Now, we only need to consider two-digit and three-digit integers.

If $n$ has two digits, let $n=\overline{ab}$. Then, $g(n)=(48+a)+(48+b)=96+a+b$. Also, $n=10a+b$, so $96+a+b=10a+b$, so $9a=96$. This is clearly not possible.

If $n$ has three digits, let $n=\overline{abc}$. Then, $g(n)=(48+a)+(48+b)+(48+c)=144+a+b+c$. Also, $n=100a+10b+c$. Thus, $144+a+b+c=100a+10b+c$, so $144=99a+9b$, and $9(11a+b)=144\implies 11a+b=16$. The only pair $(a, b)$ that satisfies this is $(1, 5)$. Since $c$ can be any digit, the desired sum is $150+151+\cdots+159=309\cdot 5=1545$, and the requested remainder is $545$.

---

545
