author: Eric Shen
desc: Incenter on circle
source: Mock AIME 2018/6
tags: [2019-12, answer, ceasy, geo, angle-chasing, length]

---

In triangle $ABC$, $X$ lies on $\overline{AB}$ and $Y$ lies on $\overline{AC}$ such that $\overline{BY}$ bisects $\angle ABC$ and $\overline{CX}$ bisects $\angle ACB$. Segments $BY$ and $CX$ intersect at a point $P$. Suppose that $P$ lies on the circumcircle of triangle $\triangle AXY$. If $AX=15$ and $AY=24$, find $AP^2$.

---

\begin{center}
    \begin{asy}
        size(8cm); defaultpen(fontsize(10pt));
        pair A, X, Y, P, B, C;
        A=(0, 20sqrt(3)/7);
        X=(-5/7, 0);
        Y=(44/7, 0);
        P=intersectionpoint(incenter(A, X, Y) -- (incenter(A, X, Y)+100*(incenter(A, X, Y)-A)), circumcircle(A, X, Y));
        B=extension(A, X, P, Y);
        C=extension(A, Y, P, X);
        draw(B -- C -- A -- B -- Y);
        draw(C -- X);
        draw(X -- Y, dashed);
        draw(A -- P, dashed);
        draw(circumcircle(A, X, Y));

        dot("$A$", A, NW);
        dot("$B$", B, W);
        dot("$C$", C, SE);
        dot("$X$", X, W);
        dot("$Y$", Y, E);
        dot("$P$", P, S);
    \end{asy}
\end{center}
Notice that \[180=\angle A+\angle XPY=\angle A+\angle BPC=\angle A+\left(90+\frac{\angle A}{2}\right)=90+\frac{3\angle A}{2},\]
whence $\angle A=60$.

Since $P$ is the incenter of $\triangle ABC$, $\overline{AP}$ is the angle bisector of $\angle A$. Then, \[\angle PAX=\angle PAY=\angle PXY=\angle PYX=30^\circ.\]
By the Law of Cosines on $\triangle AXY$, $XY=21$. Since $\triangle PXY$ is an isosceles triangle with an angle of $120^\circ$, $PX=PY=7\sqrt{3}$. Then, by Ptolemy's Theorem on quadrilateral $AXPY$, \[24\cdot 7\sqrt{3}+15\cdot 7\sqrt{3}=21\cdot AP\implies AP=13\sqrt3,\]
so $AP^2=507$.

---

507
