desc: Concentric circles
source: Mock AIME 2019/5
tags: [2019-10, answer, cmedium, alg, geo, trig, probability]
author: Eric Shen

---

Points $A$ and $B$ are randomly and uniformly chosen on the circumference of the circle $x^2+y^2=1$. Find the expected number of ordered pairs of real numbers $(p,q)$ such that the point $(p,q)$ lies on line $AB$ and there exists an integer $1\le k\le 45$ such that $p^2+q^2=\sin^2(k^\circ)$.

---

\begin{center}
    \begin{asy}
        size(6cm);
        defaultpen(fontsize(10pt));

        pen pri=red;
        pen sec=orange;
        pen fil=red+opacity(0.05);
        pen sfil=orange+opacity(0.05);

        pair O=(0, 0);
        pair A=(0, 1);
        pair B1=dir(230);
        pair B2=dir(310);

        filldraw(circle(0, 1), fil, pri);
        filldraw(circle(0, Sin(20)), sfil, sec);
        draw(B1 -- A -- B2, pri);
        draw(arc(0, B1, B2), pri+linewidth(2));
        draw(A -- O, sec+dashed); draw(foot(O, A, B1) -- O -- foot(O, A, B2), sec+dashed);
        draw(B1 -- O -- B2, pri);

        dot("$A$", A, N);
        dot("$B_1$", B1, B1);
        dot("$B_2$", B2, B2);
    \end{asy}
\end{center}
Let $O$ denote the origin. Suppose that $\omega_k$ denotes the circle centered at $O$ with radius $\sin k$. The problem is simply asking for the total number of times $AB$ intersects $\omega_k$ for all integers $1\le k\le 45$. We look at each $k$ independently. On the unit circle, fix point $A$ first.

Consider the locus of all $B$ such that $\overline{AB}$ intersects $\omega_k$. It is easy to check that this is the arc $\widehat{B_1B_2}$ not containing $A$ such that $\overline{AB_1}$ and $\overline{AB_2}$ are tangent to $\omega_k$. This is because for every point $P$ on $\omega_k$, the ray $AP$ lies within $\angle B_1AB_2$, and every ray within $\angle B_1AB_2$ must intersect $\omega_k$.

If $X$ denotes the point where $\overline{AB_1}$ touches $\omega_k$, it is easy to see that \[\widehat{B_1B_2}=2\angle B_1AB_2=4\angle XAO=4\arcsin\left(\frac{OX}{OA}\right)=4k.\]
Hence, the probability a given line $AB$ intersects $\omega_k$ is $\tfrac{4k}{360}=\tfrac{k}{90}$. The probability that $\overline{AB}$ is tangent to $\omega_k$ is infinitesimal and negligible, and if $\overline{AB}$ intersects $\omega_k$ but is not tangent to $\omega_k$, then $AB$ intersects $\omega_k$ at two points. Hence, the expected number of times $\overline{AB}$ intersects $\omega_k$ is $\tfrac{2k}{90}=\tfrac{k}{45}$. Summing over all such $k$, the answer is \[\sum_{i=1}^{45}\frac{k}{45}=\frac{45\cdot 46}{2\cdot 45}=23,\]
and we are done.

---

023
