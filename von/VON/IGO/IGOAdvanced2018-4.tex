desc: Bicentric chaos
source: IGO Advanced 2018/4
tags: [2019-10, oly, hard, geo, projective, weird, conditional]
author: Nikolai Beluhov

---

Let $ABCD$ be a tangential quadrilateral, and assume that diagonals $\overline{AC}$ and $\overline{BD}$ are not perpendicular. The angle bisectors formed by these two diagonals intersect sides $\overline{AB}$, $\overline{BC}$, $\overline{CD}$, and $\overline{DA}$ at points $K$, $L$, $M$, and $N$, respectively. Show that if quadrilateral $KLMN$ is cyclic, then quadrilateral $ABCD$ is also cyclic.

---

\begin{center}
    \begin{asy}
        size(10cm);
        defaultpen(fontsize(10pt));

        pen pri=red;
        pen sec=orange;
        pen tri=fuchsia;
        pen fil=pri+opacity(0.05);
        pen sfil=sec+opacity(0.05);
        pen tfil=tri+opacity(0.05);

        pair I, K, L, M, NN, A, B, C, D, EE, F, G, X, Y;
        I=(0, 0);
        K=dir(100);
        L=dir(20);
        M=2*foot(I, K, K-(0, 1))-K;
        NN=2*foot(I, L, foot(L, K, M))-L;
        A=2*(NN+K)/length(NN+K)^2;
        B=2*(K+L)/length(K+L)^2;
        C=2*(L+M)/length(L+M)^2;
        D=2*(M+NN)/length(M+NN)^2;
        EE=extension(K, M, L, NN);
        F=extension(K, L, M, NN);
        G=extension(K, NN, L, M);
        X=extension(A, B, C, D);
        Y=extension(A, D, B, C);

        filldraw(circumcircle(A, B, C), fil, pri);
        filldraw(circle(I, 1), sfil, sec);
        draw(F -- L -- M -- F, sec);
        draw(G -- M -- NN -- G, sec);
        draw(X -- B -- C -- X, pri);
        draw(Y -- C -- D -- Y, pri);
        draw(G -- X, tri);
        draw(F -- C, tri);
        draw(G -- D, tri);
        draw(K -- M, sec);
        draw(NN -- L, sec);

        dot("$A$", A, dir(100));
        dot("$B$", B, dir(80));
        dot("$C$", C, SE);
        dot("$D$", D, SW);
        dot("$K$", K, 2*N);
        dot("$L$", L, dir(-20));
        dot("$M$", M, S);
        dot("$N$", NN, SW/4);
        dot("$X$", X, W);
        dot("$Y$", Y, N);
        dot("$F$", F, NW);
        dot("$G$", G, NE);
        dot("$E$", EE, S);
    \end{asy}
\end{center}
Denote $E=\overline{AC}\cap\overline{BD}$, and let the incircle touch $\overline{AB}$, $\overline{BC}$, $\overline{CD}$, and $\overline{DA}$ at $K'$, $L'$ ,$M'$, and $N$, respectively. Let line $KM$ intersect $\overline{BC}$ and $\overline{DA}$ at $L^*$ and $N^*$, respectively, and let line $LN$ intersect $\overline{AB}$ and $\overline{CD}$ at $K^*$ and $M^*$, respectively.
\setcounter{claim}0
\begin{claim}
    $\overline{AC}$, $\overline{KL}$, and $\overline{MN}$ concur at point $F$, the harmonic conjugate of $E$ with respect to $\overline{AC}$. Similarly, $\overline{BD}$, $\overline{LM}$, and $\overline{NK}$ concur at point $G$, the harmonic conjugate of $E$ with respect to $\overline{BD}$.
\end{claim}
\begin{proof}
    The myriad angle bisectors yield the harmonic bundles $(AB;KK^*)$ and $(CB;LL^*)$, so $\overline{AC}$, $\overline{KL}$, and $\overline{K^*L^*}$  concur at a point $F$. Now, \[-1=(AB;KK^*)\stackrel{L^*}=(AC;EF).\]
    By symmetry, $\overline{AC}$, $\overline{KL}$, and $\overline{MN}$ concur at the harmonic conjugate of $E$ with respect to $\overline{AC}$.
\end{proof}
\begin{claim}
    $\overline{K'M'}$ and $\overline{L'N'}$ intersect at $E$; $\overline{AC}$, $\overline{K'L'}$, and $\overline{M'N'}$ concur at point $F$; and $\overline{BD}$, $\overline{L'M'}$, $\overline{N'K'}$ concur at point $G$.
\end{claim}
\begin{proof}
    By Brianchon's Theorem on $AKBCMD$ and $ABLCDN$, $E=\overline{K'M'}\cap\overline{L'N'}$. By Pascal's Theorem on $K'K'L'N'N'M'$, $F'=\overline{K'L'}\cap\overline{M'N'}$ lies on $\overline{AEC}$. It suffices to check that $-1=(AC;EF')$.

    Let $X=\overline{AB}\cap\overline{CD}$ and $Y=\overline{BC}\cap\overline{DA}$. By Pascal's Theorem on $K'K'L'M'M'N'$ and $K'L'L'M'N'N'$, $X$ and $Y$ lie on $\overline{FG}$. By Ceva-Menelaus, \[-1=(XY;GF')\stackrel B=(AC;EF'),\]
    as required.
\end{proof}
\begin{claim}
    $K=K'$, and similarly for cyclic variations.
\end{claim}
\begin{proof}
    We have proven that $(KLMN)$ and $(K'L'M'N')$ coincide as the polar circle of $\triangle EFG$. Since $K'$ lies on $\overline{AB}$, $K=K'$, and by symmetry our claim has been proven.
\end{proof}
\begin{claim}
    Quadrilateral $ABCD$ is cyclic.
\end{claim}
\begin{proof}
    This is just angle chasing: \[\angle BAD=180^\circ-\widehat{NK}=\widehat{LM}=180^\circ-\angle DCB,\]
    as desired.
\end{proof}

Hence, if quadrilateral $KLMN$ is cyclic, so is $ABCD$, and we are done.
