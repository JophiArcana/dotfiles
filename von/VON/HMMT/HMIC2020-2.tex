desc: Bishops and knights
source: HMIC 2020/2
tags: [2020-04, oly, medium, combo, graph, nice]

---

Some bishops and knights are placed on an infinite chessboard, where each square has side length $1$ unit. Suppose that the following conditions hold:
\begin{itemize}
    \item For each bishop, there exists a knight on the same diagonal as that bishop (there may be another piece between the bishop and the knight).
    \item For each knight, there exists a bishop that is exactly $\sqrt5$ units away from it.
    \item If any piece is removed from the board, then at least one of the above conditions is no longer satisfied.
\end{itemize}
If $n$ is the total number of pieces on the board, find all possible values of $n$.

---

The answer is $4\mid n$. 

To achieve $n=4k$, take $k$ copies of the following, spaced sufficiently far away from one another so that no two pieces from different arrangements attack each other.
\begin{center}
\begin{asy}
    size(4cm); defaultpen(fontsize(16pt));
    usepackage("skak");
    for (real x=0; x<=3+1e-9; x += 1)
    for (real y=0; y<=4+1e-9; y += 1) {
        if (abs( (x+y)/2-Floor( (x+y)/2))<1e-1)
        fill( (x,y)--(x+1,y)--(x+1,y+1)--(x,y+1)--cycle,lightgray);
    }
    draw( (0,0)--(4,0)--(4,5)--(0,5)--cycle);
    for (real x=1; x<=3+1e-9; x += 1)
    draw( (x,0)--(x,5),gray);
    for (real x=1; x<=4+1e-9; x += 1)
    draw( (0,x)--(4,x),gray);


    void att(pair x, pair y) {
        draw( (x+unit(y-x)*0.5)--(y+unit(x-y)*0.5),Arrow());
    }
    att( (1.5,0.5),(3.5,2.5));
    att( (3.5,2.5),(2.5,4.5));
    att( (2.5,4.5),(0.5,2.5));
    att( (0.5,2.5),(1.5,0.5));
    label("\symbishop",(1.5,0.5));
    label("\symknight",(3.5,2.5));
    label("\symbishop",(2.5,4.5));
    label("\symknight",(0.5,2.5));
\end{asy}
\end{center}

Now we show $4\mid n$ is necessary. Consider a directed graph where each node is either a bishop of a knight, and there is an edge from a bishop to a knight if the bishop attacks the knight, and vice versa. (No edges point from a bishop to a bishop, or a knight to a knight.)

We are given each node has outdegree at least $1$. For each node $v$, arbitrarily delete outgoing edges until the outdegree of $v$ is exactly $1$. Then each node has indegree $1$, as otherwise some node has indegree $0$ and may be deleted. Hence the graph consists of disjoint cycles.

It will suffice to show each cycle has length a multiple of $4$. Checkerboard color the chessboard: then for any edge from a bishop to a knight, the two endpoints are the same color, but for any edge from a knight to a bishop, the two endpoints are different colors. Hence the number of edges from a knight to a bishop is even, but this is equal to the number of edges from a bishop to a knight. It follows that the number of edges is a multiple of $4$, as desired.
