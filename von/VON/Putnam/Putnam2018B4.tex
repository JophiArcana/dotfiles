desc: Chebyshev x_{n+1}=2x_nx_{n-1}-x_{n-2}
source: Putnam 2018 B4
tags: [2020-03, oly, alg, medium, sequence, trig]

---

Given a real number $a$, we define a sequence by $x_0=1$, $x_1=x_2=a$, and $x_{n+1}=2x_nx_{n-1}-x_{n-2}$ for $n\ge 2$. Prove that if $x_n=0$ for some $n$, then the sequence is periodic.

---

Let $T_n(x)$ denote the $n^\text{th}$ Chebyshev polynomial\footnote{which obviously exists by strong induction.}, so that for all angles $\theta$, $T_n(\cos\theta)=\cos n\theta$. Also let $\{F_n\}$ denote the Fibonacci sequence, so that $F_0=0$, $F_1=1$, and $F_n=F_{n-1}+F_{n-2}$ for all $n\ge 2$.
\setcounter{claim}0
\begin{claim}
If $|a|\le 1$, then there exists an angle $\theta$ such that $x_n=\cos(F_n\theta)$ for all $n$.
\end{claim}
\begin{proof}
Let $\theta=\cos^{-1} a$. We use strong induction on $n$. The base cases, $n=0,1,2$, are trivial by definition. Now, assume the hypothesis holds for all integers not exceeding $k$. Check that
\begin{align*}
x_{n+1}&=2\cos(F_n\theta)\cos(F_{n-1}\theta)-\cos(F_{n-2}\theta)\\
&=2\cos(F_n\theta)\cos(F_{n-1}\theta)-\cos(F_n\theta-F_{n-1}\theta)\\
&=2\cos(F_n\theta)\cos(F_{n-1}\theta)-\big(\cos(F_n\theta)\cos(F_{n-1}\theta)+\sin(F_n\theta)\sin(F_{n-1}\theta)\big)\\
&=\cos(F_n\theta)\cos(F_{n-1}\theta)-\sin(F_n\theta)\sin(F_{n-1}\theta)\\
&=\cos(F_n\theta+F_{n-1}\theta)\\
&=\cos(F_{n+1}\theta),
\end{align*}
as desired.
\end{proof}
\begin{claim}
If $x_n=0$ for some $n$, then $|a|\le 1$.
\end{claim}
\begin{proof}
    Obviously $x_n$ is always a polynomial in $a$, so it is clear from Claim 1 that $x_n=T_{(F_n)}(a)$ for all $n$. However, \[T_{(F_n)}\left(\cos\left(\frac{(2k-1)\pi}{2F_n}\right)\right)\in\left\{\cos\left(\frac{\pi}2\right),\cos\left(\frac{3\pi}2\right)\right\}=\{0\}\]
for all $k=1,2,3,\ldots,F_n$, so all roots of $T_{(F_n)}$ are in the range $[-1,1]$. Hence, if $x_n=0$ for some $n$, then $|a|\le 1$.
\end{proof}

Suppose that $x_n=0$. It follows that $F_n\theta\equiv \pi\mod(2\pi)$, so $\pi/\theta$ is an integer, and $x_n$ can only take finitely many values. This implies that by the Pigeonhole Principle, there must exist $p$ and $q$ such that \[(x_p,x_{p+1},x_{p+2})=(x_q,x_{q+1},x_{q+2}),\]
and so by the recurrence, $\{x_n\}$ is periodic, as desired.
