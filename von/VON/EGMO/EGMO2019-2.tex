desc: Domino covering
source: EGMO 2019/2
tags: [2019-10, oly, medium, combo, equality]

---

Let $n$ be a positive integer. Dominoes are placed on a $2n\times2n$ board in such a way that every cell of the board is (orthogonally) adjacent to exactly one cell covered by a domino. For each $n$, determine the largest number of dominoes that can be placed in this way.

---

The answer is $\tfrac{n(n+1)}2$. We need to tile the board with the tiles of area $8$ that consist of a domino and its adjacent squares\textemdash with the condition that the boundaries of the tiles are permitted to go past the boundaries.
\begin{center}
    \begin{asy}
        size(4cm);

        void domino(pair P, bool orient) {
            if (orient) { // horiz
                filldraw(P -- (P+(2, 0)) -- (P+(2, 1)) -- (P+(0, 1)) -- cycle, lightgray, linewidth(1.25));
                //draw(P -- (P+(0, -1)) -- (P+(2, -1)) -- (P+(2, 0)) -- (P+(3, 0)) -- (P+(3, 1)) -- (P+(2, 1)) -- (P+(2, 2)) -- (P+(0, 2)) -- (P+(0, 1)) -- (P+(-1, 1)) -- (P+(-1, 0)) -- cycle, linewidth(1.5));
            } else { // vert
                filldraw(P -- (P+(0, 2)) -- (P+(1, 2)) -- (P+(1, 0)) -- cycle, lightgray, linewidth(1.25));
                //draw(P -- (P+(-1, 0)) -- (P+(-1, 2)) -- (P+(0, 2)) -- (P+(0, 3)) -- (P+(1, 3)) -- (P+(1, 2)) -- (P+(2, 2)) -- (P+(2, 0)) -- (P+(1, 0)) -- (P+(1, -1)) -- (P+(0, -1)) -- cycle, linewidth(1.5));
            }
        }

        domino((0, 0), false);
        domino((3, 0), true);
        domino((7, 0), true);
        domino((0, 4), false);
        domino((0, 8), false);
        domino((2, 2), false);
        domino((5, 2), true);
        domino((9, 2), false);
        domino((4, 4), false);
        domino((7, 4), false);
        domino((2, 6), false);
        domino((5, 7), true);
        domino((9, 6), false);
        domino((3, 9), true);
        domino((7, 9), true);

        clip((0, 0) -- (10, 0) -- (10, 10) -- (0, 10) -- cycle);

        for (int i = 0; i < 11; i += 1) {
            draw((i, 0) -- (i, 10));
            draw((0, i) -- (10, i));
        }

        draw((0, 0) -- (10, 0) -- (10, 10) -- (0, 10) -- cycle, linewidth(2));
        draw((2, 0) -- (10, 0) -- (10, 8) -- (2, 8) -- cycle, linewidth(2));
        draw((2, 2) -- (8, 2) -- (8, 8) -- (8, 2) -- cycle, linewidth(2));
        draw((4, 2) -- (8, 2) -- (8, 6) -- (4, 6) -- cycle, linewidth(2));
        draw((4, 4) -- (4, 6) -- (6, 6) -- (6, 4) -- cycle, linewidth(2));
    \end{asy}
\end{center}
First, we provide an inductive construction. For $n=1$, place one domino in the square. Now, refer to the above diagram. Add dominoes in a pattern around the border of the square such that the top left corner is always covered by a domino in the same direction as the first domino placed in the case $n=1$.

Now, it suffices to prove that $\tfrac{n(n+1)}2$ is maximal. Let $L$ be the set of squares that are adjacent to a domino-covered square but in the exterior of the board. Since each tile covers $8$ squares, the number of dominoes is $\tfrac{4n^2+|L|}8$; thus it suffices to show that \[\frac{4n^2+|L|}8<\frac{n(n+1)}2+1\iff |L|<4n+8.\]
Since dominos on the border of the square must be separated by at least $2$, the maximum possible number of border dominos is $2n-1$, so the number of squares in $L$ that are adjacent to a side of the board is at most $4n-2$. There are at most $4$ squares that are in $L$ but not adjacent to the board (that touch the corners of the board at one point), so $L\le 4n+2<4n+8$, as desired.
