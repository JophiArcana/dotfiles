desc: Circle at M with radius 0
source: Iran TST 2011/1
tags: [2019-10, oly, easy, geo, pop]

---

In acute triangle $ABC$, let $E$ and $F$ be the feet of the altitudes from $B$ and $C$, respectively. Let $M,K,L$ denote the midpoints of $\overline{BC}$, $\overline{ME}$, and $\overline{MF}$, respectively. Line $KL$ intersects the line through $A$ parallel to $\overline{BC}$ at $T$. Prove that $TA=TM$.

---

\begin{center}
    \begin{asy}
        size(10cm);
        defaultpen(fontsize(10pt));
        pair A, B, C, D, EE, F, H, M, NN, K, L, T;
        A=(0, 16);
        B=(-28+12sqrt(3), 0);
        C=(12sqrt(3)-4, 0);
        D=foot(A, B, C);
        EE=foot(B, C, A);
        F=foot(C, A, B);
        H=extension(B, EE, C, F);
        M=(B+C)/2;
        NN=circumcenter(A, F, EE);
        K=(M+EE)/2;
        L=(M+F)/2;
        T=extension(A, B+(0, 16), K, L);
        filldraw(A -- B -- C -- cycle, red+opacity(0.05), red);
        draw(A -- D, orange); draw(B -- EE, orange); draw(C -- F, orange);
        draw(L -- T -- A, red);
        draw(F -- M -- EE -- NN -- cycle, orange);
        draw(F -- EE, orange);
        draw(M -- A, red);
        filldraw(circumcircle(A, F, EE), orange+opacity(0.05), orange);
        dot("$A$", A, N);
        dot("$B$", B, SW);
        dot("$C$", C, SE);
        dot("$D$", D, S);
        dot("$E$", EE, NE);
        dot("$F$", F, W);
        dot("$M$", M, S);
        dot("$K$", K, dir(60));
        dot("$L$", L, S);
        dot("$T$", T, NE);
    \end{asy}
\end{center}
Let $\omega$ be the circle centered at $M$ with radius $0$. It is well-known that $\overline{TA}$, $\overline{ME}$, and $\overline{MF}$ are tangent to $(AEF)$, so line $KL$ is the radical axis of $(AEF)$ and $\omega$. It follows that $TA^2=\operatorname{Pow}(T,(AEF))=\operatorname{Pow}(T,\omega)=TM^2$, as desired.
