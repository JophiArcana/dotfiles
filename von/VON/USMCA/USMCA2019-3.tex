desc: AQD tangent BC
source: USMCA 2019/3
tags: [2020-03, oly, easy, geo, pop]

---

Let $ABC$ be a scalene triangle. The incircle of $ABC$ touches $\overline{BC}$ at $D$. Let $P$ be a point on $\overline{BC}$ satisfying $\angle BAP=\angle CAP$, and let $M$ be the midpoint of $\overline{BC}$. Define $Q$ to be on $\overline{AM}$ such that $\overline{PQ}\perp\overline{AM}$. Prove that the circumcircle of $\triangle AQD$ is tangent to $\overline{BC}$.

---

Let $I$ be the incenter of $\triangle ABC$, $X$ the foot of the altitude from $A$, and $L$ the midpoint of the arc $BC$ not containing $A$ on the circumcircle of $\triangle ABC$.
\begin{center}
    \begin{asy}
        size(6cm);
        defaultpen(fontsize(10pt));

        pen pri=deepgreen;
        pen sec=Cyan;
        pen tri=springgreen;
        pen qua=chartreuse;
        pen fil=pri+opacity(0.05);
        pen sfil=sec+opacity(0.05);
        pen tfil=tri+opacity(0.05);
        pen qfil=qua+opacity(0.05);

        pair A, B, C, X, I, D, L, P, M, Q;
        A=dir(120);
        B=dir(200);
        C=dir(340);
        X=foot(A, B, C);
        I=incenter(A, B, C);
        D=foot(I, B, C);
        L=dir(270);
        P=extension(A, L, B, C);
        M=(B+C)/2;
        Q=foot(P, A, M);

        draw(A -- B -- C -- A, pri);
        filldraw(circumcircle(A, B, C), tfil, pri);
        filldraw(incircle(A, B, C), tfil, pri);
        draw(A -- L, pri);
        draw(P -- Q, pri);
        draw(X -- A -- M, pri);
        filldraw(circumcircle(A, P, Q), sfil, sec);
        draw(A -- X, qua); draw(I -- D, qua); draw(L -- M, qua);

        dot("$A$", A, N);
        dot("$B$", B, SW);
        dot("$C$", C, SE);
        dot("$X$", X, S);
        dot("$I$", I, W);
        dot("$D$", D, S);
        dot("$L$", L, S);
        dot("$P$", P, SE);
        dot("$M$", M, SE);
        dot("$Q$", Q, dir(-10));
    \end{asy}
\end{center}
Remark that$$\frac{MQ\cdot MA}{MD^2}=\frac{MP\cdot MX}{MD^2}=\frac{LP\cdot LA}{LI^2}=\frac{LP\cdot LA}{LB^2}=1,$$where the first step comes from Power of a Point from $M$ to the circle with diameter $\overline{AP}$ (which passes through $X$ and $Q$), the second step comes from an affine transformation taking $\overline{XDPM}$ to $\overline{AIPL}$, the third comes from the Incenter-Excenter Lemma, and the fourth comes from the Shooting Lemma. Hence, $MQ\cdot MA=MD^2$, and the result follows. $\square$
