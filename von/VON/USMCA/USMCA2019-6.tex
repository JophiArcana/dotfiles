desc: Mirrored polynomial has unit root
source: USMCA 2019/6
tags: [2020-01, oly, hard, alg, trig, polynomial]

---

A \emph{mirrored polynomial} is a polynomial $f$ of degree $100$ with real coefficients such that the $x^{50}$ coefficient of $f$ is $1$, and $f(x)=x^{100}f(1/x)$ holds for all real nonzero $x$. Find the smallest real constant $C$ such that any mirrored polynomial $f$ satisfying $f(1)\ge C$ has a complex root $z$ obeying $|z|=1$.

---

As with above, set \[g(\theta)=\frac{f(e^{i\theta})}{2(e^{i\theta})^2}-\frac12=\frac12\sum_{k=1}^{50}a_k\left(e^{i\theta k}+e^{-i\theta k}\right)=\sum_{k=1}^{50}a_k\cos(k\theta).\]
\setcounter{claim}0
\begin{claim}
    If $f(1)\ge 51$, we can find a complex root $z$ of $f$ with $|z|=1$.
\end{claim}
\begin{proof}
    We have $g(0)\ge\tfrac{f(1)-1}2=25$ and will show that there exists real $\theta$ such that $g(\theta)=-\tfrac12$. Like above, check that \[\sum_{t=0}^{100}g\left(\frac{2\pi t}{101}\right)=\sum_{k=1}^{50}\left(a_k\sum_{t=0}^{100}\cos\left(\frac{2\pi k}{101}\cdot t\right)\right)=\sum_{k=1}^{50}\left(a_k\cdot\operatorname{Re}\left(\sum_{t=0}^{100}\cis\left(\frac{2\pi k}{101}\cdot t\right)\right)\right)=0.\]
    By the Pigeonhole Principle, there exists some $1\le t\le 100$ such that $g(\tfrac{2\pi t}{101})\le-\frac12$. Since $g$ is continuous and $g(0)\ge 25$, by the Intermediate Value Theorem there exists $\theta$ such that $g(\theta)=-\tfrac12$.
\end{proof}
\begin{claim}
    There exists $K=51-\veps$ such that if $f(1)=K$, then there exists $f$ such that $g(\theta)>-\tfrac12$ for all real $\theta$.
\end{claim}
\begin{proof}
    Set $0<\veps<1$. We take $a_k=(1-k/51)\veps$ for all $k$. Check that \[g(\theta)=\frac{\veps}{51}\sum_{k=1}^{50}(51-k)\cos(k\theta)=\frac{\veps}{51}\sum_{j=1}^{50}\sum_{k=1}^j\cos(k\theta).\]
    However, \[\sum_{k=1}^j\cos(k\theta)=\sum_{k=1}^j\frac{\sin\big((k+\tfrac12)\theta\big)-\sin\big((k-\tfrac12)\theta\big)}{2\sin\tfrac\theta2}=\frac{\sin\big((j+\tfrac12)\theta\big)}{2\sin\tfrac\theta2}-\frac12,\]
    and thus
    \begin{align*}
        g(\theta)&=\frac{\veps}{51}\sum_{j=1}^{50}\sum_{k=1}^j\cos(k\theta)\\
        &=\frac{\veps}{51}\left(\sum_{j=0}^{50}\frac{\sin\big((j+\tfrac12)\theta\big)}{2\sin\tfrac\theta2}\right)-\frac\veps2\\
        &=\frac{\veps}{51}\left(\sum_{j=0}^{50}\frac{\cos\big((j+1)\theta\big)+\cos(j\theta)}{4\sin^2\tfrac\theta2}\right)-\frac\veps2\\
        &=\frac\veps2\left(\frac{1-\cos51\theta}{1-\cos\theta}\right)-\frac\veps2\\
        &\ge-\frac\veps2>-\frac12,
    \end{align*}
    as desired.
\end{proof}

Hence, the answer is $C=51$.
