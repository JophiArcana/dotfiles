desc: Constructing seventh power residue
source: Taiwan TST 2014/3/Q4
tags: [2020-02, oly, easy, nt, expnt, cyclotomic]

---

Alice and Bob play the following game. They alternate selecting distinct nonzero digits (from $1$ to $9$) until they have chosen seven such digits, and then consider the resulting seven-digit number (i.e.\ $\seg{A_1B_2A_3B_4A_5B_6A_7}$). Alice wins if and only if the resulting number is the last seven decimal digits of some perfect seventh power. Determine which player has the winning strategy.

---

Alice has the winning strategy. I claim that $x^7$ attains all residues relatively prime to $10^7$ (so as long as $A_7\in\{1,3,7,9\}$, Alice wins).

First note that there is a primitive root $g$ modulo $5^7$, so powers of $g$ attain every residue modulo $5^7$. Since $\vphi(5^7)=4\cdot5^6$ is not divisible by $7$, every residue modulo $5^7$ not divisible by $5$ is of the form $g^{7k}$ for some $k$.

Finally we show $x^7$ attains all odd residues modulo $2^7$; it suffices to show that if $a^7\equiv b^7\pmod{2^7}$ but $a\not\equiv b\pmod{2^7}$, then $a$ and $b$ are both even. If one of $a$ and $b$ is even, then so is the other, so assume for contradiction $a$ and $b$ are both odd. Then $2^7\mid(a-b)b^7\Psi_7(a/b)$, where $\Psi_7$ denotes $7$th cyclotomic polynomial. If $2\mid\Psi_7(a/b)$, then $\ord_2(a/b)=7$. This implies $2\equiv1\pmod7$, which is absurd. Thus $2^7\mid a-b$, the desired contradiction.
