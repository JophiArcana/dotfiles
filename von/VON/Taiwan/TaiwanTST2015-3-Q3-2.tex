desc: Evan's (ST) and (BIC) orthogonal
source: Taiwan TST 2015/3/Q6
tags: [2019-10, oly, hard, geo, projective, iran-lemma]

---

In scalene triangle $ABC$ with incenter $I$, the incircle is tangent to sides $CA$ and $AB$ at points $E$ and $F$. The tangents to the circumcircle of $\triangle AEF$ at $E$ and $F$ meet at $S$. Lines $EF$ and $BC$ intersect at $T$. Prove that the circle with diameter $\overline{ST}$ is orthogonal to the nine-point circle of $\triangle BIC$.

---

\begin{center}
    \begin{asy}
        size(10cm);
        defaultpen(fontsize(10pt));

        pen pri=red;
        pen sec=orange;
        pen tri=fuchsia;
        pen fil=red+opacity(0.05);
        pen sfil=orange+opacity(0.05);
        pen tfil=fuchsia+opacity(0.05);

        pair A, B, C, I, D, EE, F, M, T, K, P, Q, NN, SS, R;
        A=dir(150);
        B=dir(220);
        C=dir(320);
        I=incenter(A, B, C);
        D=foot(I, B, C);
        EE=foot(I, C, A);
        F=foot(I, A, B);
        M=(B+C)/2;
        T=extension(B, C, EE, F);
        K=extension(A, M, EE, F);
        P=extension(B, I, EE, F);
        Q=extension(C, I, EE, F);
        NN=(EE+F)/2;
        SS=2*circumcenter((A+I)/2, EE, F)-(A+I)/2;
        R=extension(B, C, K, SS);

        filldraw(A -- B -- C -- cycle, fil, pri);
        filldraw(incircle(A, B, C), fil, pri);
        filldraw(circumcircle(A, EE, F), fil, pri);
        filldraw(circumcircle(D, P, Q), sfil, sec);
        filldraw(circumcircle(SS, NN, T), tfil, tri);
        draw(B -- T -- P, pri);
        draw(A -- SS, sec);
        draw(B -- P, sec);
        draw(C -- Q, sec);
        draw(K -- R, sec);
        draw(D -- K, tri);
        draw(A -- M, tri);
        draw(EE -- SS -- F, tri);

        dot("$A$", A, NW);
        dot("$B$", B, S);
        dot("$C$", C, SE);
        dot("$D$", D, SW);
        dot("$E$", EE, dir(60));
        dot("$F$", F, dir(120));
        dot("$M$", M, SE);
        dot("$T$", T, SW);
        dot("$K$", K, dir(80));
        dot("$P$", P, NE);
        dot("$Q$", Q, NW);
        dot("$N$", NN, dir(75));
        dot("$S$", SS, E);
        dot("$R$", R, S);
        dot("$I$", I, dir(245));
    \end{asy}
\end{center}
Let $\triangle DPQ$ be the orthic triangle of $\triangle IBC$, $M$ be the midpoint of $\overline{BC}$, $N=\overline{AI}\cap\overline{EF}$ be the midpoint of $\overline{EF}$, $K=\overline{AM}\cap\overline{EF}$, and $R=\overline{BC}\cap\overline{KS}$. It is well-known that $K\in\overline{ID}$, and furthermore $P,Q\in\overline{EF}$ by the Iran Lemma. Notice that \[-1=(BC;DT)\stackrel I=(PQ;KT)\text{ and }-1=(AI;SN)\stackrel K=(MD;RT),\]
where the former is from Ceva-Menelaus and the latter is since $S$ is the pole of $\overline{EF}$ with respect to $(AEIF)$. Hence, $\overline{KSR}$ is the polar of $T$ with respect to $(DPQ)$, the nine-point circle of $\triangle BIC$. By Self-Polar Orthogonality, $(ST)$ and $(DPQ)$ are orthogonal, and we win.
