desc: Golden ratio probabilistic
source: USAMO 2010/6
tags: [2019-10, oly, hard, combo, probabilistic, nice]

---

A blackboard contains $68$ pairs of nonzero integers. Suppose that for each positive integer $k$ at most one of the pairs $(k,k)$ and $(-k,-k)$ is written on the blackboard. A student erases some of the $136$ integers, subject to the condition that no two erased integers may add to 0. The student then scores one point for each of the $68$ pairs in which at least one integer is erased. Determine, with proof, the largest number $N$ of points that the student can guarantee to score regardless of which $68$ pairs have been written on the board.

---

Assume without loss of generality all pairs of the form $(k,k)$ obey $k>0$. Otherwise swap all instances of $k$ and $-k$. For each $k>0$, erase all instances of $k$ with probability $p$, and erase all instances of $-k$ with probability $1-p$. Then, for $k,a,b>0$:
\begin{itemize}[itemsep=0em]
    \item some element of $(k,k)$ is erased with probability $p$;
    \item some element of $(a,b)$, $a\ne b$ is erased with probability $1-(1-p)^2>p$;
    \item some element of $(a,-b)$  is erased with probability $1-p(1-p)>p$;
    \item some element of $(-a,-b)$ is erased with probability $1-p^2=p$.
\end{itemize}
Hence we expect at least $\lceil68p\rceil=43$ points.\\

Now we establish the upper bound. For each $k=1,\ldots,8$, take five instances of $(k,k)$, and for all $1\le a<b\le8$, take a single instance of $(-a,-b)$. Thus we have chosen $5\cdot8+\tbinom82=68$ pairs. It suffices to show at most $43$ points can be scored.

Iterate over $k=1,\ldots,8$ and choose whether we erase $k$ or $-k$. At any given point, we have chosen $t$ negative numbers, then we earn $7-t$ points by choosing $-k$, and we earn $5$ points by choosing $k$. Thus the maximum attainable score is $7+6+5+5+5+5+5+5=43$, the end.
