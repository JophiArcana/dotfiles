desc: Golden ratio probabilistic
source: USAMO 2010/6
tags: [2019-10, oly, hard, combo, probabilistic]

---

A blackboard contains $68$ pairs of nonzero integers. Suppose that for each positive integer $k$ at most one of the pairs $(k,k)$ and $(-k,-k)$ is written on the blackboard. A student erases some of the $136$ integers, subject to the condition that no two erased integers may add to 0. The student then scores one point for each of the $68$ pairs in which at least one integer is erased. Determine, with proof, the largest number $N$ of points that the student can guarantee to score regardless of which $68$ pairs have been written on the board.

---

The answer is $43$. Let $S=68$ and $\phi=\frac{1+\sqrt5}2$. I first claim that it is always possible to score $S/\phi$ points. Without loss of generality assume all pairs of the form $(k,k)$ obey $k>0$. For each positive integer $k$, take $+k$ with probability $1/\phi$ and $-k$ with probability $1-1/\phi=1/\phi^2$.

Suppose that $S\cdot a$ of the pairs are of the form $(k,k)$, with $k>0$, and of the pairs with unequal elements, $S\cdot b$ are of the form $(+,+)$, $S\cdot c$ are of the form $(+,-)$, and $S\cdot d$ are of the form $(-,-)$. Then the fraction of the pairs with at least one erased integer is
\begin{align*}
    &a\cdot\frac1{\phi^2}+b\left(1-\left(1-\frac1\phi\right)^2\right)+c\left(1-\frac1\phi\left(1-\frac1\phi\right)\right)+d\left(1-\frac1{\phi^2}\right)\\
    &\ge a\cdot\frac1\phi+(1-a)\left(1-\frac1{\phi^2}\right)=a\cdot\frac1\phi+(1-a)\cdot\frac1\phi=\frac1\phi,
\end{align*}
as claimed. Now consider the following construction showing $43$ is attainable: For each $1\le k\le 8$, take five instances of $(k,k)$ and for all $-8\le a<b\le-1$, take a single instance of $(a,b)$. Thus there are $40+28=68$ pairs. I claim the maximum possible number of points that can be scored is $43$.

For each $k=1,\ldots,8$, we can take either $k$ or $-k$, so make the choices iteratively. Assuming that at a given state, we have chosen a negative number a total of $t$ times, it is easy to see that the number of points gained by choosing $-k$ is $7-t$, but if we choose $+k$, we automatically gain $5$ points. Thus the maximum possible score is $7+6+5+5+5+5+5+5=43$, and we are done.
