desc: Angles and Pappus
source: USAMO 2018/5
tags: [2019-10, oly, easy, geo, angle-chasing, pop, pappus]

---

In convex cyclic quadrilateral $ABCD$, lines $AC$ and $BD$ intersect at $E$, lines $AB$ and $CD$ intersect at $F$, and lines $BC$ and $DA$ intersect at $G$. Suppose that the circumcircle of $\triangle ABE$ intersects line $CB$ at $B$ and $P$, and the circumcircle of $\triangle ADE$ intersects line $CD$ at $D$ and $Q$, where $C$, $B$, $P$, $G$ and $C$, $Q$, $D$, $F$ are collinear in that order. Prove that if lines $FP$ and $GQ$ intersect at $M$, then $\angle MAC=90^\circ$.

---

\paragraph{First solution, by Pappus' Theorem} This is a Miquel point problem with respect to quadrilateral $BPDQ$, as evidenced by the following three observations:
\begin{itemize}[itemsep=0em]
    \item $CB\cdot CP=CA\cdot CE=CD\cdot CQ$, so $BPDQ$ is cyclic.
    \item $\da AEP=\da ABP=\da ABC=\da ADC=\da ADQ=\da AEQ$, so $E$ lies on $\seg{PQ}$.
    \item It follows that $A$ is the Miquel point of $BDQP$.
\end{itemize}
Let $T=\seg{BQ}\cap\seg{DP}$; by properties of the Miquel point $\angle TAC=90\dg$. By Pappus theorem on $BQGDPF$, we have $M$, $A$, $T$ collinear, so $\angle MAC=90\dg$ as well.
\begin{center}
    \begin{asy}
        size(10cm);
        defaultpen(fontsize(10pt));
        pen pri=royalblue+linewidth(0.5);
        pen sec=deepgreen+linewidth(0.5);
        pen tri=heavycyan+linewidth(0.5);
        pen fil=royalblue+opacity(0.05);
        pen sfil=deepgreen+opacity(0.05);
        pen tfil=cyan+opacity(0.05);
        pair A, B, C, D, EE, F, G, P, Q, T, M;
        A=dir(165);
        B=dir(85);
        C=dir(350);
        D=dir(190);
        EE=extension(A, C, B, D);
        F=extension(A, B, C, D);
        G=extension(A, D, B, C);
        P=intersectionpoint(G -- (B+(G-B)*0.01), circumcircle(A, B, EE));
        Q=intersectionpoint(C -- (D+(C-D)*0.01), circumcircle(A, D, EE));
        T=extension(B, Q, D, P);
        M=extension(F, P, G, Q);

        draw(B -- F -- C, pri); draw(C -- G -- D, pri); draw(A -- C, pri); draw(B -- D, pri); filldraw(circumcircle(A, B, C), fil, pri);
        filldraw(circumcircle(A, B, EE), tfil, tri); filldraw(circumcircle(A, D, EE), tfil, tri);

        draw(F -- P, tri); draw(G -- Q, tri);
        draw(B -- T -- P, sec); draw(P -- Q, sec); filldraw(circumcircle(B, D, P), sfil, sec);
        draw(T -- M, heavygreen);

        dot("$A$", A, dir(155));
        dot("$B$", B, NE);
        dot("$C$", C, SE);
        dot("$D$", D, SW);
        dot("$E$", EE, dir(120));
        dot("$F$", F, SW);
        dot("$G$", G, N);
        dot("$P$", P, NE);
        dot("$Q$", Q, SE);
        dot("$T$", T, S);
        dot("$M$", M, dir(350));
    \end{asy}
\end{center}
\paragraph{Second solution, by harmonic bundles} First since \[\measuredangle BAC=\measuredangle BDC=\measuredangle EDQ=\measuredangle EAQ=\measuredangle CAQ,\]
$\overline{AC}$ bisects $\angle BAQ$ and similarly $\angle DAP$. Let their common external angle bisectors intersect $\overline{BC}$ and $\overline{DC}$ at $X$ and $Y$ respectively. Since $-1=(CX;PG)=(CY;FQ)$, lines $XY$, $FP$, $GQ$ concur at $M$. It is clear that $\seg{AM}\perp\seg{AC}$.


