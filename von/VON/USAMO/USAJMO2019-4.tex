desc: Can EF be tangent to excircle?
source: USAJMO 2019/4
tags: [2019-10, oly, trivial, geo]

---

Let $ABC$ be a triangle with $\angle ABC$ obtuse. The $A$-excircle is a circle in the exterior of $\triangle ABC$ that is tangent to side $BC$ of the triangle and tangent to the extensions of the other two sides. Let $E$, $F$ be the feet of the altitudes from $B$ and $C$ to lines $AC$ and $AB$, respectively. Can line $EF$ be tangent to the $A$-excircle?

---

It is not possible. Assume for contradiction line $EF$ is tangent to the $A$-excircle.

\paragraph{First solution, by similarity} Since $\angle B=90\dg$, we have $A$, $E$, $C$ collinear in that order, and also $A$, $B$, $F$ collinear in that order.

Note the $A$-excircle is tangent to all three sides of $\triangle AEF$; I contend it is its $A$-excircle as well. Indeed, the $A$-excircle lies in the interior of $\angle EAF$, but it touches $\seg{AC}$ farther away from $A$ than $E$.

But $\triangle ABC$, $\triangle AEF$ are inversely similar, and they share an $A$-exradius, so they are congruent. Now $AB=AE=AB\cos A$, so $\cos A=1$ and $\angle A=0\dg$, absurd.

\paragraph{Second solution, by length chasing} Let the $A$-excircle touch $\seg{AC}$, $\seg{AB}$, $\seg{EF}$ at $B'$, $S$, $C'$.
\begin{center}
    \begin{asy}
        size(6cm);
        defaultpen(fontsize(9pt));

        pen pri=royalblue;
        pen sec=Cyan;
        pen tri=deepgreen;
        pen fil=royalblue+opacity(0.05);

        pair A, B, C, EE, IA, Bp, Cp, SS, F;
        A=dir(140);
        B=dir(230);
        C=dir(310);
        EE=foot(B, A, C);
        IA=2*dir(270)-incenter(A, B, C);
        Bp=foot(IA, A, C);
        Cp=foot(IA, A, B);
        SS=intersectionpoints(circle(IA, length(Bp-IA)), circle((EE+IA)/2, length(EE-IA)/2))[1];
        F=extension(A, B, EE, SS);

        draw(extension(A, B, IA-(0, length(IA-Bp)), IA-(1, length(IA-Bp))) -- A -- extension(A, C, IA-(0, length(IA-Bp)), IA-(1, length(IA-Bp))), pri);
        draw(B -- C, pri);
        filldraw(circle(IA, length(Bp-IA)), fil, pri);
        fill(A--B--C--cycle,fil);
        draw(B -- EE, sec); draw(C -- F, sec);
        draw(EE -- F, tri); draw(Bp -- IA -- Cp, pri); draw(SS -- IA, tri);

        dot("$A$", A, N);
        dot("$B$", B, W);
        dot("$C$", C, NE);
        dot("$E$", EE, NE);
        dot("$I_A$", IA, SE);
        dot("$B'$", Bp, NE);
        dot("$C'$", Cp, W);
        dot("$S$", SS, dir(120));
        dot("$F$", F, W);
    \end{asy}
\end{center}
If $\rho$ denotes the semiperimeter, then
\begin{align*}
    BC\cos A&=EF=ES+FS=EB'+FC'\\
    &=(\rho-AE)+(\rho-AF)\\
    &=AB+BC+CA-(AB+CA)\cos A,
\end{align*}
from which $\cos A=1$, absurd.


