desc: Power sum always divisible by m
source: USAMO 2018/3
tags: [2019-10, oly, hard, nt, vp, sums, involved]

---

For a given integer $n\ge2$, let $\{a_1,a_2,\ldots,a_m\}$ be the set of positive integers less than $n$ that are relatively prime to $n$. Prove that if every prime that divides $m$ also divides $n$, then $a_1^k+a_2^k+\cdots+a_m^k$ is divisible by $m$ for every positive integer $k$.

---

For convenience, let $A(n)$ be the set of positive integers less than $n$ that are relatively prime to $n$, so that $|A(n)|=\varphi(n)$.

We begin with a lemma:
\begin{lemma*}
    For each $n$, $j$, and prime $p$, we have \[\nu_p\left(\sum_{x=1}^nx^j\right)\ge\nu_p(n)-1.\]
\end{lemma*}
\begin{proof}
    Let $e=\nu_p(n)$. Note the following: \[\sum_{x=1}^nx^j\equiv\frac n{p^e}\sum_{x=0}^{p^e-1}x^j\pmod{p^e},\]
    It will suffice to check $\nu_p\left(\sum_{x=0}^{p^e-1}x^j\right)\ge e-1$. Define \[T_k:=\sum_{\nu_p(x)=k}x^j,\quad\text{so that}\quad\sum_{x=0}^{p^e-1}x^j\equiv\sum_{k=0}^eT_k\pmod{p^e}.\]
    I contend $\nu_p(T_k)\ge e-1$ for each $k$, which will suffice.

    Fix $k$, and let $g$ be a primitive root modulo $p^{e-k}$. We have \[T_k\equiv p^{kj}\sum_{i=0}^{\varphi(p^{e-k})-1}g^{ij}\equiv\frac{g^{\varphi(p^{e-k})j}-1}{g^j-1}p^{kj}\pmod{p^e}.\]
    There are two cases:
    \begin{itemize}
        \item If $p-1\mid j$, then by LTE, \[\nu_p\left(\frac{g^{\varphi(p^{e-k})j}-1}{g^j-1}\right)=\nu_p\left(\varphi\left(p^{e-k}\right)\right)=e-k-1,\]
            so $\nu_p(T_k)=(e-k-1)+kj\ge e-1$.
        \item Otherwise $g^j-1$ contributes no powers of $p$, and \[\nu_p\left(g^{\varphi(p^{e-k})j}-1\right)\ge e-k\]
            since $g^{\varphi(p^{e-k})j}\equiv1\pmod{p^{e-k}}$, so $\nu_p(T_k)=(e-k)+kj\ge e$.
    \end{itemize}
    Thus the lemma is proven.
\end{proof}

Now we induct on the number of (not necessarily distinct) prime factors of $n$. The base case $n=2$ is easy to check. Recall that $n$ is always even, so we now proceed with the inductive step $n\to nq$, where $q$ is prime.
\setcounter{claim}0
\begin{claim}
    If $q\mid n$ and the problem statement holds for $n$, then it holds for $nq$.
\end{claim}
\begin{proof}
    Observe that \[A(nq)=\bigcup_{i=0}^{q-1}(A(n)+in).\]
    It follows that
    \begin{align*}
        \sum_{a\in A(nq)}a^k&=\sum_{i=0}^{q-1}\sum_{a\in A(n)+in}a^k
        =\sum_{i=0}^{q-1}\sum_{a\in A(n)}(a+in)^k\\
        &=\sum_{j=1}^k\left[\binom kjn^j\left(\sum_{i=0}^{q-1}i^j\right)\left(\sum_{a\in A(n)}a^{k-j}\right)\right]+q\sum_{a\in A(n)}a^k
    \end{align*}
    Recall that $\varphi(nq)=q\varphi(n)$. This is divisible by $\varphi(n)$ by the inductive hypothesis, and there are evidently enough factors of $q$.
\end{proof}
\begin{claim}
    If $q\nmid n$ and the problem statement holds for $n$, then it holds for $nq$.
\end{claim}
\begin{proof}
    Instead, observe that \[A(nq)=\bigcup_{i=0}^{q-1}(A(n)+in)\setminus qA(n).\]
    As computed above, we have
    \[\sum_{a\in A(nq)}a^k=\sum_{j=1}^k\left[\binom kjn^j\left(\sum_{i=0}^{q-1}i^j\right)\left(\sum_{a\in A(n)}a^{k-j}\right)\right]-\left(q^k-q\right)\sum_{a\in A(n)}a^k\]
    Here $\varphi(nq)=(q-1)\varphi(n)$. As in the first claim, this is divisible by $\varphi(n)$, so it suffices to verify there are enough factors of $p$ for $p\mid q-1$.

    The term $(q^k-q)\sum_{a\in A(n)}a^k$ is divisible by $(q-1)\varphi(n)$, so it remains to check each term of the left summation. Indeed, the $n^j$ term contributes at least $1$ factor of $p$, the $\sum_{i=0}^{q-1}i^j$ term contributes at least $\nu_p(q-1)-1$ factors of $p$ (by the lemma), and the $\sum_{a\in A(n)}a^{k-j}$ contributes $\nu_p(\varphi(n))$ factors of $p$.

    The inductive step follows.
\end{proof}

