desc: Power sum always divisible by k
source: USAMO 2018/3
tags: [2019-10, oly, brutal, nt, p-adic]

---

For a given integer $n\ge 2$, let $\{a_1,a_2,\ldots,a_m\}$ be the set of positive integers less than $N$ that are relatively prime to $n$. Prove that if every prime that divides $m$ also divides $n$, then $a_1^k+a_2^k+\cdots+a_m^k$ is divisible by $m$ for every positive integer $k$.

---

The key is to look at each prime factor of $m=\varphi(n)$. In this solution, we denote \[A(n)=\{a:\gcd(a,n)=1\}.\]
Now, we present four lemmas.
\begin{boxlemma}
    If $p$ is a prime and $e$ is a positive integer, then \[p^{e-1}\mid\sum_{a\in A(p^e)}a^k\]
    for all positive integers $k$.
\end{boxlemma}
\begin{proof}
    We use induction on $e$. The base case, $e=1$, is trivial. Now, notice that \[A\left(p^{e+1}\right)=\bigcup_{i=0}^{p-1}(ip^e+A\left(p^e\right)),\]
    whence \[\sum_{a\in A\left(p^{e+1}\right)}a^k\equiv p\left(\sum_{a\in A(p^e)}a^k\right)+p^{ek}\left(\sum_{i=0}^{p-1}i^k\right)\equiv 0\pmod{p^e},\]
    as desired.
\end{proof}
\begin{boxlemma}
    If $p$ is a prime and $e$ is a positive integer such that $p^e\mid n$, then $p^{e-1}\mid 1^k+2^k+\cdots+n^k$ for all positive integers $k$.
\end{boxlemma}
\begin{proof}
    Notice that \[\sum_{i=1}^n i^k\equiv\frac{n}{p^e}\sum_{i=1}^{p^e} i^k\pmod{p^{e-1}},\]
    so it is sufficient to prove the problem for $n=p^e$. To do this, we use induction. The base case, $e=1$, is trivial. Then, \[\sum_{i=1}^{p^e}i^k=\left(\sum_{a\in A(p^e)}a^k\right)+p_k\left(\sum_{i=1}^{p^{e-1}}i^k\right)\equiv 0\pmod{p^{e-1}},\]
    as desired.
\end{proof}
\begin{boxlemma}
    Let $q$ be a prime that divides $n$. Then, if the problem statement holds for $n$, it holds for $nq$.
\end{boxlemma}
\begin{proof}
    Consider all primes $p\ne q$ that divide $n$, and suppose that $e=\nu_p(m)$. Then, $e=\nu_p(\varphi(nq))$. Notice that \[\varphi(nq)=q\varphi(n)=mq.\]
    Since \[A(nq)=\bigcup_{i=0}^{q-1}(ni+A(n)),\]
    we have that \[\sum_{a\in A(nq)}a^k=\sum_{i=0}^{q-1}\left(\sum_{a\in A(n)}(ni+a)^k\right)=\sum_{j=0}^k\left(\binom kj n^{k-j}\left(\sum_{i=0}^{q-1}i^{k-j}\right)\left(\sum_{a\in A(n)}a^k\right)\right).\]
    However, $p^e\mid\sum_{a\in A(n)}a^k$, so $p^e\mid\sum_{a\in A(nq)}a^k$, as desired.
\end{proof}
\begin{boxlemma}
    Let $q$ be a prime that does not divide $n$. Then, if the problem statement holds for $n$, it holds for $nq$.
\end{boxlemma}
\begin{proof}
    Consider all primes $p$ that divide $n$, and suppose that $e=\nu_p(m)$ and $f=\nu_p(q-1)$. Also notice that \[\varphi(nq)=(q-1)\varphi(n)=m(q-1),\]
    so $e+f=\nu_p(\varphi(nq))$. Since \[A(nq)=\left(\bigcup_{i=0}^{q-1}(ni+A(n))\right)\setminus(qA(n)),\]
    we have that
    \begin{align*}
        \sum_{a\in A(nq)}a^k&=\left(\sum_{i=0}^{q-1}\left(\sum_{a\in A(n)}(ni+a)^k\right)\right)-q^k\left(\sum_{a\in A(n)}a^k\right)\\
        &=\left(\sum_{j=0}^{k-1}\left(\binom kj n^{k-j}\left(\sum_{i=0}^{q-1}i^{k-j}\right)\left(\sum_{a\in A(n)}a^k\right)\right)\right)-(q^k-q)\left(\sum_{a\in A(n)}a^k\right).
    \end{align*}
    However, it is given that $p\mid n^{k-j}$; by Lemma 2, $p^{f-1}\mid\sum_{i=0}^{q-1} i^{k-j}$; and $p^e\mid\sum_{a\in A(n)}a^k$. Furthermore, since $p^f\mid q-1\mid q^{k-1}-1\mid q^k-q$ and $p^e\mid\sum_{a\in A(n)}a^k$, we conclude that $p^{e+f}\mid\sum_{a\in A(nq)}a^k$, as desired.
\end{proof}

This is sufficient.
