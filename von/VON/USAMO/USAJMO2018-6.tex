desc: Karl's magic card trick
source: USAJMO 2018/6
tags: [2019-10, oly, hard, combo, anti]

---

Karl starts with $n$ cards labeled $1,2,3,\ldots,n$ lined up in a random order on his desk. He calls a pair $(a,b)$ of these cards swapped if $a>b$ and the card labeled $a$ is to the left of the card labeled $b$.

Karl picks up the card labeled $1$ and inserts it back into the sequence in the opposite position: if the card labeled $1$ had $i$ cards to its left, then it now has $i$ cards to its right. He then picks up the card labeled $2$ and reinserts it in the same manner, and so on until he has picked up and put back each of the cards $1$, $2$, $\ldots$, $n$ exactly once in that order.

For example, one such process for $n=4$ is \[3142\to3412\to2341\to2431\to2341.\]

Show that, no matter what lineup of cards Karl started with, his final lineup has the same number of swapped pairs as the starting lineup.

---

Consider another parallel process, such that whenever we move card $i$, we replace its label with $n+i$. Thus for starting position $3142$, the new process is \[3142\to3452\to6345\to6475\to6785.\]
In the final configuration, each card is exactly $n$ more than the final configuration of the original process, so it has the same number of inversions. But the new process preserves the number of inversions at each step, end proof.

