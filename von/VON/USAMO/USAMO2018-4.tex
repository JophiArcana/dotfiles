desc: Pigeonhole on remainders
source: USAMO 2018/4
tags: [2019-10, oly, easy, nt, global, pigeonhole]

---

Let $p$ be a prime, and let $a_1,\ldots,a_p$ be integers. Show that there exists an integer $k$ such that the numbers \[a_1+k,\;a_2+2k,\;\ldots,\;a_p+pk\]
produce at least $\frac12p$ distinct remainders upon division by $p$.

---

Consider $D$, the number of ordered triples $(i,j,k)$, with $1\le i\le j\le p$ and $0\le k<p$, such that $a_i+ik\equiv a_j+jk\pmod p$. Since $p$ is prime, \[k\equiv (i-j)^{-1}(a_j-a_i)\pmod p,\]
which is unique for all $i,j$. Then, $D=\tbinom p2=\tfrac{p(p-1)}2$. Hence, by the Pigeonhole Principle, there exists some $k$ such that the number of $(i,j)$ with the desired property is at most $\tfrac{p-1}2$, and so the number of distinct residues is at least $\tfrac p2$, as desired.
