desc: Posets on subsets
source: USAMO 2019/4
tags: [2019-10, oly, trivial, combo, induction]

---

Let $n$ be a nonnegative integer. Determine the number of ways that one can choose $(n+1)^2$ sets $S_{i,j}\subseteq\{1,2,\ldots,2n\}$, for integers $0\le i\le n$ and $0\le j\le n$ such that:
\begin{itemize}[itemsep=0em]
    \item for all $0\le i,j\le n$, the set $S_{i,j}$ has $i+j$ elements; and
    \item $S_{i,j}\subseteq S_{k,l}$ whenever $0\leq i\leq k\leq n$ and $0\leq j\leq l\leq n$.
\end{itemize}

---

The answer is $(2n)!\cdot2^{n^2}$.

\paragraph{First solution, by direct computation} Place the sets in an $(n+1)\times(n+1)$ grid shown below, and select the blue sets $S_{0,0}$, $S_{0,1}$, $\ldots$, $S_{0,n}$, $S_{1,n}$, $\ldots$, $S_{n,n}$. Each admits exactly one more element than the previous set in the sequence, so there are $(2n)!$ ways to select the blue sets.
\[
    \begin{bmatrix}
        {\color{blue}1357}&{\color{blue}12357}&{\color{blue}123457}&{\color{blue}1234578}&{\color{blue}12345678}\\
        {\color{blue}135}&1345&13457&134578&1345678\\
        {\color{blue}13}&135&1357&13578&135678\\
        {\color{blue}1}&15&157&1578&15678\\
        {\color{blue}\varnothing}&5&57&578&5678
    \end{bmatrix}
\]
It remains to select the remaining sets. First note that the conditions given in the problem statement are equivalent to $S_{i,j}\subset S_{i+1,j}$ and $S_{i,j}\subset S_{i,j+1}$ for all $0\le i,j\le n$ (by transitivity).

We determine the remaining sets in the following order: $S_{1,n-1}$, $S_{1,n-2}$, $\ldots$, $S_{1,0}$, $S_{2,n-1}$, $S_{2,n-2}$, $\ldots$, $S_{2,0}$, and so on. When it comes to determine $S_{i,j}$, we will have already chosen $S_{i-1,j}$ and $S_{i,j+1}$.
\[
    \begin{bmatrix}
        {\color{blue}1357}&{\color{blue}12357}\\
        {\color{blue}135}&{\color{red}1345}
    \end{bmatrix}
\]
It is given $S_{i,j}\setminus S_{i-1,j}$ is a single-element subset of $S_{i,j+1}\setminus S_{i-1,j}$, which has cardinality $2$; hence we have $2$ choices.

There were $(2n)!$ ways to select the initially blue sets, and $2^{n^2}$ ways to select the remaining sets. The conclusion follows.

\paragraph{Second solution, by bijection (Daniel Zhu)} Arrange the sets into the obvious $(n+1) \times (n+1)$ square grid. Here, we orient the grid diagonally so that $S_{0,0}$ is on the bottom and $S_{n,n}$ is on the top. Label an internal edge with $i$ if crossing an edge while going up adds the element $i$ to the set in the relevant square.
\begin{center}
    \begin{asy}
        usepackage("amssymb");
        size(8cm);
        void rdraw(path[] p, pen pen=defaultpen) {
            draw(rotate(45)*p,pen);
        }
        void rlabel(Label l,pair p,pen pen=defaultpen){
            label(l,rotate(45)*p,pen);
        }
        for (int i = 1; i < 5; ++i) {
            rdraw((i,0)--(i,4)^^(0,i)--(4,i),linewidth(0.25pt));
        }
        rdraw((0,1)--(1.7,1){right}..{down}(2,0.7)--(2,0),lightblue+linewidth(1pt));
        rdraw((0,2)--(0.7,2){right}..{down}(1,1.7)--(1,0),lightblue+linewidth(1pt));
        rdraw((0,3)--(0.7,3){right}..{down}(1,2.7)--(1,2.3){down}..{right}(1.3,2)--(2.7,2){right}..{down}(3,1.7)--(3,1.3){down}..{right}(3.3,1)--(4,1),lightblue+linewidth(1pt));
        rdraw((1,4)--(1,3.3){down}..{right}(1.3,3)--(2.7,3){right}..{down}(3,2.7)--(3,2.3){down}..{right}(3.3,2)--(4,2),lightblue+linewidth(1pt));
        rdraw((2,4)--(2,1.3){down}..{right}(2.3,1)--(2.7,1){right}..{down}(3,0.7)--(3,0),lightblue+linewidth(1pt));
        rdraw((3,4)--(3,3.3){down}..{right}(3.3,3)--(4,3),lightblue+linewidth(1pt));
        rdraw((0,0)--(0,4)--(4,4)--(4,0)--cycle,linewidth(1.5pt));
        string hlab = "514546346462";
        for (int x=0;x<4;++x) {
            for (int y=1;y<4;++y) {
                rlabel("$"+substr(hlab,x*3+y-1,1)+"$",(x+0.5,y),fontsize(9));
            }
        }
        string ylab = "114653333462";
        for (int x=1;x<4;++x) {
            for (int y=0;y<4;++y) {
                rlabel("$"+substr(ylab,x*4-4+y,1)+"$",(x,y+0.5),fontsize(9));
            }
        }
        rlabel("$\varnothing$",(0.5,0.5));
        rlabel("$5$",(0.5,1.5));
        rlabel("$15$",(0.5,2.5));
        rlabel("$145$",(0.5,3.5));
        rlabel("$1$",(1.5,0.5));
        rlabel("$15$",(1.5,1.5));
        rlabel("$145$",(1.5,2.5));
        rlabel("$1456$",(1.5,3.5));
        rlabel("$15$",(2.5,0.5));
        rlabel("$135$",(2.5,1.5));
        rlabel("$1345$",(2.5,2.5));
        rlabel("$13456$",(2.5,3.5));
        rlabel("$135$",(3.5,0.5));
        rlabel("$1345$",(3.5,1.5));
        rlabel("$13456$",(3.5,2.5));
        rlabel("$123456$",(3.5,3.5));
    \end{asy}
\end{center}

It follows from the $S_{i,j} \subseteq S_{k,\ell}$ condition that the set of edges with label $i$ must follow a path from the left end of the square to the right end of the square, so we can associate every valid arrangement of sets with a partitioning of the internal edges into $2n$ paths, labeled $1$, $2$, $\ldots$, $2n$, traveling from the left end of the square to the right end of the square. Indeed, this correspondence is reversible, since given the system of labeled paths one can associate a square with the set of all numbers $k$ so that the path labeled with $k$ passes below the square.

It suffices to show that there are $2^{n^2}$ ways to choose all the ways to partition the unlabeled edges into $2n$ unlabeled paths. However, since every edge is utilized, this is equivalent to, for each of the $n^2$ internal vertices, choosing whether the two paths that pass through the vertex end up crossing or not. Thus, we are done.


