desc: Inscribing equilateral pentagon
source: USAMO 2016/5
tags: [2019-10, oly, tricky, geo, spiral-sim, weird]

---

An equilateral pentagon $AMNPQ$ is inscribed in triangle $ABC$ such that $M\in\seg{AB}$, $Q\in\seg{AC}$, and $N,P\in\seg{BC}$. Let $S$ be the intersection of lines $MN$ and $PQ$. Denote by $\ell$ the angle bisector of $\angle MSQ$.

Prove that $\seg{OI}$ is parallel to $\ell$, where $O$ is the circumcenter and $I$ is the incenter of triangle $ABC$.

---

\paragraph{First solution, by complex numbers}     Let $X$, $Y$, $Z$ be the midpoints of arcs $BC$, $CA$, $AB$ not containing $A$, $B$, $C$ on the circumcircle of $\triangle ABC$. Toss on the complex plane, with the circumcircle of $\triangle ABC$ as the unit circle, so that $x+y+z$ denotes the incenter. We just need to show that $x+y+z$ is in the direction perpendicular to the external angle bisector of $\angle MSQ$, but since $MN=PQ$, the external angle bisector of $\angle MSQ$ is just \[(m-n)+(p-q)=(m-a)+(p-n)+(a-q)=ti(x+y+z),\]
where $t=AM=NP=QA$. This completes the proof.

\paragraph{Second solution, by spiral similarity}     Let $K=(SMQ)\cap(SNP)$ be the Miquel point of $MNPQ$, and let $X$ and $Y$ denote the midpoints of $\seg{MQ}$ and $\seg{MP}$ respectively. Let $\seg{AI}$ intersect $(ABC)$ again at the arc midpoint $L$, so that $LB=LI=LC$ by the Incenter-Excenter lemma.
\begin{center}
    \begin{asy}
        size(7.2cm); defaultpen(fontsize(9pt));
        pen pri=red;
        pen sec=orange;
        pen tri=fuchsia;
        pen qua=magenta;
        pen fil=pri+opacity(0.05);
        pen sfil=sec+opacity(0.05);
        pen tfil=tri+opacity(0.05);
        pen qfil=qua+opacity(0.05);

        pair NN,P,M,Q,A,B,C,SS,O,I,X,Y,L,K;
        NN=(-1/2,0);
        P=(1/2,0);
        M=NN+dir(140);
        Q=P+dir(105);
        A=intersectionpoints(circle(M,1),circle(Q,1))[0];
        B=extension(A,M,NN,P);
        C=extension(A,Q,NN,P);
        SS=extension(M,NN,P,Q);
        O=circumcenter(A,B,C);
        I=incenter(A,B,C);
        X=(M+Q)/2;
        Y=(NN+P)/2;
        L=extension(A,I,(B+C)/2,(B+C)/2+rotate(90)*(B-C));
        K=reflect(circumcenter(SS,M,Q),circumcenter(SS,NN,P))*SS;

        draw(X--Y,sec+dashed);
        draw(A--L,pri);
        filldraw(circumcircle(SS,NN,P),tfil,tri);
        filldraw(circumcircle(SS,M,Q),tfil,tri);
        fill(A--M--NN--P--Q--cycle,sfil);
        draw(Q--M--SS--Q,sec);
        filldraw(circumcircle(A,B,C),fil,pri);
        draw(A--B--C--A,pri);
        dot("$A$",A,unit(A-O));
        dot("$B$",B,W);
        dot("$C$",C,E);
        dot("$M$",M,dir(120));
        dot("$N$",NN,S);
        dot("$P$",P,SE);
        dot("$Q$",Q,dir(75));
        dot("$S$",SS,SE);
        dot("$O$",O,SE);
        dot("$I$",I,W);
        dot("$X$",X,NW);
        dot("$Y$",Y,S);
        dot("$L$",L,S);
        dot("$K$",K,dir(320));
    \end{asy}
\end{center}
Since $MN=PQ$, the spiral similarity at $K$ sending $\seg{MN}$ to $\seg{QP}$ is a congruence; id est we have $\triangle KMN\cong\triangle KQP$. Furthermore $KM=KQ$ and $KN=KP$, so $\seg{KX}\perp\seg{MQ}$ and $\seg{KY}\perp\seg{NP}$.

Note that the antipode of $K$ on $(SMQ)$ lies on $\ell$, whence $\da KXY=\da KMN=\da KMS=\da(\seg{KX},\ell)$. Thus $\seg{XY}\parallel\ell$, so we only need to prove $\seg{XY}\parallel\seg{IO}$. To do this, I contend $\triangle KXY\sim\triangle LIO$.

Recall $A$, $X$, $I$, $K$, $L$ are collinear, so the similarity is a homothety and is sufficient. The similarity follows from $\seg{KY}\parallel\seg{LO}$ and \[\frac{KX}{KY}=\frac{MQ}{NP}=\frac{MQ}{AQ}=\frac{BL}{OL}=\frac{IL}{OL}\]
since $\triangle AMQ\sim\triangle OBL$. This completes the proof.


