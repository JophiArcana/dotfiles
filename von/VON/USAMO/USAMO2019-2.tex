desc: Amazing conditional geometry
source: USAMO 2019/2
tags: [2019-10, oly, easy, geo, conditional, symmedians]

---

Let $ABCD$ be a cyclic quadrilateral satisfying $AD^2+BC^2=AB^2$. The diagonals of $ABCD$ intersect at $E$. Let $P$ be a point on side $AB$ satisfying $\angle APD=\angle BPC$. Show that line $PE$ bisects $\overline{CD}$.

---

\paragraph{First solution, by symmedians} There is a point $P$ on side $AB$ with $AP=AD^2/AB$ and $BP=BC^2/AB$. We have \[\frac{PA}{PB}=\left(\frac{AD}{BC}\right)^2=\left(\frac{EA}{EB}\right)^2\]
since $\triangle AED\sim\triangle BEC$. Then $\seg{EP}$ is the $E$-symmedian of $\triangle EAB$, so $\seg{EP}$ bisects $\seg{CD}$.

By monotonicity it remains to check $\angle APD=\angle BPC$. Since $AP\cdot AB=AD^2$, we have $\triangle APD\sim\triangle ADB$, and analogously $\triangle BPC\sim\triangle BCA$. Hence $\da APD=\da BDA=\da BCA=\da CPB$, end proof.

\paragraph{Second (elementary) solution, from official solutions packet} As before, construct $P$ so that $AP\cdot AB=AD^2$ and $BP\cdot BA=BC^2$. Then $\triangle APD\sim\triangle ADB$, $\triangle BPC\sim\triangle BCA$, so \[\theta:=\da APD=\da BDA=\da BCA=\da CPB.\]
The task is to show $\seg{PE}$ bisects $\seg{CD}$.
\begin{center}
    \begin{asy}
        size(7cm); defaultpen(fontsize(10pt));
        pen pri=blue;
        pen sec=lightblue;
        pen tri=purple+pink;
        pen fil=cyan+opacity(0.05);
        pen sfil=lightblue+opacity(0.05);
        pen tfil=purple+pink+opacity(0.05);

        pair O,A,B,C,Y,D,P,EE,K,L,M;
        O=(0,0);
        A=dir(160);
        B=reflect( (0,0),(0,1))*A;
        C=dir(-80);
        path omega=circle(O,1);
        path gamma=circle((A+B)/2, length(A-B)/2);
        path wb=circle(B,length(C-B));
        Y=intersectionpoint(gamma,wb);
        path wa=circle(A,length(Y-A));
        D=intersectionpoints(omega,wa)[1];
        P=extension(A,B,C,reflect(A,B)*D);
        EE=extension(A,C,B,D);
        K=extension(A,C,P,D);
        L=extension(B,D,P,C);
        M=(C+D)/2;

        filldraw(circumcircle(A,B,K),tfil,tri+dashed);
        draw(K--L,tri);
        draw(C--P--D,tri);
        filldraw(circumcircle(A,P,D),sfil,sec+dashed);
        filldraw(circumcircle(B,P,C),sfil,sec+dashed);
        draw(A--C,sec);
        draw(B--D,sec);
        filldraw(omega,fil,pri);
        draw(P--M,pri+Dotted);
        draw(A--B--C--D--cycle,pri);

        real t=1.1;
        clip( (-100,t)--(100,t)--(100,-100)--(-100,-100)--cycle);

        dot("$A$",A,dir(165));
        dot("$B$",B,dir(15));
        dot("$C$",C,S);
        dot("$D$",D,dir(255));
        dot("$P$",P,N);
        dot("$E$",EE,W);
        dot("$K$",K,dir(210));
        dot("$L$",L,dir(-30));
        dot(M);
    \end{asy}
\end{center}
Let $K=\seg{AC}\cap\seg{PD}$, $L=\seg{BD}\cap\seg{PC}$.
\setcounter{claim}0
\begin{claim}
    $APLD$ and $BPKC$ are cyclic.
\end{claim}
\begin{proof}
    These follow from $\da LDA=\theta=\da LPA$ and $\da BCK=\theta=\da BPK$ respectively.
\end{proof}
\begin{claim}
    $AKLB$ is cyclic.
\end{claim}
\begin{proof}
    We have $\da AKB=\da CKB=\da CPB=\theta$ and similarly $\da ALB=\theta$.
\end{proof}

By Reim's theorem on $(ABCD)$, $(AKLB)$, we have $\seg{KL}\parallel\seg{CD}$. If $M=\seg{PE}\cap\seg{CD}$, then $MC=MD$ by Ceva's theorem on $\triangle PCD$.

\paragraph{Third solution, by radical axes} Let $\Omega$ be the circumcircle of $ABCD$, let $\Gamma$ be the circle with diameter $AB$, and let $\omega_A$, $\omega_B$ be the circles centered at $A$, $B$ with radii $AD$, $BC$, respectively.
\begin{center}
    \begin{asy}
        size(9cm); defaultpen(fontsize(10pt));
        pen pri=blue;
        pen sec=lightblue;
        pen tri=purple+pink;
        pen qua=purple;
        pen fil=cyan+opacity(0.05);
        pen sfil=lightblue+opacity(0.05);
        pen tfil=purple+pink+opacity(0.05);

        pair O,A,B,C,Y,D,P,EE,M,Z,T,U,V;
        O=(0,0);
        A=dir(148);
        B=reflect( (0,0),(0,1))*A;
        C=dir(-70);
        path omega=circle(O,1);
        path gamma=circle((A+B)/2, length(A-B)/2);
        path wb=circle(B,length(C-B));
        Y=intersectionpoint(gamma,wb);
        path wa=circle(A,length(Y-A));
        D=intersectionpoints(omega,wa)[1];
        P=extension(A,B,C,reflect(A,B)*D);
        EE=extension(A,C,B,D);
        M=(C+D)/2;
        Z=reflect(A,B)*Y;
        T=2*A*B/(A+B);
        U=2*foot(O,D,P)-D;
        V=2*foot(O,C,P)-C;

        filldraw(gamma,tfil,tri+dashed);
        draw(A--T--B,qua);
        draw(D--U,tri);
        draw(C--V,tri);
        filldraw(wa,sfil,sec+dashed);
        filldraw(wb,sfil,sec+dashed);
        draw(A--C,sec);
        draw(B--D,sec);
        filldraw(omega,fil,pri);
        draw(P--M,pri+Dotted);
        draw(A--B--C--D--cycle,pri);

        dot("$A$",A,dir(165));
        dot("$B$",B,dir(15));
        dot("$C$",C,S);
        dot("$D$",D,SW);
        dot("$P$",P,N);
        dot("$E$",EE,dir(60));
        dot("$Y$",Y,dir(105));
        dot("$Z$",Z,dir(265));
        dot("$T$",T,N);
        dot("$U$",U,dir(75));
        dot("$V$",V,dir(150));
        dot(M);
    \end{asy}
\end{center}
The condition $AD^2+BC^2=AB^2$ means $\omega_A$, $\omega_B$ are orthogonal, so their intersection points $Y$, $Z$ lie on $\Gamma$. Let $P$ be the midpoint of $\seg{YZ}$. Then $P$ lies on $\seg{AB}$, the radical axis of $\Omega$, $\Gamma$, so $P$ is the common radical center of $\Omega$, $\Gamma$, $\omega_A$, $\omega_B$.

Let the tangents to $\Omega$ at $A$, $B$ meet at $T$, and let $\seg{DP}$, $\seg{CP}$ meet $\Omega$ again at $U$, $V$. Since $\seg{DU}$ is the radical axis of $\Omega$, $\omega_A$, we have $\seg{AT}\parallel\seg{DU}$. Similarly $\seg{BT}\parallel\seg{CV}$, so $\da APD=\da BAT=\da TBA=\da CPB$. By monotonicity, $P$ is the point described in the problem statement.

Finally we have \[\frac{PA}{PB}=\left(\frac{YA}{YB}\right)^2=\left(\frac{AD}{BC}\right)^2=\left(\frac{EA}{EB}\right)^2,\]
so $\seg{EP}$ is the $E$-symmedian of $\triangle EAB$, which bisects $\seg{CD}$.
\begin{remark}
    This was my solution in-contest, and I believe it is pretty motivated. The four circles in the problem are the most natural way to construct the diagram, and an in-scale diagram suggests $P$ lies on $\seg{YZ}$. The rest is not hard to prove.
\end{remark}

\paragraph{Fourth solution, by inversion (Evan Chen)} As noted above, the circle $\omega_A$ centered at $A$ with radius $AD$ is orthogonal to the circle $\omega_B$ centered at $B$ with radius $BC$. We let $\mathbf I_A$, $\mathbf I_B$ denote inversion with respect to $\omega_A$, $\omega_B$.

Let the radical axis of $\omega_A$, $\omega_B$ intersect $\seg{AB}$ at $P$; by design, $P=\mathbf I_A(B)=\mathbf I_B(A)$. This already implies that \[\da APD\stackrel{\mathbf I_A}=\da BDA=\da BCA\stackrel{\mathbf I_B}=\da CPB,\]
so by monotonicity, $P$ is the point described in the problem statement.
\begin{claim*}
    The point $K=\mathbf I_A(C)$ lies on $\omega_B$ and $\seg{DP}$. Similarly $L=\mathbf I_B(D)$ lies on $\omega_A$ and $\seg{CP}$.
\end{claim*}
\begin{proof}
    Since $\omega_A\perp\omega_B$, it follows that $K\in\omega_B$. For $K\in\seg{DP}$, note $ABCD$ is cyclic, so $P=\mathbf I_A(B)$, $K=\mathbf I_A(C)$, $D=\mathbf I_A(D)$ are collinear.
\end{proof}

Finally since $C$, $L$, $P$ collinear, $A$ is concyclic with $K=\mathbf I_A(C)$, $L=\mathbf I_A(L)$, $B=\mathbf I_A(B)$, i.e.\ $AKLB$ is cyclic. Hence $\seg{KL}\parallel\seg{CD}$ by Reim's theorem, and $\seg{PE}$ bisects $\seg{CD}$ by Ceva's theorem.


