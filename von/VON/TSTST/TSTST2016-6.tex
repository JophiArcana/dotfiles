desc: Gergonne point madness
source: TSTST 2016/6
tags: [2019-10, oly, brutal, geo, angle-chasing, pop, incidence, involved]
author: Danielle Wang

---

Let $ABC$ be a triangle with incenter $I$, and whose incircle is tangent to $\overline{BC}$, $\overline{CA}$, $\overline{AB}$ at $D$, $E$, $F$, respectively. Let $K$ be the foot of the altitude from $D$ to $\overline{EF}$. Suppose that the circumcircle of $\triangle AIB$ meets the incircle at two distinct points $C_1$ and $C_2$, while the circumcircle of $\triangle AIC$ meets the incircle at two distinct points $B_1$ and $B_2$. Prove that the radical axis of the circumcircles of $\triangle BB_1B_2$ and $\triangle CC_1C_2$ passes through the midpoint $M$ of $\overline{DK}$.

---

\begin{center}
    \begin{asy}
        size(9.5cm);
        defaultpen(fontsize(9pt));

        pen pri=royalblue;
        pen sec=Cyan;
        pen tri=green;
        pen qua=springgreen;
        pen qui=deepgreen;
        pen sex=deepcyan;
        pen fil=pri+opacity(0.05);
        pen sfil=sec+opacity(0.05);
        pen tfil=tri+opacity(0.05);
        pen qfil=qua+opacity(0.05);
        pen qifil=qui+opacity(0.05);
        pen sefil=sex+opacity(0.05);

        pair O,A,B,C,MA,MB,MC,I,D,EE,F,
        B1,B2,C1,C2,X,Y,Z,K,M,P,Q,G;
        O=(0,0);
        A=dir(130);
        B=dir(200);
        C=dir(340);
        MA=dir(270);
        MB=intersectionpoint((A+C)/2 -- (O+((A+C)/2-O)*100),circle(O,1));
        MC=intersectionpoint((A+B)/2 -- (O+((A+B)/2-O)*100),circle(O,1));
        I=incenter(A,B,C);
        D=foot(I,B,C);
        EE=foot(I,C,A);
        F=foot(I,A,B);
        B1=intersectionpoints(circumcircle(A,I,C),
        circumcircle(D,EE,F))[0];
        B2=intersectionpoints(circumcircle(A,I,C),
        circumcircle(D,EE,F))[1];	C1=intersectionpoints(circumcircle(A,I,B),
        circumcircle(D,EE,F))[0];	C2=intersectionpoints(circumcircle(A,I,B),
        circumcircle(D,EE,F))[1];
        X=(EE+F)/2;
        Y=(F+D)/2;
        Z=(D+EE)/2;
        K=foot(D,EE,F);
        M=(D+K)/2;
        P=intersectionpoints(circumcircle(B,B1,B2),
        circumcircle(C,C1,C2))[0];
        Q=intersectionpoints(circumcircle(B,B1,B2),
        circumcircle(C,C1,C2))[1];
        G=extension(B,EE,C,F);

        filldraw(circumcircle(A,B,C),fil,pri);
        filldraw(circumcircle(B,I,C),sfil,sec);
        filldraw(circumcircle(C,I,A),sfil,sec);
        filldraw(circumcircle(A,I,B),sfil,sec);
        filldraw(circumcircle(B,B1,B2),qfil,qua);
        filldraw(circumcircle(C,C1,C2),qfil,qua);
        filldraw(incircle(A,B,C),sfil,sec);
        clip((-1.1,-1.1)--(-1.1,1.1)--(1.1,1.1)--(1.1,-1.1)-- cycle);
        draw(A--MA,sec+dashed);
        draw(B--MB,sec+dashed);
        draw(C--MC,sec+dashed);
        draw(P--Q,sex);
        draw(B1--B2,qui);
        draw(C1--C2,qui);
        draw(Y--Z,qui);
        draw(Y--X,qui);
        draw(B--EE,tri);
        draw(C--F,tri);
        draw(D--K,sec);
        draw(A--B--C--A,pri);
        draw(D--EE--F--D,pri);

        dot("$A$",A,unit(A-I));
        dot("$B$",B,unit(B-I));
        dot("$C$",C,unit(C-I));
        dot("$D$",D,S);
        dot("$E$",EE,NE);
        dot("$F$",F,dir(210));
        dot("$I$",I,N);
        dot("$M_A$",MA,S);
        dot("$M_B$",MB,MB);
        dot("$M_C$",MC,MC);
        dot("$X$",X,dir(50));
        dot("$Y$",Y,dir(190));
        dot("$Z$",Z,dir(15));
        dot("$B_1$",B1,N);
        dot("$B_2$",B2,S);
        dot("$C_1$",C1,N);
        dot("$C_2$",C2,dir(240));
        dot("$K$",K,NW);
        dot("$P$",P,W);
        dot("$Q$",Q,dir(250));
        dot("$G$",G,SE);
        dot("$M$",(D+K)/2,dir(300));
    \end{asy}
\end{center}
Let $\overline{AI}$, $\overline{BI}$, $\overline{CI}$ intersect $(ABC)$ again at $M_A$, $M_B$, $M_C$, respectively, and let $X$, $Y$, $Z$ be the midpoints of $\overline{EF}$, $\overline{FD}$, $\overline{DE}$, respectively. Denote by $G$ the Gergonne point of $\triangle ABC$, and let $(BB_1B_2)$ and $(CC_1C_2)$ intersect at $P$ and $Q$.

By the Incenter-Excenter Lemma, $M_A$, $M_B$, $M_C$ are the circumcenters of triangles $BIC$, $CIA$, and $AIB$, respectively.
\setcounter{claim}0
\begin{claim}
    $\overline{AI}$, $\overline{EF}$, $\overline{B_1B_2}$, $\overline{C_1C_2}$, and $\overline{PQ}$ concur at $X$.
\end{claim}
\begin{proof}
    Clearly $\overline{AI}$ and $\overline{EF}$ intersect at $X$. By the Radical Axis Theorem on $(AEIF)$, $(DEF)$, and $(AIC)$, $\overline{B_1B_2}$ passes through $X$, and similarly $C\in\overline{C_1C_2}$. Finally, $XB_1\cdot XB_2=XC_1\cdot XC_2$, so $X$ lies on the radical axis of $(BB_1B_2)$ and $(CC_1C_2)$, which is $\overline{PQ}$, as desired.
\end{proof}
\begin{claim}
    $\overline{B_1B_2}$ and $\overline{C_1C_2}$ are the $E$- and $F$-midsegments of $\triangle DEF$.
\end{claim}
\begin{proof}
    Note that $\overline{B_1B_2}$ is the radical axis of $(AIC)$ and $(DEF)$, so $\overline{B_1B_2}$ is perpendicular to $\overline{BIM_B}$. However, so is $\overline{FD}$, so $\overline{FD}\parallel\overline{B_1B_2}$. By Claim 1, $X\in\overline{B_1B_2}$, so the claim is proven.
\end{proof}
\begin{claim}
    $G$ lies on $\overline{PQ}$.
\end{claim}
\begin{proof}
    First, since $B_1Z\cdot B_2Z=IZ\cdot CZ$, $\overline{BZ}$ is the radical axis of $(BB_1B_2)$ and $(BIC)$. By the Radical Axis Theorem on $(BB_1B_2)$, $(CC_1C_2)$, and $(BIC)$, $\overline{BZ}\cap\overline{CY}$ lies on $\overline{PQ}$.

    Now, by Cevian Nest on $\triangle GBC$, $\triangle DEF$, and $\triangle XYZ$, $\overline{GX}$ lies on $\overline{PQ}$, whence $G$ lies on $\overline{PQ}$, as desired.
\end{proof}

Since $G$ is the symmedian point of $\triangle DEF$, it is well-known that $M$, $G$, $X$ are collinear, whence $M$ lies on $\overline{PQ}$, as desired.
