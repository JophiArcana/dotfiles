desc: n^2 divides b^n+1
source: TSTST 2018/8
tags: [2020-01, oly, hard, nt, expnt, zsig, nice]

---

For which positive integers $b>2$ do there exist infinitely many positive integers $n$ such that $n^2$ divides $b^n+1$?

---

The answer is all $b$ such that $b+1$ is not a power of $2$.
\setcounter{claim}0
\begin{claim}
    All such $n$ are odd.
\end{claim}
\begin{proof}
    Assume for contradiction $n=2k$ for some $k$. Since $b^2\equiv0,1\pmod4$, \[4\mid b^{2k}+1=\left(b^2\right)^k+1\equiv1,2\pmod4,\]
    which is absurd.
\end{proof}
\begin{claim}
    If $b+1$ is a power of $2$ and $n^2\mid b^n+1$, then $n=1$.
\end{claim}
\begin{proof}
    Assume that $b+1$ is a power of $2$, and let $p$ be the least odd prime dividing some $n>1$. Then $b^{2n}\equiv1\pmod p$ and $b^{p-1}\equiv1\pmod p$. Since $\gcd(2n,p-1)=2$, we must have $b^2\equiv1\pmod p$, or $p\mid b^2-1$. Since $b+1$ is a power of $2$, $p$ divides $b-1$. It follows that \[p\mid b^n+1\equiv2\pmod p,\]
    contradiction, thus no $b$ with $b+1$ a power of $2$ work.
\end{proof}
\begin{claim}
    If $p$ is an odd prime dividing $b+1$, then $p^2\mid b^p+1$.
\end{claim}
\begin{proof}
    Indeed, we take an odd prime $p$ dividing $b+1$, and note that \[\nu_p\left(b^p+1\right)=\nu_p(b+1)+1\ge2\]
    by lifting the exponent, so $p^2\mid b^p+1$, as claimed.
\end{proof}
\begin{claim}
    If $n>1$ and $n^2\mid b^n+1$, then there is an odd prime $p\nmid n$ with $(np)^2\mid b^{np}+1$.
\end{claim}
\begin{proof}
    The pith of this claim is that by Zsigmondy theorem, we can choose a prime $p$ dividing $b^n+1$ but not $n$. Then $n^2\mid b^n+1\mid b^{np}+1$ and \[\nu_p\left(b^{np}+1\right)=\nu_p\left(b^n+1\right)+1\ge2,\]
    so $p^2\mid b^{np}+1$, which proves the claim.
\end{proof}

Claim 2 ensures that no $b$ with $b+1$ a power of $2$ satisfy the problem condition. On the other hand, if $b+1$ is not a power of $2$, Claim 3 shows there are $n>1$ with $n^2\mid b^n+1$, and Claim 4 proves  there are infinitely many such $n$. This completes the proof.
\begin{remark}
    This problem is just a combination of IMO 1990/3 and IMO 2000/5.
\end{remark}
