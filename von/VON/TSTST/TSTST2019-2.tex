desc: Michael Ma complex bash in front of class
source: TSTST 2019/2
tags: [2019-10, oly, easy, geo, steiner, angle-chasing]

---

Let $ABC$ be an acute triangle with circumcircle $\Omega$ and orthocenter $H$. Points $D$ and $E$ lie on segments $AB$ and $AC$ respectively, such that $AD=AE$. The lines through $B$ and $C$ parallel to $\overline{DE}$ intersect $\Omega$ again at $P$ and $Q$, respectively. Denote by $\omega$ the circumcircle of $\triangle ADE$.
\begin{itemize}[itemsep=0em]
    \item[(a)] Show that lines $PE$ and $QD$ meet on $\omega$.
    \item[(b)] Prove that if $\omega$ passes through $H$, then lines $PD$ and $QE$ meet on $\omega$ as well.
\end{itemize}

---

\begin{customsol}{to part (a)}
    \begin{center}
        \begin{asy}
            size(10cm);
            defaultpen(fontsize(10pt));

            pair O, A, B, C, H, L, P, Q, Y, D, EE, X, SS, T;
            O=(0,0);
            A=dir(110);
            B=dir(220);
            C=dir(320);
            H=A+B+C;
            L=dir(270);
            P=2*foot(O,B,foot(B,A,L))-B;
            Q=2*foot(O,C,foot(C,A,L))-C;
            Y=A+P+Q;
            D=extension(A,B,P,Y);
            EE=extension(A,C,Q,Y);
            X=extension(P,EE,Q,D);
            SS=foot(P,A,Q);
            T=foot(Q,A,P);

            draw(circle(O,1));
            draw(A--B--C--A);
            draw(A--L,gray);
            draw(B--P);
            draw(C--Q);
            draw(circumcircle(A,D,EE),gray);
            draw(D--EE);
            draw(P--X--Q,gray);
            draw(P--D,gray);
            draw(Q--T,gray);
            draw(A--P--Q--A);
            draw(H--A--Y,dashed);

            dot("$A$",A,N);
            dot("$B$",B,B);
            dot("$C$",C,C);
            dot("$H$",H,S);
            dot("$L$",L,S);
            dot("$P$",P,P);
            dot("$Q$",Q,Q);
            dot("$Y$",Y,unit(Y-A));
            dot("$D$",D,dir(210));
            dot("$E$",EE,dir(18));
            dot("$X$",X,NW);
            dot("$S$",SS,unit(D-P)*dir(45));
            dot("$T$",T,NE);
        \end{asy}
    \end{center}
    Let $X=\overline{PE}\cap\overline{QD}$. It is easy to see that $\measuredangle APQ=\measuredangle ACQ=\measuredangle AED$ and $\measuredangle PQA=\measuredangle PBA=\measuredangle EDA$, whence $\triangle ADE\sim\triangle APQ$, and $A$ is the Miquel point of $DEPQ$.

    Hence, $X$ lies on both $\Omega$ and $\omega$, as desired. 
\end{customsol}
\begin{customenv}{First solution to part (b), using reverse construction}
    Let $Y$ be the orthocenter of $\triangle APQ$ and set $D=\overline{AB}\cap\overline{PY}$ and $E=\overline{AC}\cap\overline{QY}$. Also let $S=\overline{AQ}\cap\overline{PY}$ and $T=\overline{AP}\cap\overline{QY}$.
    \setcounter{iclaim}0
    \begin{iclaim}
        $AD=AE$ and $Y$ lies on $\omega$.
    \end{iclaim}
    \begin{proof}
        Since $\widehat{BQ}=\widehat{CP}$, $\triangle ADS\cong\triangle AET$. It is immediate that $AD=AE$, and furthermore \[\measuredangle DYE=\measuredangle SYT=\measuredangle SAT=\measuredangle QAP=\measuredangle DAE,\]
        as desired.
    \end{proof}
    \begin{iclaim}
        $H$ lies on $\omega$.
    \end{iclaim}
    \begin{proof}
        Let $O$ be the center of $\Omega$. Clearly $AP=AQ$, $OP=OQ$, and $YP=YQ$, so $A,O,P$ lie on the perpendicular bisector of $\overline{PQ}$.

        Now, if $R$ denotes the radius of $\Omega$, $AH=2R\cos\angle BAC=2R\cos\angle QAP=AY$, and furthermore $\overline{AH}$ and $\overline{AY}$ are isogonal, so $H$ lies on $\omega$, as desired.
    \end{proof}

    Since $D$ and $E$ are unique, we are done. 
\end{customenv}
\begin{customenv}{Second solution to part (b), using complex numbers}[Michael Ma]
    Let $L$ be the midpoint of arc $\widehat{BC}$ not containing $A$. Toss on the complex plane, with lowercase letters denoting complex numbers; set $a=u^2$, $b=v^2$, $c=w^2$, and $l=-vw$. We first prove a crucial claim:
    \setcounter{iclaim}0
    \begin{iclaim}
        $\overline{DH}\parallel\overline{BL}$ and $\overline{EH}\parallel\overline{CL}$.
    \end{iclaim}
    \begin{proof}
        This is just angle chasing. \[\angle ADH=180^\circ-\angle DHA-\angle HAD=90^\circ-\angle DEA+\angle B=\tfrac12\angle A+\angle B=\angle ABL,\]
        as desired.
    \end{proof}

    Now, everything is computable. Since $\overline{AM}\perp\overline{BP}$, we have that \[am+bp=0\implies p=-\frac{am}b=\frac{u^2w}v,\]
    and similarly $q=\tfrac{u^2v}w$.

    To compute $d$, note that we have $D\in\overline{AB}$, so $\overline d=\tfrac{u^2+v^2-d}{u^2v^2}$. Furthermore, $\overline{DH}\parallel\overline{BL}$, so \[\frac{d-h}{b-m}\in\mathbb R\implies\frac{d-u^2-v^2-w^2}{v^2+vw}\in\mathbb R.\]
    Now, we have that
    \begin{align*}
        \frac{d-u^2-v^2-w^2}{v^2+vw}&=\frac{\frac{u^2+v^2-d}{u^2v^2}-\frac1{u^2}-\frac1{v^2}-\frac1{w^2}}{\frac1{v^2}+\frac1{vw}}\\
        \implies d-u^2-v^2-w^2&=v^3w\left(-\frac d{u^2v^2}-\frac1{w^2}\right)\\
        \implies d&=\frac{u^2\left(u^2w+v^2w+w^3-v^3\right)}{w\left(u^2+vw\right)}.
    \end{align*}
    Similaly, \[e=\frac{u^2\left(u^2v+vw^2+v^3-w^3\right)}{v\left(u^2+vw\right)}.\]
    Notice that \[d-p=\frac{(w-v)\left(v^3-u^2w\right)u^2}{vw\left(u^2+vw\right)}\implies-\frac{d-p}{e-q}=\frac{v^2-u^2w}{w^2-u^2v}.\]
    Finally, we want \[x=-\frac{b-a}{c-a}\bigg/\frac{d-p}{e-q}=\frac{\left(v^2-u^2\right)\left(w^3-u^2v\right)}{\left(w^2-u^2\right)\left(v^3-u^2w\right)}\]
    to be real, but \[\overline x=\left(\frac{u^2-v^2}{u^2v^2}\cdot\frac{u^2v-w^3}{u^2vw^3}\right)\bigg/\left(\frac{u^2-w^2}{u^2w^2}\cdot\frac{u^2w-v^3}{u^2v^3w}\right)=x,\]
    and we are done. 
\end{customenv}
\begin{customenv}{Third solution to part (b), using orthocenter reflections}
    Let $H_B$ and $H_C$ denote the reflections of $H$ across $\seg{CA}$ and $\seg{AB}$ respectively.
    \setcounter{iclaim}0
    \begin{iclaim}
        $H_C$ lies on $\seg{PD}$ and $H_B$ lies on $\seg{QE}$.
    \end{iclaim}
    \begin{proof}
        Since $\triangle ADE\sim\triangle APQ$, \[\da AH_CD=\da DHA=\da DEA=\da AQP=\da AH_CP,\]
        as requested.
    \end{proof}
    \begin{iclaim}
        Let $Y$ denote the reflection of $H$ across the $A$-angle bisector. $Y$ lies on $\omega$ and also on $\seg{PD}$ and $\seg{QE}$.
    \end{iclaim}
    \begin{proof}
        Clearly $Y$ lies on $\omega$, as $DEYH$ is an isosceles trapezoid. Check that \[\da YDE=\da YAE=\da DAH=\da H_CAB=\da H_CPB=\da PDE,\]
        as desired.
    \end{proof}

    Hence, $Y=\seg{PD}\cap\seg{QE}$ lies on $\omega$, and we are done. 
\end{customenv}
