desc: f(x)+f(1/x)=1 and f(2x)=2f(f(x))
source: Ireland 1991/9
tags: [2019-10, oly, medium, alg, FE, induction, nice, favorite, waltz]

---

Find all functions $f:\mathbb Q_{>0}\to\mathbb Q_{>0}$ such that
\begin{enumerate}[label=(\roman*),itemsep=0em]
    \item $f(x)+f(1/x)=1$ and
    \item $f(2x)=2f(f(x))$ for all $x\in\mathbb Q_{>0}$.
\end{enumerate}

---

The answer is $f(x)\equiv\tfrac x{x+1}$, which can be easily verified to work. We will show $f(\tfrac mn)=\frac m{m+n}$ for all relatively prime positive integers $m$ and $n$ by strong induction on $m+n$. The base case of $m+n=2$ is obvious: $f(1)=\tfrac12$ by $x=1$ in the first condition.

I claim $f(\tfrac mn)$ is uniquely determined. Since we know $f(\tfrac mn)=\tfrac m{m+n}$ is a valid solution, this will prove the desired conclusion.

\bigskip

\textbf{Case 1: $m+n$ is even.}     If $m$ and $n$ are both even, there is nothing to prove; henceforth assume they are both odd. If $m<n$, then \[f\left(\frac mn\right)=f\left(f\left(\frac m{n-m}\right)\right)=\frac12f\left(\frac m{\frac12(n-m)}\right),\]
which is uniquely determined. Otherwise $f(\tfrac mn)=1-f(\tfrac nm)$, so the hypothesis holds for all such $\tfrac mn$.

\bigskip

\textbf{Case 2: $m+n$ is odd.}     If $m$ is even and $n$ is odd, \[f\left(\frac mn\right)=2f\left(f\left(\frac{m/2}n\right)\right)=2f\left(\frac{m/2}{m/2+n}\right).\]
Let $K=m+n$ and $g(j)=f(\tfrac j{K-j})$. By the above equation, $g(j)=2g(\tfrac j2)$ for $j$ even, and by the problem statement, $g(j)+g(K-j)=1$. The task is to show $g(j)$ is uniquely determined by these two relations.

We have either $g(j)=2g(\tfrac j2)$ or $g(j)=1-g(K-j)=1-2g(\tfrac{K-j}2)$ for each $j$. The coefficient of the term wrapped in $g$ increases by a factor of $2$ each iteration, and by Euler's theorem we eventually obtain $g(j)$ is a linear function in either $g(j)$ or $g(K-j)$, with slope a power of two greater than $1$. Recall that $g(K-j)=1-g(j)$, so $g(j)$ is uniquely determined, the end.

