desc: Tangentially defined circles concur on BC
source: ELMO 2012/1
tags: [2019-11, oly, trivial, geo, pop]

---

In acute triangle $ABC$, let $D$, $E$, $F$ denote the feet of the altitudes from $A$, $B$, $C$, respectively, and let $\omega$ be the circumcircle of $\triangle AEF$. Let $\omega_1$ and $\omega_2$ be the circles through $D$ tangent to $\omega$ at $E$ and $F$ respectively. Show that $\omega_1$ and $\omega_2$ meet at a point $P$ on line $BC$ other than $D$.

---

Let $M$ be the midpoint of $\seg{BC}$, so that $\seg{ME}$ and $\seg{MF}$ are tangent to $(AEF)$.
\begin{center}
    \begin{asy}
        size(7cm); defaultpen(fontsize(10pt));

        pair A,B,C,M,D,EE,F,P;
        A=dir(130);
        B=dir(210);
        C=dir(330);
        M=(B+C)/2;
        D=foot(A,B,C);
        EE=foot(B,C,A);
        F=foot(C,A,B);
        P=extension(B,C,EE,F);

        draw(circumcircle(P,D,EE),gray);
        draw(circumcircle(P,D,F),gray);
        draw(P--EE,gray);
        draw(circumcircle(A,EE,F));
        draw(F--M--EE);
        draw(P--C--A--B);

        dot("$A$",A,N);
        dot("$B$",B,S);
        dot("$C$",C,SE);
        dot("$M$",M,S);
        dot("$D$",D,SE);
        dot("$E$",EE,dir(30));
        dot("$F$",F,dir(115));
        dot("$P$",P,SW);
    \end{asy}
\end{center}
Then $M$ is the radical center of $\omega$, $\omega_1$, $\omega_2$, so it lies on the radical axis of $\omega_1$, $\omega_2$. Point $D$ lies on this radical axis, so this radical axis is line $BC$.
\begin{remark}
    We can show that $P=\seg{BC}\cap\seg{EF}$ is the harmonic conjugate of $D$ wrt.\ $\seg{BC}$. Indeed, by the Midpoint of Harmonic Bundles lemma, $MD\cdot MP=MB^2=ME^2$ since $M$ is the center of $(BCEF)$. Similarly, $MD\cdot MP=MF^2$, so $P\ne D$ lies on both circles $\omega_1$ and $\omega_2$.
\end{remark}
